\documentclass[a4paper, 11pt]{article}
\usepackage{blindtext}
\usepackage[T1]{fontenc}
\usepackage[utf8]{inputenc}
\usepackage{amsmath, amsthm, amssymb, amsfonts}
\usepackage{bm}
\usepackage{graphics}
\usepackage{tikz-cd}
\usepackage{enumerate}
\usepackage{enumitem}
\usepackage{romannum}
\usepackage[a4paper,top=3cm,bottom=2cm,left=3cm,right=3cm,marginparwidth=1.75cm]{geometry}
\newtheorem{difinition}{Definition}

\newtheorem{theorem}{Theorem}
\theoremstyle{definition}
%\theorembodyfont{\normalfont}

\theoremstyle{remark}
\newtheorem*{remark}{Remark}
\newenvironment{myremark}
{\renewcommand\qedsymbol{$ $}\begin{proof}[$\mathbf{REMARK}$]}
  {\end{proof}}

\newenvironment{myprf}
{\renewcommand\qedsymbol{$ $}\begin{proof}[$\mathbf{Proof}$]}
  {\end{proof}}

\newenvironment{mydef}
{\renewcommand\qedsymbol{$ $}\begin{proof}[$\mathbf{Definition}$]}
  {\end{proof}}

\newenvironment{myexam}
{\renewcommand\qedsymbol{$ $}\begin{proof}[$\mathbf{EXAMPLE}$]}
  {\end{proof}}

\newenvironment{myprop}
{\renewcommand\qedsymbol{$ $}\begin{proof}[$\mathbf{Proposition}$]}
  {\end{proof}}

\theoremstyle{definition}
\newtheorem{lemma}[theorem]{Lemma}

\title{\textbf{Compactness of Topological Space}}
\date{December 1st, 2020}
\begin{document}
\maketitle
\section{Compact Spaces} 

\indent Compact Spaces is a kind of special topological spaces. In such a
topological space, a local property may be true in the whole space. In mathmatical
analysis, we have already seen some compact spaces, for example, the closed 
interval. A basis but important theorem in analysis says that a continuous 
function must be bounded in a closed interval. The key point of the proof to
this theorem is the concept of $\textit{compactness}$.
\vspace{0.5cm}
\begin{mydef}
        Let $X$ be topological space. A collection $\mathcal{A}$ of subsets of
        $X$ is said to be a $\textbf{\textit{covering}}$ of $X$, if their
        union equals to $X$.If elements of this collection are
        all open sets, then $\mathcal{A}$ is said to be an $\textbf{\textit{open
        covering}}$ of $X$.
\end{mydef}
\vspace{0.5cm}
\begin{mydef}
        A topological space $X$ is said to be $\textbf{\textit{compact}}$, 
        if every open covering of $X$ has finite subcollection that covers $X$.
\end{mydef}
Finite subcollection means we can pick up finite many open set to form a
new collection. Here are some examples about compact topological spaces.
\begin{myexam}
       The following subspace of $\mathbb{R}$ is compact:
       $$
       X=\{0\}\cup \{1/n\mid n\in \mathbb{N}^{+}\}
       $$
       \indent Let $\mathcal{A}$ be an open covering of $X$, we will pick up 
       finite of them to cover $X$. Since $\displaystyle
       0\in X=\bigcup_{U\in \mathcal{A}} U$, there must be some open set that
       contains 0, we pick up this open set. Notice that 0 is a limit point of 
       $X\setminus\{0\}$,  there are only finite many points not included in this
       open set. So we can pick finite many open sets to cover them.
\end{myexam}
\begin{myexam}
        Consider $\mathbb{R}$ and general topology on $\mathbb{R}$, then $(0,1]$
        is not compact. The reason is that we have an open cover:
        $$
        \mathcal{A}=\{(1/n,1] \mid n\in \mathbb{N}^{+}\}
        $$ but $\mathcal{A}$ has no finite sub-cover.
\end{myexam}
\vspace{0.5cm}
\begin{mydef}
      Let $X$ be a topological space and $Y$ a subset of $X$. $Y$ is said to be 
      a $\textbf{\textit{compact set}}$ (of $X$), 
      if any open covering of $Y$ has finite subcover.
\end{mydef}
\begin{myremark}
        We say a collection $\mathcal{A}$ of $X$ is a cover of $Y$, if:
        $$
        Y\subset \bigcup_{U\in \mathcal{A}}U
        $$ and $\mathcal{A}$ is said to be an open cover iff every elements of
        $\mathcal{A}$ is an open set.\\
        \indent Different from the definition of $\textit{compact space}$, a
        $\textit{compact set}$ specifies the compactness of a subset,but there
        are no substantial difference between this two definitions. We will
        demonstrate you a theorem proof(very easy, just follow the 
        definition):
\end{myremark}
\begin{theorem}
        Let $X$ be a topological space. $Y$ is a compact set if and only if $Y$
        is compact space under subspace topology.
\end{theorem}
\noindent
Now we may not distinguish $\textit{compact set}$ and $\textit{compact space}$ 
deliberately.
\vspace{0.5cm}
\begin{theorem}
        Every closed set of a compact space is a compact set, thus a compact
        space.
\end{theorem}
\begin{myprf}
       Let $X$ be a compact space, and $Y$ a closed set of $X$. We shall see that
       every open cover of $Y$ has a finite sub-cover.\\
       \indent Let $\mathcal{A}=\{U_{\alpha}\mid \alpha\in I\}$ is an open cover
       of $Y$, st. $Y\subset \displaystyle \bigcup_{\alpha\in I} U_{\alpha}$.
       Then $\displaystyle \bigcup_{\alpha\in I}U_{\alpha} \cup Y^{c}$ is an open
       covering of $X$ as $Y$ is closed in $X$($Y^{c}=X\setminus Y$). Thus there
       are finite sub-cover of $X$ for $X$ is compact.  Let: $
       \displaystyle \bigcup_{i=1}^{n}U_i=X$,  which is also a finite sub-cover
       of $Y$. If $Y^{c}$ is one of these open sets, kick it out, and we get a 
       finite sub-cover from $\mathcal{A}$ for $Y$, which concludes that $Y$ 
       is compact set in $X$.
\end{myprf}
\vspace{0.5cm}

In mathmatical analysis, we have proved so-called "finite-covering theorem" for
closed interval, therefore every closed interval of $\mathbb{R}$ is 
compact. One may naively think that compact set must be closed set. This
is not true: Consider $X=\{0, 1, 2\}, \mathcal{T}=\{\varnothing, X, {1,2}\}$. 
Then $\{1\}$ is compact as there are totally finite open set, however, 
$\{1\}$ is not closed. But we will see this assertion is true in some more 
particular space.
\vspace{0.4cm}

\begin{theorem}
        Every compact set of a Hausdorff space is closed.
\end{theorem}
\begin{myprf}
        Let $X$ be a Hausdorff space and $Y$ a compact set of $X$. We will show
        that $Y$ is closed. We will prove that every elements not in $Y$ is
        also not a limit point of $Y$.\\
        \indent Fix a point of $Y^{c}$, say $x$. For any $y\in Y$, there exists
        two open set $U_y, V_y$, such that:$x\in U_y, y\in V_y$ but 
        $U_y\cap V_y=\varnothing$ for $X$ is Hausdorff. $\{V_y\mid y\in Y\}$
        is obviously an open covering of $Y$. Then there are finite sub-cover,say
        $\{V_{y_1}, V_{y_2},...,V_{y_n}\}$. Consider $\displaystyle U=
        \bigcup_{i=1}U_{y_i}$. Then $U$ is an open set that contains $x$, and it's 
        easy to prove that $U\cap V_{y_i}=\varnothing, i=1,2,...,n$, thus 
        $U\cap Y=\varnothing$. This indicates that $x$ is not a limit point of 
        $Y$. Hence, $\bar{Y}=Y$ and $Y$ is closed. The proof is done.
\end{myprf}
The following theorem is a direct corollary of $\textbf{\textit{Theorem 2}}$(not
$\textbf{\textit{Theorem 3}})$. It discusses what a compact set is in a more 
special space.
\begin{theorem}($\textbf{Heine–Borel theorem}$)
        A subset of $\mathbb{R}^n$ is compact if and only if it's closed and 
        bounded
\end{theorem}
\begin{myprf}
        We will just give a sketch of proof for this theorem.\\
        ($\Rightarrow$): If $Y$ is compact, it must be closed as $\mathcal{R}^n$
        is Hausdorff. To see $Y$ is bounded, we consider the distance between any
        point of $Y$ and the original point $\mathbf{0}$. 
        For any $y\in Y$, there is an open
        ball $B(y, r_y)$ that contains $y$ but doesn't contain $\mathbf{0}$. Then 
        $Y\subset\bigcup_{y\in Y} B_y$. We can pick finite of these open ball to
        cover $Y$, say $B(y_1,r_{y_1}),....,B(y_n,r_{y_n})$. 
        Then every point of $Y$ must be
        one of these open balls. Assume $y\in B(y_1,r_{y_1})$ then 
        $d(y,\mathbf{0})\leq d(y, y_1)+d(y_1, \mathbf{0})$. Note that there are
        only finite many open ball, then we can let $M=\max r_{y_i}, N=
        \max d(y_i, \mathbf{0})$. Thus $d(y,\mathbf{0}) \leq M+N$ and the proof
        is done.\\
        \noindent
        ($\Leftarrow$) If $Y$ is bounded, then $Y\subset [a_1,b_1]\times\cdots
        [a_n,b_n]$ for some $a_i, b_i$. We denote this cubic with $U$.
        Note that $Y$ is also closed in U because $U\cap Y$ is closed in $U$.
        We assert that $U$ is compact, and by $\textit{theorem 3}$, $Y$
        is compact.\\
        ( There are many methods to proving $U$ is compact. A direct way
        is use the same technique of proving compactness of closed interval.)

\end{myprf}
\end{document}
