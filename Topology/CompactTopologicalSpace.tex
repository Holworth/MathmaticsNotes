\documentclass[a4paper, 11pt]{article}
\usepackage{blindtext}
\usepackage[T1]{fontenc}
\usepackage[utf8]{inputenc}
\usepackage{amsmath, amsthm, amssymb, amsfonts}
\usepackage{bm}
\usepackage{graphics}
\usepackage{tikz-cd}
\usepackage{enumerate}
\usepackage{enumitem}
\usepackage{romannum}
\usepackage[a4paper,top=3cm,bottom=2cm,left=3cm,right=3cm,marginparwidth=1.75cm]{geometry}
\newtheorem{difinition}{Definition}

\newtheorem{theorem}{Theorem}
\theoremstyle{definition}
%\theorembodyfont{\normalfont}

\theoremstyle{remark}
\newtheorem*{remark}{Remark}
\newenvironment{myremark}
{\renewcommand\qedsymbol{$ $}\begin{proof}[$\mathbf{REMARK}$]}
  {\end{proof}}

\newenvironment{myprf}
{\renewcommand\qedsymbol{$ $}\begin{proof}[$\mathbf{Proof}$]}
  {\end{proof}}

\newenvironment{mydef}
{\renewcommand\qedsymbol{$ $}\begin{proof}[$\mathbf{Definition}$]}
  {\end{proof}}

\newenvironment{myexam}
{\renewcommand\qedsymbol{$ $}\begin{proof}[$\mathbf{EXAMPLE}$]}
  {\end{proof}}

\newenvironment{myprop}
{\renewcommand\qedsymbol{$ $}\begin{proof}[$\mathbf{Proposition}$]}
  {\end{proof}}

\theoremstyle{definition}
\newtheorem{lemma}[theorem]{Lemma}

\title{\textbf{Compactness of Topological Space}}
\date{December 1st, 2020}
\begin{document}
\maketitle
\section{Compact Spaces} 

\indent Compact Spaces is a kind of special topological spaces. In such a
topological space, a local property may be true in the whole space. In mathmatical
analysis, we have already seen some compact spaces, for example, the closed 
interval. A basis but important theorem in analysis says that a continuous 
function must be bounded in a closed interval. The key point of the proof to
this theorem is the concept of $\textit{compactness}$.
\vspace{0.5cm}
\begin{mydef}
        Let $X$ be topological space. A collection $\mathcal{A}$ of subsets of
        $X$ is said to be a $\textbf{\textit{covering}}$ of $X$, if their
        union equals to $X$.If elements of this collection are
        all open sets, then $\mathcal{A}$ is said to be an $\textbf{\textit{open
        covering}}$ of $X$.
\end{mydef}
\vspace{0.5cm}
\begin{mydef}
        A topological space $X$ is said to be $\textbf{\textit{compact}}$, 
        if every open covering of $X$ has finite subcollection that covers $X$.
\end{mydef}
Finite subcollection means we can pick up finite many open set to form a
new collection. Here are some examples about compact topological spaces.
\begin{myexam}
       The following subspace of $\mathbb{R}$ is compact:
       $$
       X=\{0\}\cup \{1/n\mid n\in \mathbb{N}^{+}\}
       $$
       \indent Let $\mathcal{A}$ be an open covering of $X$, we will pick up 
       finite of them to cover $X$. Since $\displaystyle
       0\in X=\bigcup_{U\in \mathcal{A}} U$, there must be some open set that
       contains 0, we pick up this open set. Notice that 0 is a limit point of 
       $X\setminus\{0\}$,  there are only finite many points not included in this
       open set. So we can pick finite many open sets to cover them.
\end{myexam}
\vspace{0.5cm}
\begin{mydef}
      Let $X$ be a topological space and $Y$ a subset of $X$. $Y$ is said to be 
      a $\textbf{\textit{compact set}}$ (of $X$), 
      if any open covering of $Y$ has finite subcover.
\end{mydef}
\begin{myremark}
        We say a collection $\mathcal{A}$ of $X$ is a cover of $Y$, if:
        $$
        Y\subset \bigcup_{U\in \mathcal{A}}U
        $$ and $\mathcal{A}$ is said to be an open cover iff every elements of
        $\mathcal{A}$ is an open set.\\
        \indent Different from the definition of $\textit{compact space}$, a
        $\textit{compact set}$ specifies the compactness of a subset,but there
        are no substantial difference between this two definitions. We will
        demonstrate you a theorem proof(very easy, just follow the 
        definition):
\end{myremark}
\begin{theorem}
        Let $X$ be a topological space. $Y$ is a compact set if and only if $Y$
        is compact space under subspace topology.
\end{theorem}
\noindent
Now we may not distinguish $\textit{compact set}$ and $\textit{compact space}$ 
deliberately.
\end{document}
