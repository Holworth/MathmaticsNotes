\documentclass[a4paper, 11pt]{article}
\usepackage{blindtext}
\usepackage[T1]{fontenc}
\usepackage[utf8]{inputenc}
\usepackage{amsmath, amsthm, amssymb, amsfonts}
\usepackage{bm}
\usepackage{graphics}
\usepackage{tikz-cd}
\usepackage{enumerate}
\usepackage{enumitem}
\usepackage{romannum}
\usepackage[a4paper,top=3cm,bottom=2cm,left=3cm,right=3cm,marginparwidth=1.75cm]{geometry}
\newtheorem{difinition}{Definition}

\newtheorem{theorem}{Theorem}

\theoremstyle{remark}
\newtheorem*{remark}{Remark}
\newenvironment{myremark}
{\renewcommand\qedsymbol{$ $}\begin{proof}[$\mathbf{REMARK}$]}
  {\end{proof}}

\newenvironment{myprf}
{\renewcommand\qedsymbol{$ $}\begin{proof}[$\mathbf{Proof}$]}
  {\end{proof}}

\newenvironment{mydef}
{\renewcommand\qedsymbol{$ $}\begin{proof}[$\mathbf{Definition}$]}
  {\end{proof}}

\newenvironment{myexam}
{\renewcommand\qedsymbol{$ $}\begin{proof}[$\mathrm{EXAMPLE}$]}
  {\end{proof}}

\newenvironment{myprop}
{\renewcommand\qedsymbol{$ $}\begin{proof}[$\mathbf{Proposition}$]}
  {\end{proof}}

\theoremstyle{definition}
\newtheorem{lemma}[theorem]{Lemma}

\title{Topological Spaces}
\date{October 26th, 2020}
\begin{document}
\maketitle
\section{Topological Spaces}
\begin{mydef}
        A $\mathbf{topology}$ on a set $X$ is a collection $\mathcal{T}$of subsets of $X$ having the 
        following properties:\\
        \indent(1) $\varnothing$ and $X$ are in $\mathcal{T}$\\
        \indent(2) For any subcollection of $\mathcal{T}$, indexed by set
        $I$, we have: $\displaystyle \bigcup_{\alpha\in I}U_{\alpha} \in 
        \mathcal{T}$\\
        \indent(3) For any finite subcollection of $\mathcal{T}$ with $n$ 
        elements, we have: $\displaystyle \bigcap_{i=1}^{n}U_{i}\in \mathcal{T}$\\
        A set for which a topology $\mathcal{T}$ is specified is called a 
        $\mathbf{topological\;space}$. And the element of $\mathcal{T}$ is called 
        $\mathbf{Open\;Set}$
\end{mydef}
        With the element of $\mathcal{T}$ is defined as open set, we could say a topology
        is a collection of subsets of $X$ such that $\varnothing$ and $X$ itself are open
        and satisfies that arbitrary unions and finite intersections of open sets are 
        open. We often write set $X$ and its topology $\mathcal{T}$ as the ordered pair:
        $(X,\mathcal{T})$. And when we say :"Let XXX be open sets", that means we defined
        a topology on $X$ and $\mathcal{T}$ consists the subsets mentioned above.
 \begin{myexam}
        If $X$ is any set, the collection of all subsets of $X$ is a topology on $X$, 
        called $\mathbf{discrete\;topology}$. The collection which has only $\varnothing$
        and $X$ itself is called $\mathbf{trivial\;topology}$.
 \end{myexam}
 
 \begin{myexam}
        Let $X$ be a set; let $\mathcal{T}_{f}$ be the collectino of all subsets $U$ of
        $X$ such that $X-U$ is either finite or all of $X$. Then $\mathcal{T}_{f}$ is 
        a topology of $X$, called $\bm{finite\;complement\;topology}$. Note that
        $varnothing=U-U$ is finite and $U=U-\varnothing$, therefore we have $\varnothing$
        and $U$ belong to $\mathcal{T}_{f}$. Let $\{U_{\alpha}\}$ be a subcollection of 
        $\mathcal{T}$ indexed by $I$. Then we have:
        $$
        X-\bigcup U_{\alpha}=\bigcap(X-U_{\alpha})
        $$ Since each $X-U_{\alpha}$ is finite, we have $X-\bigcup U_{\alpha}$ is finite. If
        $U_1,...,U_n \in \mathcal{T}_{f}$. Then:
        $$
        X-\bigcap_{i=1}^{n} U_i=\bigcup_{i=1}^{n}(X-U_i)
        $$ Since each $X-U_i$ is finite, the finite union of sets with finite cardinal 
        numbers are also finite. Thus $\displaystyle \bigcap_{i=1}^{n} U_i\in \mathcal{T}_{f}$\\
        \indent In conclusion, $\mathcal{T}_{f}$ is a topology on set $X$.
         
 \end{myexam}
 
 \begin{myexam} 
Let $X$ be a set and $\mathcal{T}$ a topology on $X$. If $Y$ is a subset of $U$. We
define the following collection:
$$ 
\mathcal{T}_{Y} = \{Y\cap U\mid U\in \mathrm{T}\}
$$ It is easy to see that $\mathcal{T}_{Y}$ is a topology on $Y$:
$$
\varnothing = \varnothing \cap Y\;\;\;\;Y=X\cap Y
$$If $\{V_{\alpha}\}$ is a subcollection of $\mathcal{T}_{Y}$, then each $V_{\alpha}$
could be written as $U_{\alpha}\cap Y$, we have:
$$ 
\bigcup{V_{\alpha}} = \bigcup({U_{\alpha}}\cap Y) = (\bigcup U_{\alpha}) \cap Y
$$ Note that $\displaystyle \bigcup U_{\alpha}$ is in $\mathcal{T}$,hence we have 
$\displaystyle \bigcup V_{\alpha}\in \mathcal{T}_{Y}$. \\
If $ V_i=U_i\bigcap Y,i=1,2,...,n$ is a finite collection of $\mathcal{T}_{Y}$. Then:
$$
\bigcap_{i=1}^{n} V_i=\bigcap_{i=1}^{n} (U_i\cap Y) =(\bigcap_{i=1}^{n} U_i)\bigcap Y
$$ Note that $\displaystyle \bigcap_{i=1}^{n} U_i\in \mathcal{T}$, thus we have 
$\displaystyle \bigcap_{i=1}^{n}V_i\in \mathcal{T}_{Y}$.
The above new collection consists of the intersection of $Y$ and open sets are called 
$\bm{subspace\; topology}$, and therefore, $Y$ is a topological space.
\end{myexam}

\begin{myremark}
        It is easy to see that a set could be assigned with different topology. A typical
        example is the discrete topology and trivial topology of the same set $X$. These
        two topology represents different topological structure. A trivial topology looks
        like a steel while discrete topology is fine enough to make generate any subset
        of $X$.
\end{myremark}

\vspace{0.5cm}
\begin{mydef}
        Suppose $\mathcal{T}$ and $\mathcal{T}^{'}$ are two topologies on a given set
        $X$. If $\mathcal{T}\subset\mathcal{T}^{'}(\mathcal{T}\subsetneqq\mathcal{T}^{'})$
        ,we say that $\mathcal{T}^{'}$ is  
        $\bm{finer}(\bm{strictly\;finner})$ than $\mathcal{T}$, or $\mathcal{T}$ is 
        $\bm{coarser}(\bm{stricly\;coarser})$ than $\mathcal{T}^{'}$. We say
        $\mathcal{T}$ is $\bm{comparable}$ with $\mathcal{T}^{'}$ if either
        $\mathcal{T}\subset \mathcal{T}^{'}$ or $\mathcal{T}^{'}\subset \mathcal{T}$
\end{mydef}
Sometimes we also say that $\mathcal{T}^{'}$ is larger than $\mathcal{T}$ or $\mathcal{T}$
is smaller than $\mathcal{T}^{'}$, but not as vivid as finer.

\section{Closed Sets and Limit Point}
\subsection{Closed Set}
\begin{mydef}
        Let $(X,\mathcal{T})$ be a topological space. We say a subset $A$ of $X$ is 
        $\bm{closed}$ if $X-A$ is open.
\end{mydef}

\begin{myexam}
        Let $(X,\mathcal{T})$ be a topological space and $\mathcal{T}$ be the discrete 
        topology, then any subset of $X$ is a closed set. On the other hand, let
        $\mathcal{T}$ be trivial topology, then any subset that is neither $\varnothing$
        nor $X$ is neither open nor closed.
\end{myexam}

\begin{myexam}
        Let $(\mathbb{R}^{2},\mathcal{T})$ be a topological space and $\mathcal{T}$ 
        generated by all open ball. And consider the set:
        $$
        \{(x,y)\mid x\ge 0,y\ge 0\}
        $$The set is closed as its complement is:
        $$
        (-\infty,0)\times \mathbb{R}\cup \mathbb{R}\times(-\infty,0)
        $$ And each of them are open.
\end{myexam}

\begin{myexam}
        Let $(\mathbb{R}, \mathcal{T})$ be a topological space with topology $\mathcal{T}$
        consists of all open sets under the metric space $(\mathbb{R}, \mathrm{d})$.
        Consider $Y=[0,1]\cup(2,3)$ and the subspace topology.\\
        We claim hat $[0,1]$ is an open set of $Y$, because $[0,1]=(-1,\frac{3}{2})\cap
        Y$. Similarly, $(2,3)$ is also open in $Y$. And the complement of each of them is
        another interval, therefore $[0,1]$ and $(2,3)$ are both open and closed.
\end{myexam}

\begin{myremark}
        By these three examples, we could see that a subset of $X$ can be open, or closed,
        or both, or neither. In addtition, we see that a subset is open(closed) or not depends on
        the whole space you consider: $[0,1]$ in EXAMPLE3 is not open in $\mathbb{R}$ but
        open in $Y$.$(2,3)$ is not closed in $\mathbb{R}$ but closed in $Y$.
\end{myremark}
\begin{theorem}
        Let $X$ be a topology space. Then the following conditions hold:\\
        $\mathrm{(1)}$ $\varnothing$ and $X$ are closed\\
        $\mathrm{(2)}$ For any collection of closed set $\{V_{\alpha}\mid \alpha\in I\}$, we have
        $\displaystyle \bigcap_{\alpha\in I}V_{\alpha}$ is closed\\
        $\mathrm{(3)}$ The intersection of any finite many closed sets are closed.\\
\end{theorem}
\begin{myprf}
        (1) is trivial with $\varnothing=X-X$ and $X=X-\varnothing$.\\
        As for (2), notice that :\\
        $$
        \bigcap_{\alpha\in I}V_{\alpha}=\bigcap_{\alpha\in I}U_{\alpha}^{c}=
        (\bigcup_{\alpha\in I}U_{\alpha})^{c}
        $$ where $U_{\alpha}$ is an open set. And we denote $X-U_{\alpha}$ with 
        $U_{\alpha}^{c}$. (3) follows the same way with the fact that:
        $$\bigcup_{i=1}^{n} V_{\alpha}=\bigcup_{i=1}^{n} U_{\alpha}^{c}=
        (\bigcap_{i=1}^{n} U_{\alpha})^{c}$$
\end{myprf}

\begin{theorem}
    Let Y be a subsapce of X. Then a set A is closed in Y if and only if it equals the intersection of a closed set of X with Y.        
\end{theorem}
\begin{myprf}
        Consider the subspace topology of $Y$ and let $V_{Y}$ is a closed set under such
        subspace topology. Then we have $V_{Y}=Y-U_{Y}$ for some open set
        $U_{Y}$ in $Y$. With the definition of subspace topology, we have 
        $U_{Y}=U\cap Y$ with $U$ an open set in $X$. Then $V_{Y}=Y-U_{Y}
        =Y-U\cap Y=Y-U=Y\cap (X-U)$ where $(X-U)$ is closed in $X$. Therefore 
        if $V_{Y}$ is closed in $Y$, then $V_{Y}$ is intersection of $Y$ and a 
        closed set in $X$. \\
        On the other hand, if $V_{Y}=Y\cap V$ for some closed 
        set $V$ of $X$. We have $V_{Y}=Y\cap (X-U)=Y-U=Y-(Y\cap U)$, which is 
        closed in $Y$
\end{myprf}
\begin{myremark}
        General speaking, a set that is closed in a subspace may not be closed in the larger 
        topological space. For example, let $X=\mathbb{R}$ and open set consists
        of conventional open set in $\mathbb{R}$. Consider the subspace $Y$ generated
        by the intersection of $[0,1)$ and $X$. Then notice that $[0,\cfrac{1}{2})$
        is open in $Y$ as $[0,\cfrac{1}{2})=(-1,\cfrac{1}{2})\cap [0,1)$. Therefore
        $Y-[0,\cfrac{1}{2})=[\cfrac{1}{2},1)$ is closed in $Y$, however, it's not
        closed in $\mathbb{R}$.\\
        But we have the following theorem explained the so called "transitivity"
        of closed property:
        \vspace{0.1cm}
        \begin{theorem}
               Let Y be a subspace of X. If A is closed in Y and Y is closed in X,
               then A is closed in X.
        \end{theorem}
        \begin{myprf}
               By theorem 2, $A=Y\cap V$ with $V$ closed in $X$. Therefore, $A$
               is closed in $X$ by the fact that the intersection of two closed 
               sets is closed.
        \end{myprf}
\end{myremark}
\subsection{Limit Point and Closure}
\begin{mydef}
    Let $X$ be a topological space and $A$ a subset of $X$. An element $x$ of $X$
    is said to be $\bm{limit point}$ of $A$ if: for every open set $U$ that 
    contains $x$, $U\cap A\neq \varnothing$ or $\{x\}$.
\end{mydef}
\begin{mydef}
        Let $A$ be a subset of the topologival space $X$; let $A^{'}$ be the set
        of all limit points of $A$, we define the closure of $A$ as the union
        of $A$ and $A^{'}$, denoted by $\bar{A}$. Which is:
        $$
        \bar{A}=A\cup A^{'}
        $$
\end{mydef}
\begin{theorem}
    Let X be a topological space. Then A is closed in  X if and only if:$\bar{A}=A$
\end{theorem}
\begin{myprf}
        $\Leftarrow$: If $\bar{A}=A$, we need to show that $A$ is closed, or to
        show that
        $X-A$ is open. For any element $x\in X-A$, $x$ is neither an element of
        $A$ nor the limit point of $A$. $x\notin A^{'}$ means there is some 
        open set $U$ that contains $x$ but $U\cap A=\varnothing$ or
        $\{x\}$. Now that $x\notin A$, we have $U\cap A=\varnothing$. For any
        $x\in X-A$, we have such open set $U_x$. And thus:
        $$
        X-A=\bigcup_{x\in (X-A)} U_x
        $$ is union of open set in $X$, therefore an open set. Hence we have $A$
        is closed.\\
        \\
        $\Rightarrow$: If $A$ is closed. To prove $A=\bar{A}$, we only need to 
        show that $A^{'}\subset A$, which is: any limit point of $A$ is in $A$.
        Suppose $x$ is a limit point of $A$ but $x\in X-A$. Then notice that $X-A$ is
        an open set that contains $x$ but $(X-A)\cap A=\varnothing$, which 
        contradicts the definition of limit point. Therefore any limit point of 
        $A$ is in $A$, and hence $A=\bar{A}$.
\end{myprf}
\vspace{0.5cm}
\begin{theorem}
        Let X be a topological space and A a subset of X, then $\bar{A}$ is the
        smallest closed set that contains A.
\end{theorem}
\begin{myprf}
        The proof are divided into two parts:
        \begin{enumerate}[label=(\roman*)]
                \item $\bar{A}$ is closed.
                \item Every closed set that contains $A$ must contain $\bar{A}$.
        \end{enumerate}
        For (\romannum{2}), we only need to show that every closed set that
        contains $A$ must contain the limit point of $A$. This is easy to show:
        Let $B$ a closed set that contains $A$ and $x$ a limit point of $A$, then
        $x$ must be a limit point of $B$ as $A\subset B$. By theorem 4 and 
        the fact that $B$ is closed, we have: $x\in \bar{B}=B$. Therefore, 
        $\bar{A}\subset B$\\
        For (\romannum{1}), we only need to show that $\bar{A}=\bar{\bar{A}}$ by
        theorem 4. which is concluded as the following lemma.
\end{myprf}
\begin{lemma}
        Let $X$ be a topological space and $A$ a subset of $X$, then 
        $\bar{\bar{A}}=\bar{A}$.
\end{lemma}
\begin{myprf}
        $\bar{A}\subset \bar{\bar{A}}$ according to the definition of closure.
        As for the other side ,we need to show that the limit point of $\bar{A}$
        is in $\bar{A}$.\\
        If $x$ is a limit point of $\bar{A}$. If $x\in A$, we're done. 
        Otherwise let $U$ be any open set that contains $x$, we have:
        $$
        U\cap \bar{A}\neq \varnothing, \{x\}
        $$ We claim that $x$ is a limit point of $A$, by claiming that
        $U\cap A\neq \varnothing$(of course it can't be $\{x\}$ as $x\notin A$).
        \begin{enumerate}[label=(\roman*)]
                \item If $U\cap A\neq \varnothing$, we're done.
                \item Otherwise $U\cap A=\varnothing$ but $U\cap A^{'}\neq 
                        \varnothing$. $U\cap A^{'}\neq \varnothing$ shows that
        there is some point, say $y$, is a limit point of $A$, and $y\notin A$. 
        Therefore $U\cap A\neq \varnothing$ as $U$ is an open set
       containing $y$, this contradicts that assumption that $U\cap A=\varnothing$
        \end{enumerate}
        In conclusion, $U\cap A\neq \varnothing$ and thus $x$ is a limit point 
        of $A$ by definition.\\

        \noindent
        Both sides contains the other side, therefore we have:$\bar{\bar{A}}=
        \bar{A}$.
\end{myprf}

\noindent
By using the result of lemma 6, we may draw the conclusion of theorem 5 as 
explained in the proof.
\begin{myremark}
     The name "closure" means that $\bar{A}$ remains constant under the map 
     by mapping a set of topological space into the union of $A$ and 
     $A^{'}$. Or, as explained 
     int theorem 5, closure is the smallest closed set that contains $A$.
     Further more, we can easily prove the closure of $A$ 
     has an equvalent definition:
     $$
     \bar{A}=\bigcap_{A\subset V,V\;closed}V
     $$\\
     So far,we have actually given two ways of explaining what is a closed set
     is. One by clarifying the relationship between open set and closed set; and
     the other by using the definition of limit point. Theorem 4, 5 and lemma 6
     has showed the equivalence of these two expression, and we conclude it as:
\end{myremark}
\begin{theorem}
     Let $X$ be a topological space, a subset $A$ of $X$ is closed iff every limit
     point of $A$ is in $A$.
\end{theorem}
\begin{myprf}
       Omitted, see theorem 4.
\end{myprf}
\vspace{0.3cm}
\noindent
The following theorem describes the closure of a subset in subspace.
\begin{theorem}
       Let $X$ be a topological space and $Y$ a subspace of $X$; let $A$ be 
       a subset of $Y$; let $\bar{A}$ denote the closure of $A$ in $X$, Then 
       the closure of $A$ in $Y$ equals $\bar{A}\cap Y$.
\end{theorem}
\begin{myprf}
        By theorem 5 and its remark, we know that the closure of $A$ in $Y$, 
        denoted by $\bar{A_{Y}}$, equals to the insersection of all closed
        set in $Y$ that contains $A$. Note that any closed set in $Y$ equals 
        to the intersection of $Y$ and a closed set in $X$. Therefore 
        we have:
        $$
        \begin{aligned}
                \bar{A_Y}&=\bigcap_{A\subset V_Y,V_Y\;closed\;in\;Y}V_Y\\
                         &=\bigcap_{A\subset V\cap Y,V\;closed\;in\;X}(V\cap Y)\\
                         &=(\bigcap_{A\subset V,V\;closed\;in\;X}V)\bigcap Y\\
                         &=\bar{A_X}\bigcap Y
        \end{aligned}
        $$ 
\end{myprf} 
\begin{myremark}
\noindent
The equivalence between the second line and the third line is easy to prove with
the following claim:
$$
A\subset V,V\;closed\;in\;X\Leftrightarrow A\subset V\cap Y,V\cap Y\;closed\;in\;Y
$$
        A question is that whether the following proposition is true:
        \begin{myprop}
               Let $X$ be a topological space and $Y$ is a subspace of $X$;
               $A$ is a subset of $X$(not $Y$), then:
               $$
               \overline{A\cap Y}=\bar{A_X}\cap Y
               $$
        \end{myprop}
        \noindent
        Unfortunately, this proposition is false. But we have the left side
        subsets the right side.
\end{myremark}
\subsection{Dense Set}
\begin{mydef}
    Let $X$ be a topological space and $A$ a subset of $X$. $A$ is said to be 
    a $\bm{dense}$ set of $X$ if $\bar{A}=X$. 
\end{mydef}
In other words, a set $A$ is called dense, if any point $x$ in $X$ belongs to $A$ or $x$ is a limit point of $A$. There are many examples of dense set, the most
common one,  which is frequently mentioned in Mathematical Analysis, is that
$\mathbb{Q}$ is dense in $\mathbb{R}$. 
\begin{mydef}
        Let $X$ be a topological space, the $\bm{interior}$ of $A$ is defined as
        the union of all open sets that is contained in $A$, denoted as 
        $\mathbf{IntA}$. The element of $\mathrm{Int}A$ is called 
        $\bm{interior\; point}$
\end{mydef}
\noindent
It's easy to see the following relation between $\mathrm{Int}A$, $\bar{A}$ and
$A$:
$$
\mathrm{Int}A\subset A\subset \bar{A}
$$ 
\noindent
And further more, $A$ is open iff $\mathrm{Int}A=A$, $A$ is closed iff 
$\bar{A}=A$.
\begin{theorem}
        $X$ is a topological space and $A$ a subset of $X$.
        The following conditions are equalvalent:
        \begin{enumerate}[label=(\roman*)]
                \item   $x$ is an interior point of $A$.
                \item   There is an open set $U$, such that: $x\in U\subset A$
        \end{enumerate}
\end{theorem}
\begin{myprf}
        (i)$\Rightarrow$ (ii): \\
        $$x\in \mathrm{Int}A\Rightarrow x\in \displaystyle
        \bigcup_{U\subset A,U\;open}U$$
        Therefore, there is an open set in the right side that contains
        $x$ and it's done.\\
        (ii)$\Rightarrow$ (i):\\
        $$x\in U\subset \bigcup_{U\subset A,U\;open}U=\mathrm{Int}A$$
\end{myprf}

\end{document}

