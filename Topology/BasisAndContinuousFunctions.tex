\documentclass[a4paper, 11pt]{article}
\usepackage{blindtext}
\usepackage[T1]{fontenc}
\usepackage[utf8]{inputenc}
\usepackage{amsmath, amsthm, amssymb, amsfonts}
\usepackage{bm}
\usepackage{graphics}
\usepackage{tikz-cd}
\usepackage{enumerate}
\usepackage{enumitem}
\usepackage{romannum}
\usepackage[a4paper,top=3cm,bottom=2cm,left=3cm,right=3cm,marginparwidth=1.75cm]{geometry}
\newtheorem{difinition}{Definition}

\newtheorem{theorem}{Theorem}
\theorembodyfont{\upshape}

\theoremstyle{remark}
\newtheorem*{remark}{Remark}
\newenvironment{myremark}
{\renewcommand\qedsymbol{$ $}\begin{proof}[$\mathbf{REMARK}$]}
  {\end{proof}}

\newenvironment{myprf}
{\renewcommand\qedsymbol{$ $}\begin{proof}[$\mathbf{Proof}$]}
  {\end{proof}}

\newenvironment{mydef}
{\renewcommand\qedsymbol{$ $}\begin{proof}[$\mathbf{Definition}$]}
  {\end{proof}}

\newenvironment{myexam}
{\renewcommand\qedsymbol{$ $}\begin{proof}[$\mathrm{EXAMPLE}$]}
  {\end{proof}}

\newenvironment{myprop}
{\renewcommand\qedsymbol{$ $}\begin{proof}[$\mathbf{Proposition}$]}
  {\end{proof}}

\theoremstyle{definition}
\newtheorem{lemma}[theorem]{Lemma}

\title{$\textbf{Topology Basis And Continuous Functions}$}
\date{December 21st, 2020}
\begin{document}
       \maketitle 
\section{Basis For Topology}
\begin{mydef}$\textbf{(Topology Basis)}$ Let $(X,\mathcal{T})$ be a topological 
        space. A $\textbf{basis}$ for topology $\mathcal{T}$ is a collection
        of subsets $\mathcal{B}$ such that for any open set $U\in \mathcal{T}$,
        $$
        U=\bigcup_{\alpha\in I} B_{\alpha}, B_{\alpha}\in \mathcal{B}
        $$ In other words, any open set could be denoted as the union of a 
        collection of subsets in $\mathcal{B}$
\end{mydef}
\vspace{0.5cm}
\begin{myexam}
       Let $X=\mathbb{R}$ and $\mathcal{T}$ be the conventional topology on 
       $\mathbb{R}$. Let $\mathcal{B}$ consists of all open interval with 
       rational endpoint, which is:
        $$
        \mathcal{B}=\{(a,b)\mid a,b\in \mathbb{Q},a<b\}
        $$ Then for any $\displaystyle U \underset{open}{\subset} \mathbb{R}$, 
        and any $x\in U$, there is an interval $I$ with: $x\in I_x\subset U$ and
        therefore: $\displaystyle U=\bigcup_{x\in U} I_x$.
\end{myexam}
\vspace{0.5cm}
Note that we say $\mathcal{B}$ is a basis for some specific topology if 
$\mathcal{B}$ satisfies the above definition. Under such condition, we already 
pointed out what topology this basis is corresponding to.
The next question is: what kind of
collection of subsets could be a basis for some topology on $X$?(If we didn't
specify the topology yet)\\
\indent
Suppose there is a collection of subsets of $X$, say $\mathcal{B}$. If $B$ is 
a basis for some topology $\mathcal{T}$, then every open set in $\mathcal{T}$
could be denoted as union of elements in $\mathcal{B}$. Note that 
$X\in \mathcal{T}$ therefore $X=\displaystyle \bigcup_{\alpha\in I}B_{\alpha}$.\\
\indent
Moreover, consider $U,V\in \mathcal{T}$, and denote them as union of basis 
elements of $\mathcal{B}$:
$$
\begin{aligned}
        U&=\bigcup_{\alpha\in I}U_{\alpha},U_{\alpha}\in \mathcal{B}\\
        V&=\bigcup_{\beta\in J}V_{\beta},V_{\beta}\in \mathcal{B}
\end{aligned}
$$ Then $U\cap V$ is supposed to be denoted as union of basis elements in 
$\mathcal{B}$. However, notice that:
$$
U\cap V=\bigcup_{\alpha\in I}\bigcup_{\beta\in J}(U_{\alpha}\cap V_{\beta})
$$ Therefore, we only need that $U_{\alpha}\cap V_{\beta}$ could be denoted as union of basis elements in $\mathcal{B}$.So far, we have got a sufficient condition 
that makes $\mathcal{B}$ be a basis of some topology on $X$:
\begin{theorem}
        Let $X$ be a set and $\mathcal{B}$ be a collection of subsets in $X$. If
        $\mathcal{B}$ satifies the following two requirements, then $\mathcal{B}$
        is a basis for some topology:\\
        (\romannum{1}) For any $x\in X$, there is some $B\in \mathcal{B}$ 
        such that $x\in B\subset X$\\
        (\romannum{2}) For any $B_1,B_2\in mathcal{B}$ with $U\cap V\neq \varnothing$
        , and any $x\in B_1\cap B_2$, there is some $B_3\in \mathcal{B}$ such that
        $x\in B_3\subset B_1\cap B_2$
\end{theorem}
\begin{myprf}
       Define $\mathcal{T}=\{\bigcup_{\alpha\in I} B_{\alpha}\mid B_{\alpha}\in
       \mathcal{B}\}$, then it's easy to verify that $\mathcal{T}$ is a topology
       on $X$ and $\mathcal{B}$ is a basis of $\mathcal{T}$.
\end{myprf}
\vspace{0.5cm}
\begin{mydef}
       If $\mathcal{B}$ satisfies these two conditions in theorem 1, then we
       define $\textbf{the topology} $ $\textbf{generated by}$ $\mathcal{B}$ as
       follows: A subset $U$ of $X$ is said to be open in $X$ if for each 
       $x\in U$, there is a basis element $B\in \mathcal{B}$ such that 
       $x\in B\subset U$.
\end{mydef}
\begin{myremark}
        The topology generated by $\mathcal{B}$ is apparently equivalent
        to the topology mentioned in the proof of theorem 1. And now we know
        the topology generated by $\mathcal{B}$ is actually those sets which
        can be denoted as union of elements in $\mathcal{B}$.
\end{myremark}

\indent The word "basis" might be confusing. We know the open set of $\mathcal{B}$-generated topology equals to the union of some subsets in $\mathcal{B}$ but this
expression is not unique. However, in other subjects, for example, linear algebra
,  a basis means element could be uniquely expressed as linear combination of basis elements.\\
\indent
However, there is one thing follows the same for the basis concepts in linear 
albegra and topology, that is there might be multiple basis for the same topology,or linear space.But the following theorem, gives a sufficient condition for 
equivalent topology.
\vspace{0.5cm}
\begin{theorem}
        We say two basis are $\textbf{equalvalent}$ if they generates the same
        topology. If two basis $\mathcal{B}_1$, $\mathcal{B}_2$ satisfies the 
        following two conditions, then they are equivalent.\\
        (\romannum{1}) For any $B_1\in \mathcal{B}_1$, any $x\in B_1$, there
        is some $B_2\in \mathcal{B}_2$ such that $x\in B_2\subset B_1$\\
        (\romannum{2}) For any $B_2\in \mathcal{B}_2$, any $x\in B_2$, there
        is some $B_1\in \mathcal{B}_1$ such that $x\in B_1\subset B_2$\\
\end{theorem}
\begin{myprf}
        Trivial. The condition specifies that every element in $\mathcal{B}_1$ 
        could be expressed as union of elements in $\mathcal{B}_2$ and vice versa.
\end{myprf}
\vspace{0.5cm}
\begin{mydef}$(\textbf{subbasis})$ A $\textbf{subbasis}$ $\mathcal{S}$
        for a topology on $X$
        is a collection of subsets of $X$ whose union equals to $X$. The
        $\textbf{topology generated by the subbasis}$ $\mathcal{S}$ is defined
        to be the collection $\mathcal{T}$ of all unions of finite intersection
        of elements of $\mathcal{S}$
        
\end{mydef}

\section{Continuous Function}
\begin{mydef}
       Let $X$ and $Y$ be topological spaces. A function $f:X\rightarrow Y$ is
       said to be $\textbf{continuous}$ if for each open set $V$ of $Y$, 
       the set $f^{-1}(V)$ is an open set of $X$
\end{mydef}
It is easy to see that the continuity of a function $f$ depends not only on the
function itself, but also the specified topology of its domain and range. To 
emphasize this fact, we may say $f$ is continuous $\textbf{\textit{relative}}$ to specific topologies on $X$ and $Y$.

\begin{myexam}
       In analysis, a real-value function of real variable is said to be 
       continuous if it's continuous at every point of its domain. And a function
       $f:\mathbb{R}\rightarrow \mathbb{R}$ is continous at $x_0$ is define as
       follows:\\
       \indent
       $\forall \epsilon >0, $ there is some $\delta$ such that $\mid f(x)-f(x_0)
       \mid $ if $\mid x-x_0 \mid< \delta $.\\
       Then a continous real variable function is continous from $\mathbb{R}$
       to $\mathbb{R}$.
\end{myexam}

We have seen many other theorems about continuity in mathmatical analysis, for 
exmaple, a continuous function would map a limit point to a limit point, which is
$\lim\limits_{n\to \infty}f(a_n)=f(\lim\limits_{n\to \infty}a_n)$. Some of these
theorems are generalized for more general space, the following theorem describes
this.
\begin{theorem}
       Let X and Y be topological spaces; let $f\;:\;X\rightarrow Y$. Then the 
       following conditions are equivalent:
       \begin{enumerate}
              \item $f$ is continuous 
              \item Let $B$ be a basis for $Y$, then $f^{-1}(B_{\beta})$ is open
                      in $X$
              \item For every subset $A$ of $X$, one has $f(\bar{A})\subset 
                      \overline f(A)$
              \item For every closed set $B$ of $Y$, $f^{-1}(B)$ is closed in $X$.
       \end{enumerate}
\end{theorem}

\end{document}


