\documentclass[a4paper, 11pt]{article}
\usepackage{blindtext}
\usepackage[T1]{fontenc}
\usepackage[utf8]{inputenc}
\usepackage{amsmath, amsthm, amssymb, amsfonts}
\usepackage{bm}
\usepackage{graphics}
\usepackage{tikz-cd}
\usepackage{enumerate}
\usepackage{enumitem}
\usepackage{romannum}
\usepackage[a4paper,top=3cm,bottom=2cm,left=3cm,right=3cm,marginparwidth=1.75cm]{geometry}
\newtheorem{difinition}{Definition}

\newtheorem{theorem}{Theorem}
\theoremstyle{definition}
%\theorembodyfont{\normalfont}

\theoremstyle{remark}
\newtheorem*{remark}{Remark}
\newenvironment{myremark}
{\renewcommand\qedsymbol{$ $}\begin{proof}[$\mathbf{REMARK}$]}
  {\end{proof}}

\newenvironment{myprf}
{\renewcommand\qedsymbol{$ $}\begin{proof}[$\mathbf{Proof}$]}
  {\end{proof}}

\newenvironment{mydef}
{\renewcommand\qedsymbol{$ $}\begin{proof}[$\mathbf{Definition}$]}
  {\end{proof}}

\newenvironment{myexam}
{\renewcommand\qedsymbol{$ $}\begin{proof}[$\mathrm{EXAMPLE}$]}
  {\end{proof}}

\newenvironment{myprop}
{\renewcommand\qedsymbol{$ $}\begin{proof}[$\mathbf{Proposition}$]}
  {\end{proof}}

\theoremstyle{definition}
\newtheorem{lemma}[theorem]{Lemma}

\title{$\textbf{Topology Basis And Continuous Functions}$}
\date{December 21st, 2020}
\begin{document}
       \maketitle 
\section{Basis For Topology}
\begin{mydef}$\textbf{(Topology Basis)}$ Let $(X,\mathcal{T})$ be a topological 
        space. A $\textbf{basis}$ for topology $\mathcal{T}$ is a collection
        of subsets $\mathcal{B}$ such that for any open set $U\in \mathcal{T}$,
        $$
        U=\bigcup_{\alpha\in I} B_{\alpha}, B_{\alpha}\in \mathcal{B}
        $$ In other words, any open set could be denoted as the union of a 
        collection of subsets in $\mathcal{B}$
\end{mydef}
\vspace{0.5cm}
\begin{myexam}
       Let $X=\mathbb{R}$ and $\mathcal{T}$ be the conventional topology on 
       $\mathbb{R}$. Let $\mathcal{B}$ consists of all open interval with 
       rational endpoint, which is:
        $$
        \mathcal{B}=\{(a,b)\mid a,b\in \mathbb{Q},a<b\}
        $$ Then for any $\displaystyle U \underset{open}{\subset} \mathbb{R}$, 
        and any $x\in U$, there is an interval $I$ with: $x\in I_x\subset U$ and
        therefore: $\displaystyle U=\bigcup_{x\in U} I_x$.
\end{myexam}
\vspace{0.5cm}
Note that we say $\mathcal{B}$ is a basis for some specific topology if 
$\mathcal{B}$ satisfies the above definition. Under such condition, we already 
pointed out what topology this basis is corresponding to.
The next question is: what kind of
collection of subsets could be a basis for some topology on $X$?(If we didn't
specify the topology yet)\\
\indent
Suppose there is a collection of subsets of $X$, say $\mathcal{B}$. If $B$ is 
a basis for some topology $\mathcal{T}$, then every open set in $\mathcal{T}$
could be denoted as union of elements in $\mathcal{B}$. Note that 
$X\in \mathcal{T}$ therefore $X=\displaystyle \bigcup_{\alpha\in I}B_{\alpha}$.\\
\indent
Moreover, consider $U,V\in \mathcal{T}$, and denote them as union of basis 
elements of $\mathcal{B}$:
$$
\begin{aligned}
        U&=\bigcup_{\alpha\in I}U_{\alpha},U_{\alpha}\in \mathcal{B}\\
        V&=\bigcup_{\beta\in J}V_{\beta},V_{\beta}\in \mathcal{B}
\end{aligned}
$$ Then $U\cap V$ is supposed to be denoted as union of basis elements in 
$\mathcal{B}$. However, notice that:
$$
U\cap V=\bigcup_{\alpha\in I}\bigcup_{\beta\in J}(U_{\alpha}\cap V_{\beta})
$$ Therefore, we only need that $U_{\alpha}\cap V_{\beta}$ could be denoted as union of basis elements in $\mathcal{B}$.So far, we have got a sufficient condition 
that makes $\mathcal{B}$ be a basis of some topology on $X$:
\begin{theorem}
        Let $X$ be a set and $\mathcal{B}$ be a collection of subsets in $X$. If
        $\mathcal{B}$ satifies the following two requirements, then $\mathcal{B}$
        is a basis for some topology:\\
        (\romannum{1}) For any $x\in X$, there is some $B\in \mathcal{B}$ 
        such that $x\in B\subset X$\\
        (\romannum{2}) For any $B_1,B_2\in mathcal{B}$ with $U\cap V\neq \varnothing$
        , and any $x\in B_1\cap B_2$, there is some $B_3\in \mathcal{B}$ such that
        $x\in B_3\subset B_1\cap B_2$
\end{theorem}
\begin{myprf}
       Define $\mathcal{T}=\{\bigcup_{\alpha\in I} B_{\alpha}\mid B_{\alpha}\in
       \mathcal{B}\}$, then it's easy to verify that $\mathcal{T}$ is a topology
       on $X$ and $\mathcal{B}$ is a basis of $\mathcal{T}$.
\end{myprf}
\vspace{0.5cm}
\begin{mydef}
       If $\mathcal{B}$ satisfies these two conditions in theorem 1, then we
       define $\textbf{the topology} $ $\textbf{generated by}$ $\mathcal{B}$ as
       follows: A subset $U$ of $X$ is said to be open in $X$ if for each 
       $x\in U$, there is a basis element $B\in \mathcal{B}$ such that 
       $x\in B\subset U$.
\end{mydef}
\begin{myremark}
        The topology generated by $\mathcal{B}$ is apparently equivalent
        to the topology mentioned in the proof of theorem 1. And now we know
        the topology generated by $\mathcal{B}$ is actually those sets which
        can be denoted as union of elements in $\mathcal{B}$.
\end{myremark}

\indent The word "basis" might be confusing. We know the open set of $\mathcal{B}$-generated topology equals to the union of some subsets in $\mathcal{B}$ but this
expression is not unique. However, in other subjects, for example, linear algebra
,  a basis means element could be uniquely expressed as linear combination of basis elements.\\
\indent
However, there is one thing follows the same for the basis concepts in linear 
albegra and topology, that is there might be multiple basis for the same topology,or linear space.But the following theorem, gives a sufficient condition for 
equivalent topology.
\vspace{0.5cm}
\begin{theorem}
        We say two basis are $\textbf{equalvalent}$ if they generates the same
        topology. If two basis $\mathcal{B}_1$, $\mathcal{B}_2$ satisfies the 
        following two conditions, then they are equivalent.\\
        (\romannum{1}) For any $B_1\in \mathcal{B}_1$, any $x\in B_1$, there
        is some $B_2\in \mathcal{B}_2$ such that $x\in B_2\subset B_1$\\
        (\romannum{2}) For any $B_2\in \mathcal{B}_2$, any $x\in B_2$, there
        is some $B_1\in \mathcal{B}_1$ such that $x\in B_1\subset B_2$\\
\end{theorem}
\begin{myprf}
        Trivial. The condition specifies that every element in $\mathcal{B}_1$ 
        could be expressed as union of elements in $\mathcal{B}_2$ and vice versa.
\end{myprf}
\vspace{0.5cm}
\begin{mydef}$(\textbf{subbasis})$ A $\textbf{subbasis}$ $\mathcal{S}$
        for a topology on $X$
        is a collection of subsets of $X$ whose union equals to $X$. The
        $\textbf{topology generated by the subbasis}$ $\mathcal{S}$ is defined
        to be the collection $\mathcal{T}$ of all unions of finite intersection
        of elements of $\mathcal{S}$
        
\end{mydef}

\section{Continuous Function}
\subsection{Continuous Function}
\begin{mydef}
       Let $X$ and $Y$ be topological spaces. A function $f:X\rightarrow Y$ is
       said to be $\textbf{continuous}$ if for each open set $V$ of $Y$, 
       the set $f^{-1}(V)$ is an open set of $X$
\end{mydef}
It is easy to see that the continuity of a function $f$ depends not only on the
function itself, but also the specified topology of its domain and range. To 
emphasize this fact, we may say $f$ is continuous $\textbf{\textit{relative}}$ to specific topologies on $X$ and $Y$.

\begin{myexam}
       In analysis, a real-value function of real variable is said to be 
       continuous if it's continuous at every point of its domain. And a function
       $f:\mathbb{R}\rightarrow \mathbb{R}$ is continous at $x_0$ is define as
       follows:\\
       \indent
       $\forall \epsilon >0, $ there is some $\delta$ such that $\mid f(x)-f(x_0)
       \mid $ if $\mid x-x_0 \mid< \delta $.\\
       Then a continous real variable function is continous from $\mathbb{R}$
       to $\mathbb{R}$.
\end{myexam}

We have seen many other theorems about continuity in mathmatical analysis, for 
exmaple, a continuous function would map a limit point to a limit point, which is
$\lim\limits_{n\to \infty}f(a_n)=f(\lim\limits_{n\to \infty}a_n)$. Some of these
theorems are generalized for more general space, the following theorem describes
this.
\begin{theorem}
       Let X and Y be topological spaces; let $f\;:\;X\rightarrow Y$. Then the 
       following conditions are equivalent:
       \begin{enumerate}[label=(\arabic*)]
              \item $f$ is continuous 
              \item Let $\mathcal{B}$ be a basis for $Y$, then $f^{-1}(B_{\beta})$ is open
                      in $X$
              \item For every subset $A$ of $X$, one has $f(\bar{A})\subset 
                      \overline {f(A)}$
              \item For every subset $B$ of $Y$, one has $\overline{f^{-1}(B)}
                              \subset f^{-1}(\bar{B})$
              \item For every closed set $B$ of $Y$, $f^{-1}(B)$ is closed in $X$.
       \end{enumerate}
\end{theorem}
\vspace{0.3cm}
\begin{myprf}
        $\mathbf{(1)\Rightarrow (2)}$: Every $B_{\beta}\in \mathcal{B}$ is 
        an open set in $Y$,  therefore $f^{-1}(B_{\beta})$ is open in $X$.\\
        \\
        $\mathbf{(1)\Rightarrow (3)}$: Let $x$ is a limit point of $A$, it's sufficient
        to show that $f(x)$ is a limit point of $f(A)$ if $f(x)\notin f(A)$
        (There
        are other situations where $x\in A$ or $f(x)\in f(A)$, but it's
        simple). We prove as proceed:
        For any open set $U$ of $Y$ that contains $f(x)$, consider $f^{-1}(U)$,
        note that $x\in f^{-1}(U)$ as $f(x)\in U$. And by definition of continuous
        function we have $f^{-1}(U)$ is an open set of $X$ that contains $x$. 
        Therefore $f^{-1}(U)\cap A \neq \varnothing$ and $U\cap f(A)\neq 
        \varnothing$. By definition, $f(x)$ is a limit point of $f(A)$.\\
        \\
        $\mathbf{(2)\Rightarrow (1)}$ We will show that $f$ is continous if it
        satisfies condition in $(2)$. For any open set $U$ in $Y$, we can denote
        $U$ as union of the basis elements, such that:$\displaystyle 
        U=\bigcup_{\beta\in I}B_{\beta}$, and by theorem about inverse image 
        of union, we have:
        $$
        f^{-1}(\bigcup_{\beta\in I}B_{\beta})=\bigcup_{\beta\in I} f^{-1}(B_{\beta})
        $$, which is open in $X$. The proof is done.\\
        \\
        $\mathbf{(3)\Rightarrow (4)}$ By the definition of inverse image, it's 
        sufficient to show that $f(\overline{f^{-1}(B)})\subset \bar{B}$.
        By (3), we have:
        $$
        f(\overline{f^{-1}(B)})\subset \overline{f(f^{-1}(B))}\subset \bar{B}
        $$
        The last subsets holds as $f(f^{-1}(B)\subset B$.\\
        \\
        $\mathbf{(4)\Rightarrow (5)}$ If $F$ is closed in $Y$, then 
        $\bar{F}=F$. Then we have: $\overline{f^{-1}(F)}\subset f^{-1}(\bar{F})
        =f^{-1}(F)$ by $(3)$. It's obvious that $f^{-1}(F)\subset \overline{f^{-1}(F)}$ and hence $f^{-1}(F)=\overline{f^{-1}(F)}$, which shows
                $f^{-1}(F)$ is closed in $X$\\
        \\
        $\mathbf{(5)\Rightarrow (1)}$ For any open set $U\in Y$, consider
        $f^{-1}(U)$. Let $V=Y\setminus U$, then $V$ is closed and
        $f^{-1}(U)=f^{-1}(Y\setminus V)=X\setminus f^{-1}(V)$. Notice that
        $f^{-1}(V)$ is closed, therefore $f^{-1}(U)$ is open and $f$ is 
        a continous function.
\end{myprf}
\subsection{Limit and Hausdorff Space}
We mention before the proof of theorem 3, that in $\mathbb{R}-\mathbb{R}$ function
we have $\lim\limits_{n\to \infty}f(a_n)=f(\lim\limits_{n\to \infty}a_n)$ if 
$f$ is continous. We may generalize this theorem if we
define the $\textbf{limit}$ of a sequence of point in topological space.
\vspace{0.5cm}
\begin{mydef}
        Let $X$ be a topological space, $\{x_n\}\subset X$ is a sequence of point
        in $X$, we say $x$ is the $\textbf{limit}$ of $\{x_n\}$, denoted by
        $\lim\limits_{n\to\infty}x_n=x$(this denotion might be invalid), if :\\
        \indent
        $$
        U\underset{open}{\subset}X, x\in X\Rightarrow
        \exists N\in \mathbb{N}, st. \forall n>N, x_n\in U
        $$ The open set $U$ of $X$ that contains $x$ is called a $\textbf
        {neighborhood}$ of $x$
\end{mydef}
\begin{myremark}
        The limit of $\{x_n\}$ might not be unique in a general topological space. 
\\
\indent
        Consider $X=\{0,1\}, \mathcal{T}=\{\varnothing, X, \{0\}\}$. And consider
        a point sequence :$\{x_n=0\},\forall n\in \mathbb{N}$. Then both 0 and 1 ar         e limit of this sequence by definition.\\
        \indent
        As explained above, $\lim\limits_{n\to\infty} x_n=x$ is an invalid denotion
        if there are two limits. However, we may put a constraints on the space
        to make there exists only one limit(if there is any) for a sequence.
\end{myremark}
\vspace{0.5cm}
\begin{mydef}$\textbf{(Hausdorff space)}$
       Let $X$ be a topological space. $X$ is said to be a $\textbf{Hausdorff 
       space}$ if for any $x_1,x_2\in X$ and $x_1\neq x_2$, there exists two
       open sets $U_1,U_2\subset X$ such that $x_1\in U_1, x_2\in U_2$ and 
       $U_1\cap U_2=\varnothing$
\end{mydef}
\vspace{0.3cm}
\begin{theorem}
        Let $X$ be a Hausdorff space, then any sequence $\{x_n\}$ has unique limit
        (if there is any)
\end{theorem}
\begin{myprf}
        If there are two limits of $\{x_n\}$, say $a$ and $b$. Then by definition
        of Hausdorff space, there are two open sets $U_a,U_b$ contains $a$ and $b$
        respectively but $U_a\cap U_b=\varnothing$. According to the definition
        of limit, there exists $N_1,N_2$ such that $x_n\in U_a,\forall n>N_1$;
        $x_n\in U_b, \forall n>N_2$. Let $N=\max(N_1,N_2)$, then $x_n
        \in U_a\cap U_b, n>N$, a contradiction.
\end{myprf}
There are multiple other interesting facts about Hausdorff space. Recall that we
mentioned a fact in last section: a single point might not be closed in a general
topological space. A typical example is as follows:
$$
X=\{a,b,c\}, \mathcal{T}=\{\{a,b\},\{b\},\{b,c\},\varnothing,X\}
$$
In this topological space, $\{b\}$ is not closed. And further more, the sequence
$\{x_n\}$ where $x_n=b,\forall n\in \mathbb{N}$ has two limits $a$ and $c$.\\
\indent
We have proved in Hausdorff space, sequence has only one limit. Further more, a
single point is closed in Hausdorff space.
\begin{theorem}
       Let $X$ be a Hausdorff space, then every finite set of $X$ is closed.
\end{theorem}
\begin{myprf}
        It suffices to show that any single point is closed. Let $\{x_0\}$ be a
        one-point set. For any $x\in X,x\neq x_0$, there is some open set
        $U_x$ contains $x$ but does not intersect $x_0$, therefore $x$ is not 
        a limit point of $x_0$. We have this fact for arbitraty $x\neq x_0$. So
        the closure of $\{x_0\}$ is $\{x_0\}$ itself. Therefore $\{x_0\}$ is closed
        .
\end{myprf}
The condition that finite point sets be closed is called $\mathbf{T_1\;axiom}$. 
And $T_1$ axiom is $\textbf{weaker}$ than Hausdorff condition, which means: a topological 
space that satisfies $T_1$ axiom may not be a Hausdorff space. We will explain
the reason in later chapters.
\vspace{0.5cm}
\begin{theorem}
      Let $X$ be a space satisfying the $T_1$ axiom; let $A$ be a subset of $X$.
      Then the point $x$ is a limit point of $A$ if and only if
      every neighborhood of $x$
      contains infinitely many points of $A$.
\end{theorem}
\begin{myprf}
        $(\Leftarrow):$ 
        If any neighborhood of $x$ contains infinitely many points of 
      $A$, the neighborhood must contain some points other than $x$ iteself, 
      so that $x$ is a limit point of $A$.\\
      $(\Rightarrow):$ Conversely, suppose that $x$ is a limit point of $A$, and
      suppose there is some open set $U$ containing $x$ but intersects only
      finitely many points with $A$. Consider $A\setminus\{x\}\cap U$, denoted
      by $V$, then $V$ also has finitely many points. \\
      \indent
      By $T_1$ axiom, $V$ 
      is closed in $X$ and $X\setminus V$ is open. Notice that $U\cap (X\setminus
      V)$ is an open set that contains $x$. However, this open set intersects
      an empty set with $A$, which contradicts the assumption 
      that $x$ is a limit point of $A$. 
\end{myprf}




\end{document}


