\documentclass[a4paper, pdf, 11.5pt]{article}
\usepackage{amsthm} %lets us use \begin{proof}
\usepackage{amsmath}
\usepackage{amssymb} %gives us the character \varnothing
\usepackage[dvipsnames]{xcolor}
\usepackage{enumitem}
\usepackage{pst-node}
\usepackage{auto-pst-pdf}
\usepackage{tikz-cd} 
\usepackage{tikz}
\usepackage{mathrsfs}
\usepackage{extarrows}
\usepackage{amsmath, amsthm, amssymb, calrsfs, wasysym, verbatim, bbm, color, graphics, geometry}

\geometry{tmargin=.75in, bmargin=.75in, lmargin=.75in, rmargin = .75in}  

\newcommand{\catname}[1]{{\normalfont\textbf{#1}}}
\newcommand{\divides}{\mid}
\newcommand{\notdivides}{\nmid}
\newtheorem{definition}{Definition}
\newtheorem{prop}{Proposition}
\newtheorem{theorem}{Theorem}

\newlist{notes}{enumerate}{1}
\setlist[notes]{label=Note: ,leftmargin=*}
\setlist[itemize]{leftmargin=*}
\date{}

\title{Chapter3 Rings and Modules}
\begin{document}
\maketitle
\tableofcontents
\section{Definiton of ring}
\subsection{Basic Motivation:$\mbox{End}(G)$}
The basic motivation of introduction of ring is the $\mbox{Hom}_{\mathbf{Ab}}(G, G)$(or simply $\mbox{End}(G)$), that is, the set of all endmorphism over an abelian group $G$. We can define the so called addition over this set as follows:
$$
(\forall f,g\in \mbox{End}(G): (f+g)(a) = f(a) + g(a)
$$
It's easy to show that $\mbox{End}(G)$ forms an abelian group  if $G$ is abelian. One thing to rememebr is that not any general group $G$ satisfies $\mbox{End}(G)$ is an abelian group. The key point is that the above-defined $f+g$ might not be a group homomorphism if $G$ is not abelian:
$$
\begin{aligned}
(\forall f,g\in \mbox{End}(G), a,b\in G):\\(f+g)(a+b) &= f(a+b) + g(a+b)\\
&= f(a) + f(b) + g(a) + g(b)\\
&\xlongequal{G\;is\;abelian} f(a) + g(a) + f(b) + g(b)\\
&= (f+g)(a) + (f+g)(b)
\end{aligned} 
$$

In conclusion, $\mbox{End}(G)$ forms an abelian group under homomorphism addition. However, there is another type of operation: **Composition of homomorphisms**
$$
(f,g\in \mbox{End}(G)): (f\circ g)(a) = f(g(a))
$$

Thus, there are two kinds of different operations within set $\mbox{End}(G)$. That's the basic motivation of a new algebra structure, called \textbf{ring}.

\subsection{Definition of ring}
A ring $(R, +, ·)$ is an \textbf{abelian group} $(R, +)$ endowed with a second binary operation$·$(often omit this dot notion), satisfying of its own the requirements of being associative and having a two-sided identity:
\begin{itemize}
  \item \textbf{Associativity}: $(\forall r,s,t\in R):\quad (rs)t=r(st)$
  \item \textbf{Existence of Identity}: $(\exists 1_{R}\in R)(\forall r\in R):\quad 1_{R}r=r1_{R} = r$
\end{itemize}

Also, there are laws combining two different operations, called \textbf{distributive law}:
\begin{itemize}
  \item  $(\forall r,s,t\in R):\quad r(s+t) = rs+rt, (r+s)t = rt+st$
\end{itemize}
The operation $+$ and $\cdot$ are called addition and multiplication resepctively. 

Here are one point to note :  Within this book, a ring is always to be considered have \textbf{multiplication identity}. Some other definition may not require a ring to have an identity. \\
\\
\noindent
\textbf{Examples}
\begin{itemize}
  \item  \textbf{Trivial ring}.  There is only one element $\{0\}$, which is the addition identity.
  \item  \textbf{Integer ring}.  $(\mathbb{Z}, +, \times)$ forms a ring, where addition and multiplication are naturally integer addition and multiplication.
  \item \textbf{Modular ring}.  The addition group $\mathbb{Z}/n\mathbb{Z}$ forms a ring. The addition and multiplication is modular addition and multiplication. 
  \item  \textbf{Matrix ring}.  All square matrix of order $n$ forms a ring, the addition and multiplication are matrix addition and multiplication.
\end{itemize}

\subsection{Rings with special properties}
\subsubsection{Commutative ring}
\begin{definition}(Commutative Ring)
  A ring $R$ is commutative, if  multiplication is commutative, that is 
$$(\forall r,s\in R):\quad rs = sr$$
$R$ is called commutative ring under in such case.
\end{definition}

In our examples, $\mathbb{Z}, \mathbb{Z}/n\mathbb{Z}$ are commutative rings. But matrix ring is not. 

\subsubsection{Zero divisors and Integral domain}
\begin{definition}(zero-devisor)
  Let $R$ be a ring, and an element $r\in R$ is a left(resp.right) zero-divisor, if 
  $$(\exists s\in R, s\neq0):\quad rs = 0(sr=0)$$
\end{definition}
The following proposition depicts the property of a zero-divisor:
\begin{prop}
Let $R$ be a ring and $r\in R$ is an element. The following statements are equavialent:
\begin{itemize}
  \item  $r$ is \textbf{not} a left zero divisor.
  \item Function: $f:R\longrightarrow R, a\mapsto ra$ is injective.
\end{itemize}
\end{prop}
\vspace{0.5cm}
It is easy to prove the proposition and the right zero divisor case. By the definiton of $R$, we give the following definiton of integral domain:

\begin{definition}
 A ring $R$ is called an integral domain, if it is \textbf{commutative} and has no zero-divisors, i.e. 
$$
(\forall a,b\in R)\quad ab=0\Longrightarrow a=0\;\mbox{or}\;b=0
$$
\end{definition}
\vspace{0.5cm}
According to the definiton of integral domain and the property of zero-divisors We have the cancellation law holds:
\\
\noindent
\textbf{(Cancellation)} If $R$ is an integral domain, then:
$$
(\forall a,b,c\in R, a\neq 0):\quad
ab=ac\Longrightarrow b =c
$$
That is, in integral domain we can simply cut off the same component in a multiplication expression, which is the same as we do in group.\\
\noindent
\textbf{Examples}
$\mathbb{Z}$ is an integral domain. However, both $Z/n\mathbb{Z}$ and matrix ring are not integral domain in general case. 
For example, in matrix ring of order 2, we have:
$$
\begin{pmatrix}
1 & 0\\
0 & 0
\end{pmatrix}
\begin{pmatrix}
0 & 0\\
0 & 1
\end{pmatrix}
= \begin{pmatrix}
0 & 0\\
0 & 0
\end{pmatrix}
$$
In $\mathbb{Z}/6\mathbb{Z}$, we have $[2]_{6}\times[3]_{6}=[0]_{6}$. Thus $\mathbb{Z}/6\mathbb{Z}$ is not an integral domain.
However, there is a class of $n$ that makes $\mathbb{Z}/n\mathbb{Z}$ integral domain, in particular, they are actually field. 

\subsubsection{Unit and division ring}

\begin{definition}
Let $R$ be a ring and an element $r\in R$. $r$ is called a left(resp. right) unit if 
$$\exists v\in R, uv=1(\mbox{resp.}\;vu=1)$$ $r$ is an unit if it is both left and right side unit.
\end{definition}
\vspace{0.5cm}

Similar to zero divisor, we given a depiction of unit as the following proposition:

\begin{prop}
Let $R$ be a ring and $r\in R$. 
\begin{itemize}
  \item  $r\;\mbox{is a left unit}\Longleftrightarrow f:R\longrightarrow R,a\mapsto ra\;\mbox{is surjective}$
  \item  $r\;\mbox{is a left unit}\Longrightarrow r\;\mbox{is not a right zero-divisor}$
  \item $\mbox{The inverse of two-sided unit is unique}$
  \item  $\mbox{The set of all two-sided unit forms a group}$.
\end{itemize}
\end{prop}
\vspace{0.5cm}
\begin{proof}
The proof of above propositions are easy. For the third proposition we could actually prove that if $r$ is a two sided unit, then the left-inverse and right inverse equals. 
$$
u = u1 = u(rv) = (ur)v = 1v = v
$$
That's why we can use the word $\textit{inverse}$ to denote both left and right inverse.
\end{proof}
\vspace{0.3cm}

\begin{definition}(division ring and field)
A division ring is a ring in which every non-zero element is an unit. A field is a non-zero commutative division ring.
\end{definition}
\vspace{0.5cm}
\noindent
It's obviously that both $\mathbb{Q}, \mathbb{R}$ are fields. The following theorem implies a class of special modular group:
\begin{theorem}
$\mathbb{Z}/n\mathbb{Z}$ is field if and only  if $n$ is a prime.
\end{theorem}
\noindent
$\textit{Hint}.$ We only need to show that $[a]_{n}$ is unit if and only if $\mbox{gcd}(a, n) = 1$. 
\vspace{0.5cm}

\begin{theorem}
 $R$ is a finite commutative ring, then $R$ is field if and ony if $R$ is integral domain.
\end{theorem}
\noindent
$\textit{Hint}.$ Field is naturally an integral domain. If $R$ is an integral domain, prove that each $r\in R$ is unit by considering the left multiplication function. It must map some element to 1 since $R$ is finite and this map is injective.
More specificly, one injective map from a finite set to iteself must be surjective
\vspace{0.5cm}

\subsection{Other examples of rings}
\subsubsection{Polynomial rings}
Let $R$ be a ring and define a polynomial $f(x)$ over $R$ as the following form:
$$
f(x) = \sum_{i\geq 0}a_ix^{i} = a_0+ a_1x+a_2x^{2}+\cdots
$$
Note that each $f(x)$ only has finitely many summation. The set of all $f(x)$ is a ring, called Polynomial ring over $R$, denoted as $R[x]$. 

\begin{definition}(Degree of polynomial)
Let $f(x)\in R[x]$, the degree of $f(x)$, denoted as $\mbox{deg}f$, is the maximum $n$ such that $a_{n}\neq 0$.  Typically we define $\mbox{deg}r = 0, r\in R$ and $\mbox{deg}0 = -\infty$.
\end{definition}

When $R$ is an integral domain, $R[x]$ is also an integral domain. And the following equation holds:
$$
\mbox{deg}(fg) = \mbox{deg}f + \mbox{deg}g
$$

\subsubsection{Monoid rings}
Monoid rings is a ring constructed from a monoid and a ring. Here is the definition:
\begin{definition}(Monoid rings)
Let $R$ be a ring and $M$ a monoid, then consider all the following linear combinations:
$$
\sum_{m\in M}a_{m}\cdot m, a_{m}\in R
$$
Where only finitely many $a_{m}\neq 0$. The addition and multiplication are defined as follows:
$$
\begin{aligned}
\sum_{m\in M}a_{m}\cdot m+\sum_{m\in M}b_{m}\cdot m &= \sum_{m\in M}(a_m+b_m)\cdot m\\
(\sum_{m\in M}a_{m}\cdot m)(\sum_{m\in M}b_{m}\cdot m) &=\sum_{m\in M}(\sum_{m_1m_2=m}a_{m_1}b_{m_2})m
\end{aligned}
$$
\end{definition}
\vspace{0.3cm}
Under this definition, it's easy to show that all combination forms a ring. It is called Monoid rings, 
denoted as $R[M]$. Actually, the polynomial ring is a special case of general monoid ring, where we take 
$M=\{1, x, x^{2}, x^{3}, \dots \}$. 

\section{Category Ring}
\subsection{Ring homomorphism}
A ring homomorphism is a function between two rings that maintains two operations: $\cdot$ and $+$, that is:
\begin{definition}
Let $R, S$ be rings, a function $f: R\rightarrow S$ is a ring homomorphism, if:
\begin{itemize}
  \item  $(\forall a, b\in R):\quad f(a+b) = f(a) + f(b)$
  \item  $(\forall a,b\in R): \quad f(ab) = f(a)f(b)$
  \item  $f(1_{R}) = 1_{S}$
\end{itemize}
\end{definition}
\vspace{0,3cm}

Since $f$ is a function maintains $+$, it is basically a group homomorphism of the underlying abelian group. Thus, it naturally has: $f(0_{R})=0_{S}$. However, the second axiom does not induce the third one. That is: a function maintains both $\cdot$ and $+$ might be send identity to identity. 
For example:
$$
f: \mathbb{Z}\rightarrow \mathbb{Z}, a\mapsto 0
$$
is a function maintaining both addition and multiplication. But it is not a ring homomorphism since it does not meet the 
third requirements.
\begin{prop}
Let $R, S$ be non-zero rings, $f:R\rightarrow S$ is a ring homomorphim, the following statement is true:
\begin{itemize}
  \item  If $r\in R$ is an unit, then $f(r)$ is an unit in $S$, $f(r)^{-1} = f(r^{-1})$
  \item If $r\in R$ is a zero-divisor, then $f(r)$ might not be a zero-divisor as $f(r)$ might be zero. 
  \item The composition of ring homomorphism is still a ring homorphism.
\end{itemize}
\end{prop}
\vspace{0.3cm}

\subsection{Category $\textbf{Ring}$}
The category "Ring" consists all rings, and the morphism set between two objects is the ring homomorphisms. 

There are some interesting results in $\textbf{Ring}$: 
 $\{0\}$ is a final object in $\textbf{Ring}$ but not a initial object. The reason is that the identity in $\{0\}$ is exactly its zero, which means $\{0\}$ only has a ring homomorphism to itself.
 $(\mathbb{Z}, \cdot, +)$ is an initial object in $\textbf{Ring}$: any homomorphism $f: \mathbb{Z} \rightarrow R$ is uniquely determined by $f(1)$, i.e $f(n) = nf(1)=n1_{R}$.

The following proposition describes the universal property of polynomial ring on $\mathbb{Z}$: 


\begin{theorem}
Let $A$ be a finite set: $A=\{a_1, a_2,\ldots, a_{k}\}$. Consider a new category $\mathscr{R}_{A}$: The object of $\mathscr{R}_{A}$ is $(j, R)$, where $R$ is a ring, and $j$ is 
a set-function from $A$ to $R$.
$$
j:A\rightarrow R
$$ 
The morphism from object $(j_{1}, R_{1})$ to $(j_{2}, R_{2})$ is the following diagram:
$$
\begin{tikzcd}
  R_1 \arrow[r, "\varphi"]              & R_2 \\
  A \arrow[u, "j_1"] \arrow[ru, "j_2"'] &    
\end{tikzcd}
$$
Then $(\mathbb{Z}[x_1,x_2,\ldots,x_{k}], \iota)$ is an initial object in $\mathscr{R}_{A}$, where $\iota(a_{i}) = x_{i}, i=1,2,\ldots, k$.
\end{theorem}
\vspace{0.2cm}
\begin{proof}
The proof this this theorem is pretty straight forward. We need to show for each $(j, R)$ in $\mathscr{R}_{A}$, the following diagram is true:
To prove this, one ituitive way is to map each polynomial to its coresponding "value": $x_{i}$ is replaced as $j(a_{i})$, and the whole polynomial forms a linear summation of multiplication consists of $j(a_{i})$. The uniqueness is determined by the property of homomorphism.

For each object $(R, j)$, we need to show the following diagram holds:
$$
\begin{tikzcd}[row sep=huge]
  {\mathbb{Z}[x_1,x_2,\ldots,x_k]} \arrow[rr, "\exists!\varphi"] &  & R \\
  A \arrow[rru, "j"'] \arrow[u, "\iota"]                         &  &  
  \end{tikzcd}
$$
For a fixed object $(R, j)$, define $\varphi$ as follows:
$$
\begin{aligned}
\varphi(\sum_{i}a_{i}x_1^{i_1}x_2^{i_2}\cdots x_{k}^{i_k}) &= \sum_{i}\varphi(a_{i})\varphi(x_1)^{i_1}\varphi(x_2)^{i_2}\cdots\varphi(x_k)^{i_k}\\
&= \sum_{i} (a_i1_{R})\varphi(\iota(a_1))^{i_1}\varphi(\iota(a_2))^{i_2}\cdots\varphi(\iota(a_k))^{i_k}\\
&= \sum_{i} (a_i1_{R})(j(a_1))^{i_1}(j(a_2))^{i_2}\cdots(j(a_k))^{i_k}
\end{aligned}
$$
In the above construction, we do not only present a ring homomorphim that maps from 
$\mathbb{Z}[x_1,x_2,\ldots, x_k]$ to $R$, but also shows the uniqueness by using the fact 
that $R$ must maintains both addition and multiplication. Thus $\varphi$ is unique. The key to 
the proof is that the fact that $\mathbb{Z}$ is initial in \textbf{Ring}.
\end{proof}

\subsection{Monomorphism and Epimorphism}
\subsubsection{Monomorphism}
\begin{definition}(Kernel of ring homomorphim)
Let $R, S$ be rings and $f$ a ring homomorphism from $R$ to $S$, define kernel of this homomorphism as:
$$
\ker f=\{r\in R\mid f(r) = 0_{S}\}
$$
\end{definition}
\vspace{0.3cm}
\begin{theorem}(Equavalence of ring monomorphism)
Let $f$ be a ring homomorphism from $R$ to $S$, the following statements are equavalent: 
\begin{enumerate}
  \item  $f$ is monomorphism 
  \item  $\ker f=\{0_{R}\}$
  \item  $f$ is injective as set-function
\end{enumerate}
\end{theorem}
\begin{proof}
  $(1)\Rightarrow(2)$ Consider the following diagram:
  $$
  $$
  
\end{proof}
\subsubsection{Epimorphism}
\vspace{0.3cm}
\begin{definition}
A ring homomorphism $f:R\rightarrow S$ is a ring homomorphism, if and only if for any ring $T$ and ring homomorphism $S\rightarrow T, \varphi_1, \varphi_2$:
$$
f\circ \varphi_1 = f\circ \varphi_2\Longrightarrow \varphi_1 = \varphi_2 
$$
That is, the following commutative diagram
$$
\begin{tikzcd}[column sep=huge]
  R \arrow[r, "f"] & S \arrow[r, "\varphi_1", shift left] \arrow[r, "\varphi_2"', shift right] & T
\end{tikzcd}
$$
indicates $\varphi_1 = \varphi_2$. 
\end{definition}

\section{Ideals and quotients: remarks and examples}
Most of contents used in this section would assume $R$ is a commutative ring. We would explicitly 
point it out if $R$ is non-commutative.
\begin{definition}(Principle Ideal)
  $Ra$ is a left-side ideal, and $aR$ is a right-side ideal. If $R$ is commutative, then $Ra$ 
  is a two-sided ideal, called principle ideal generated by $a$, denoted as $(a)$.
\end{definition}

Let $S=\{a_1, a_2,\ldots a_{k}\}$, the ideal generated by $S$ is the minimal ideal that contains 
$S$. It is easy to prove that the ideal generated by $S$ is:
$$
(a_1) + (a_2) + \cdots + (a_k) = \{\sum_{i=1}^{k}r_{i}a_{i}\mid r_{i}\in R\}
$$
And this ideal is denoted as $(a_1, a_2,\ldots, a_{k})$. If $k$ is finite, then $(a_1, a_2,\ldots, a_{k})$ is 
called \textit{finitely-generated}. A special class of ring that widely used in algebraic geometry is as follows:
\begin{definition}
  A commutative ring $R$ is called Noetherian if every ideal is finitely generated.
\end{definition}

\begin{definition}
  An integral domain $R$ is called principle ideal domain(PID), if every ideal of $R$ is principle ideal.
\end{definition}
\vspace{0.3cm}
\noindent
It's that PID is a special case of Noetherian ring. There are some basic facts that we know as follows:
\begin{enumerate}
  \item $\mathbb{Z}$ is a PID.
  \item If $k$ is a field, then the polynomial ring $k[x]$ is a PID.
  \item $\mathbb{Z}[x]$ is not PID, for example, ideal $(2, x)$ is not a principle ideal.
\end{enumerate}
Both $\mathbb{Z}$ and $k[x]$ are \textit{Euclidean domain}, namely, it means a domain where 
the Euclidean algorithm holds. $\mathbb{Z}[x]$, however, are not that special, but it is an
UFD, which stands for \textit{Unique Factorization Domain}.

\subsection{Quotient of polynomial ring}
This section considers a special quotient of polynomial ring $R[x]$. Let $f(x)\in R[x]$ be a 
\textit{monic} polynomial, i.e. the leading coefficient of $f(x)$ needs to be exactly 1. Then 
for each $g(x)\in R[x]$, there exists unique $h(x), r(x)\in R[x]$, such that:
$$
g(x) = h(x)f(x) + r(x), \deg r(x) < \deg f(x)
$$
This is basically what Euclidean algorithm tells us, but note that this might not 
be true if $f(x)$ is not monic. For example, consider $\mathbb{Z}[x]$, and $f(x) = 2x+1$. Then 
$$
x+1 = (2x+1) \times 1 + (-x)
$$
Addmiting the above equations, we continue by considering the principle ideal $(f(x))$, and quotient
$R[x]/(f(x))$. $R[x]/(f(x))$ consists of all cosets such that:
$$
R[x]/(f(x)) = \{r(x) + (f(x))\mid r(x)\in R[x]\}
$$
Considering the following fact: 
$$
g(x) = f(x)g(x) + r(x) \Rightarrow g(x) + (f(x)) = r(x) + (f(x))
$$
This means:
$$
R[x]/(f(x)) = \{\overline{r(x)}\mid \deg r(x) < \deg f(x)\}
$$
Thus, we basically view $R[x]/(f(x))$ (in group concept) as addtiion group 
consists of all polynomials with degree less than $\deg f(x)$. More formly, 
that is:
$$
R[x]/(f(x))\cong R^{\oplus d}
$$
where $d$ is the degree of $f(x)$. The isomorphism $\varphi$ would be:
$$
\varphi(\overline{r(x)}) = \varphi(\sum_{i=0}^{d-1}r_{i}x^{i})=(r_0, r_1,\ldots, r_{d-1})
$$
It's basically naive to verify this function is indeed a group homomorphism between two abelian groups. 
One of most trivial examples would be:
$$
R[x]/(x - a)\cong R
$$
One more complex example is considering $f(x)$ with degree of 2, for example:
$$
\mathbb{R}[x]/(x^{2} + 1) \cong \mathbb{R}^{2}
$$
The isomorphism, as explained, is $\varphi(a_0 + a_1x) = (a_0, a_1)$. However, in what 
condition does this isomorphism is also a ring homomorphism? This typically requires:
$$
\begin{aligned}
\varphi((a_0+a_1x)(b_0+b_1x)) &= \varphi(a_0+a_1x)\varphi(b_0+b_1x)\\
\end{aligned}
$$
Note that 
$
\varphi(a_0b_0 + (a_0b_1 + a_1b_0)x + a_1b_1x^{2}) = 
\varphi((a_0b_0-a_1b_1)+(a_0b_1+a_1b_0)x+(a_1b_1)(x^2+1)) = (a_0b_0-a_1b_1, a_0b_1+a_1b_0)
$, thus we must have 
$$
(a_0b_0-a_1b_1, a_0b_1+a_1b_0) = (a_0, b_0)(a_1, b_1)
$$
Conversely, if we define multiplication over $\mathbb{R}^{2}$ as above, $\mathbb{R}^{2}$ does 
form a ring. And this ring, is exactly isomorphic to $\mathbb{C}$.

\subsection{Prime ideal and maximal ideal}
There are actually two equavalent definitions of prime ideal and maximal ideal. 
\begin{definition}
  Let $I\neq (1)$ be an ideal of a commutative ring $R$.
  \begin{enumerate}
    \item $I$ is a prime ideal, if $R/I$ is an integral domain
    \item $I$ is a maximal ideal, if $R/I$ is a field
  \end{enumerate}
\end{definition}
\noindent
The equavalent definitions would be:
\begin{definition}
  Let $I\neq (1)$ be an ideal of a commutative ring $R$.
  \begin{enumerate}
    \item $I$ is a prime ideal, if $ab\in I\Rightarrow a\in I$ or $b\in I$
    \item $I$ is a maximal ideal, if for any ideal $J\subseteq R$, $I\subseteq J\Rightarrow I=J$ or $J=R$.
  \end{enumerate}
\end{definition}
\noindent
There is no difficulty to prove these two definitions equals to each other. But it seems more reasonable if we call the
second group definition and the first group `properties'.

According to the definition, one obvious fact would be $\mbox{maximal}\Rightarrow \mbox{prime}$ since $R/I$ is a field 
induces $R/I$ is an integral domain. Further more, the following theorem extends this relationship between 
prime ideal and maximal ideal:
\begin{theorem}
  If $R$ is a PID, then an ideal $I\neq R$ is prime if and only if $I$ is maximal.
\end{theorem}
\begin{proof}
  We only prove $\mbox{prime}\Rightarrow \mbox{maximal}$. If $I=(a)$ is prime, consider another ideal $J=(b)$ 
  that contains $I$, then we have $a\in J=(b)$. Thus, there exists $c\in R$, such that $a=bc$. Note that $bc=a\in I$ and 
  $I$ is prime. This indicates $b\in I$ or $c\in I$. If $b\in I$, we have $(b)\subseteq I$ and $(b)=I$. If $c\in I$, we further
  have some $d$ and $c = da$. Then:
  $$ a = bc = bda \stackrel{!}{\Longrightarrow} 1 = bd\quad(\mbox{$R$ is integral domain})
  $$
  $b$ is an unit and $J=(b)=R$. In conclusion, $J$ is either $I$ or $R$. 
  $I$ is maximal according to the definition.
\end{proof}
The set of all prime ideals of $R$ is called the \textit{spectrum} of $R$.
\end{document}