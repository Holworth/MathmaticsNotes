\documentclass[a4paper, pdf, 12pt]{article}
\usepackage{amsthm} %lets us use \begin{proof}
\usepackage{amsmath}
\usepackage{amssymb} %gives us the character \varnothing
\usepackage[dvipsnames]{xcolor}
\usepackage{enumitem}
\usepackage{pst-node}
\usepackage{auto-pst-pdf}
\usepackage{tikz-cd} 
\usepackage{mathrsfs}

\newcommand{\catname}[1]{{\normalfont\textbf{#1}}}
\newcommand{\divides}{\mid}
\newcommand{\notdivides}{\nmid}
\makeatletter
\renewenvironment{proof}[1][\proofname]{\par
  \pushQED{\qed}%
  \normalfont \topsep6\p@\@plus6\p@\relax
  \trivlist
  \item[%
    \hskip\labelsep
    \normalfont\bfseries % was \itshape
    #1%
    \@addpunct{.}% remove this if you don't want punctuation
  ]\ignorespaces
}{%
  \popQED\endtrivlist\@endpefalse
}
\let\qed\relax % avoid a warning
\DeclareRobustCommand{\qed}{%
  \ifmmode \mathqed
  \else
    \leavevmode\unskip\penalty\@M\hbox{}\nobreak\hspace{.5em minus .1em}% was \hfill
    \hbox{\qedsymbol}%
  \fi
}
\makeatother

\newlist{notes}{enumerate}{1}
\setlist[notes]{label=Note: ,leftmargin=*}
\setlist[itemize]{leftmargin=*}

\title{Chapter3 Rings and Modules}
\begin{document}
\section*{1.Definiton of ring}
\subsection*{1.3} 
Let $R$ be a ring, and let $S$ be any set. Explain how to endow the set $R^{S}$ of
set-functions $S \rightarrow R$ of two operations $+, ·$ so as to make $R^{S}$ into a ring, such that
$R^{S}$ is just a copy of $R$ if $S$ is a sigleton.
\begin{proof}
  The construction is straight forward, for any $f, g\in R^{S}$, let:
  $$
  f+g: S\rightarrow R, s\mapsto f(s) + g(s)
  $$
  $$
  fg: S\rightarrow R, s\mapsto f(s)g(s)
  $$
\end{proof}

\subsection*{1.12}
Just as complex numbers may be viewed as combinations $a + bi$, where $a, b \in \mathbb{R}$, and $i$ satisfies the relation 
$i^2 = −1$ (and commutes with $\mathbb{R}$), we may construct a ring $\mathbb{H}$ by considering linear 
combinations $a + bi + cj + dk$ where $a, b, c, d \in \mathbb{R}$, and $i, j, k$ commute with $\mathbb{R}$ and 
satisfy the following relations:
$$
i^2 = j^2 = k^2 = -1 , ij = -ji = k , jk = -kj = i , ki = -ik = j .
$$
\noindent
Addition in $\mathbb{H}$ is defined componentwise, while multiplication is defined by imposing distributivity 
and applying the relations. For example,
$$
(1+i+j)·(2+k) = 1·2+i·2+j·2+1·k+i·k+j·k = 2+2i+2j+k−j+i = 2+3i+j+k
$$
\noindent
\begin{enumerate}[leftmargin=0cm,itemindent=.2cm,labelwidth=\itemindent,labelsep=0.2cm,align=right,label=(\roman*)]
  \item Verify that this prescription does indeed define a ring.
  \item Compute $(a + bi + cj + dk)(a-bi-cj-dk)$, where $a, b, c, d \in \mathbb{R}$.
  \item Prove that $\mathbb{H}$ is a division ring\\
  Elements of $\mathbb{H}$ are called quaternions. Note that $\mathbb{Q}_8 := \{ \pm1, \pm i, \pm j, \pm k \}$ forms a subgroup of the group 
  of units of $\mathbb{H}$; it is a noncommutative group of order 8, called the quaternionic group.
  \item List all subgroups of $\mathbb{Q}_8$, and prove that they are all normal.
  \item Prove that $\mathbb{Q}_8$, $D_8$ are not isomorphic.
\end{enumerate}

\begin{proof}
  The proof is as follows:
  \begin{enumerate} [leftmargin=0cm,itemindent=.2cm,labelwidth=\itemindent,labelsep=0.2cm,align=right,label=(\roman*)]
    \item It's obviously the set $\mathbb{H}$ forms an abelian group where $0\in \mathbb{R}$ is the identity and each element 
    $a+bi+cj+dk$ has addition inverse $-a-bi-cj-dk$. For multiplication, the operation is close and has identity 1, 
    and distribution law is nativaly true because multiplication is defined in this way.

    \item 
    $$
    \begin{aligned}
      &(a + bi + cj + dk)(a -bi-cj-dk) \\
      &= a^2 - (bi + cj + dk)^2 \\
      & = a^2 - (-b^2-c^2-d^2 + bcij + bdik 
       + cdjk + bcji + bdki + cdkj)\\
      & = a^2+b^2+c^2+d^2
    \end{aligned}
    $$

    \item 
    To prove that $\mathbb{H}$ is a division ring, it suffices to show that each 
    element is an unit.
    According to (i), we have 
    $$(a + bi + cj + dk)(a-bi-cj-dk) = a^2 + b^2+c^2+d^2$$
    and:
    $$(a - bi - cj -dk)(a+bi+cj+dk) = a^2 + (-b)^2 + (-c)^2 + (-d)^2$$
    Thus, the multiplication inverse of $a + bi+cj+dk$ is $(a-bi-cj-dk) / (a^2 + b^2+c^2+d^2)$

    \item 
    Since the order of $\mathbb{Q}_{8}$ is 8, the only possible size of the subgroup of 
    $\mathbb{Q}_{8}$ could only be 2 and 4. For the first case, it's impossible since no element 
    of $\mathbb{Q}_{8}$ has order of 2. For the second case, recall that there are only two possible 
    structure of group with order 4: \\
    The first one is isomorphic to $\mathbb{Z}_{2}\times \mathbb{Z}_{2}$, 
    with means there are four elements of order 2, which is impossible as explained before.\\
    The second one is isomorphic to $\mathbb{Z}_{4}$, generated by an element of order 4. Thus, subgroups 
    of 4 are exactly $\{i, -1, -i, 1\}$ or $\{j, -1, -j, 1\}$, $\{k, -1, -k, 1\}$. For any element $g$ of 
    $\mathbb{Q}_{8}$, we have $gig^{-1}$ is still an element of this subgroup. Thus this
    subgroup is normal.

    \item TODO
  \end{enumerate}
\end{proof}

\subsection*{1.13}
Verify that the multiplication defined in $R[x]$ is associative.
\begin{proof}
  We have to prove for any $f(x), g(x), h(x)\in R[x]$, $(f(x)g(x))h(x) = f(x)(g(x)h(x))$. 
  Suppose that:
  $$
  \begin{aligned}
    f(x) &= \sum_{i=0}^{n}a_{i}x^{i},
    g(x) &= \sum_{i=0}^{m}b_{i}x^{i},
    h(x) &= \sum_{i=0}^{l}c_{i}x^{i}
  \end{aligned}
  $$
  Then for $(f(x)g(x))h(x)$ the coefficient of $x^{p}$ is:
  $$
  \sum_{i+j=p}(fg)_{i}h_{j} = \sum_{i+j=p}(fg)_{i}c_{j}=\sum_{i+j=p}(\sum_{k+l=i}a_{k}b_{l})c_{j}\stackrel{!}{=}\sum_{k+l+j=p}a_kb_lc_j
  $$
  Similarly, for $f(x)(g(x)h(x))$, the coefficient of $x^{p}$ is:
  $$
  \sum_{i+j=p}f_{i}(gh)_{j}=\sum_{i+j=p}f_{i}(\sum_{k+l=j}b_kc_l)\stackrel{!}{=}\sum_{i+k+l=p}a_{i}b_{k}c_{l}
  $$
  Note that the equation labeled with $!$ is induced by the associativity and distributive law of R itself.
\end{proof}

\subsection*{1.14}
Let $R$ be a ring, and let $f(x), g(x) \in R[x]$ be nonzero polynomials. Prove that 
$$
\deg(f(x) + g(x)) \leq \max(\deg(f(x)), \deg(g(x))) .
$$
\noindent
Assuming that R is an integral domain, prove that
$$
\deg(f(x) · g(x)) = \deg(f(x)) + \deg(g(x)).
$$
\begin{proof}
  Let $n=\deg (f(x) + g(x))$, then $\exists f_{i}\neq 0, i\geq n$ or $\exists g_{i}\neq 0, i\geq n$. Thus 
  $\max(\deg(f(x)), \deg(g(x)))\geq \deg (f(x) + g(x))$\\

  For the second part, let $n=\deg f(x), m=\deg g(x)$, then $(fg)_{n+m} = f_{n}g_{m}\neq 0$. And for any $i > n+m$, 
  we must have $(fg)_{i} = 0$ as $f_{i} = 0, i > n$ and $g_{i} = 0, i > m$.
\end{proof}

\subsection*{1.15}
Prove that $R[x]$ is an integral domain if and only if $R$ is an integral domain
\begin{proof}
  If $R[x]$ is an integral domain, then $R$ is an integral domain as $R$ can be viewed as 
  element of $R[x]$. If $R$ is integral domain, then $$\deg (fg) = \deg f + \deg g >= \max (\deg f, \deg g) \geq 0$$ 
  when $\deg f,\deg g\geq 0$. Thus $R[x]$ is an integral domain.
\end{proof}

\subsection*{1.16}
Let $R$ be a ring, and consider the ring of power series $R[[x]]$
\begin{enumerate} [leftmargin=0cm,itemindent=.2cm,labelwidth=\itemindent,labelsep=0.2cm,align=right,label=(\roman*)]
  \item Prove that a power series $a_0+ a_1x + a_2x^2+ \ldots$ is a unit in $R[[x]]$ if and only if
  $a_0$ is a unit in R. What is the inverse of $1−x$ in $R[[x]]$?
  \item Prove that $R[[x]]$ is an integral domain if and only if $R$ is.
\end{enumerate}
\begin{proof}
  The proof is as follows:
  \begin{enumerate} [leftmargin=0cm,itemindent=.2cm,labelwidth=\itemindent,labelsep=0.2cm,align=right,label=(\roman*)]
    \item If $a_0+ a_1x + a_2x^2+ \ldots$ has inverse, let the inverse be $b_0 + b_1x + b_2x^2 + \ldots$, then we have 
    $$
    \begin{aligned}
    1 &= (a_0+ a_1x + a_2x^2+ \ldots)(b_0 + b_1x + b_2x^2 + \ldots)\\
    &= a_0b_0 + (a_0b_1 + a_1b_0)x +(a_0b_2 + a_1b_1 + a_2b_0)x^2+\ldots\\
    \end{aligned}
    $$
    We must have $a_0b_0=1$, similarly we have $b_0a_0=1$. Thus indicates $a_0$ is an unit.\\
    \noindent
    On the other hand, if $a_0$ has inverse, we formally write the inverse of $f$ as:
    $f^{-1} = b_0 + b_1x + b_2x^{2} + \ldots$. Thus $ff^{-1} = 1$ implies the followsing 
    equations:
    $$
    \begin{aligned}
      &a_0b_0 = 1\\
      &a_0b_1 + a_1b_0 = 0\\
      &a_0b_2 + a_1b_1 + a_2b_0 = 0\\
      &a_0b_3 + a_1b_2 + a_2b_1 + a_3b_0 = 0\\
      &\ldots
    \end{aligned}
    $$
    $g$ is constructed by solve these equations:
    $$
    \begin{aligned}
      &b_0 = a_0^{-1}\\
      &b_1 = -a_0^{-1}a_1b_0\\
      &b_2 = -a_0^{-1}(a_1b_1 + a_2b_0)\\
      &\ldots\\
      &b_k = -a_0^{-1}(\sum_{i=1}^{k}a_{i}b_{k-i})
    \end{aligned}
    $$
    This indicates $f$ is an unit.

  \item If $f, g\in R[[x]]$ and $f, g\neq 0$. Then write them in the following form:
  $$
  f = x^{p}(a_p + a_{p+1}x + \ldots), g = x^{q}(b_q + b_{q+1}x + \ldots)
  $$
  Then $fg = x^{p+q}(a_pb_q + \ldots)\neq 0$. In addition, $R$ is Commutative indicates $R[[x]]$ is also
  commutative, thus $R[[x]]$ is an integral domain.
  \end{enumerate}
\end{proof}
\section*{2. Category \textbf{Ring}}
\subsection*{2.3}
Let $S$ be a set, and consider the power set ring $\mathscr{P}(S)$ (Exercise 1.2), and the ring 
$(\mathbb{Z}/2\mathbb{Z})^{S}$ you constructed in Exercise 1.3. Prove that these two rings are isomorphic. 
(Cf. Exercise I.2.11.)
\begin{proof}
  First note that $\mathscr{P}(S)$ and $(\mathbb{Z}/2\mathbb{Z})^{S}$ are isomorphic in 
  \textbf{Set}. For each $f\in (\mathbb{Z}/2\mathbb{Z})$, maps $f$ to $\varphi(f)$ by the 
  following subset of $S$:
  $$
  \varphi(f) = \{s\in S\mid f(s) = [1]_{2}\}
  $$
  Then it's easy to show that $\varphi$ is both bijective and a ring homomorphim, therefore a ring isomorphism.
\end{proof}

\subsection*{2.6}
Let $\alpha: R \rightarrow S$ be a fixed ring homomorphism, and let 
$s \in S$ be an element commuting with $\alpha(r)$ for all $r \in R$. Then there is a unique ring 
homomorphism $\overline{\alpha}: R[x] \rightarrow S$ extending $\alpha$, and sending $x$ to $s$
\begin{proof}
  Define $\overline{\alpha}$ as follows:
  $$
  \overline{\alpha}(\sum_{i\geq 0} a_ix^{i}) = \sum_{i\geq 0}\alpha(a_i)s^{i}
  $$
  To prove this is a ring homomorphism, we need to show that $\overline{\alpha}$ maintains both addition and
  multiplication(and send identity to identity, which is obvious). Addition is easy to verify, for multiplication, 
  it is worthy noted $s$ conmmutes with $\alpha(r), r\in R$ makes it maintains multiplication:
  $$
  \overline{\alpha}((\sum_{i\geq 0}a_i x^i)(\sum_{i\geq 0}b_i x^{i}))=\overline{\alpha}(\sum_{i\geq 0}(\sum_{k+l=i}a_kb_l)x^{i})=\sum_{i\geq 0}\alpha(\sum_{k+l=i}a_kb_l)s^{i}
  $$
  $$
  \begin{aligned}
  \overline{\alpha}(\sum_{i\geq 0}a_ix^{i})\overline{\alpha}(\sum_{i\geq 0}b_ix^{i})&=(\sum_{i\geq 0}\alpha(a_i)s^{i})(\sum_{i\geq 0}\alpha(b_i)s^{i})\\
  &=\sum_{i\geq 0}(\sum_{k+l=i}\alpha(a_k)s^{k}\alpha(b_l)s^{l})\\
  &\stackrel{!}{=}\sum_{i\geq 0}(\sum_{k+l=i}\alpha(a_k)\alpha(b_l)s^{i})\\
  &=\sum_{i\geq 0}(\alpha(\sum_{k+l=i}a_kb_l)s^{i})\\
  &=\overline{\alpha}((\sum_{i\geq 0}a_ix^{i})(\sum_{i\geq 0}b_ix^{i}))
  \end{aligned}
  $$
  Note that $!$ is true because $s$ commutates with all $\alpha(a_k)$ and $\alpha(b_l)$.
  The uniqueness of $\overline{\alpha}$ comes from the fact that $\overline{\alpha}$ is homomorphism, and $\overline{\alpha}(r) = \alpha(r), \overline{\alpha}(x)=s$.
\end{proof}
\noindent
\textbf{NOTE} Example 2.2 asks for particular situation, where a ring homomorphism $\varphi: \mathbb{Z}[x]\rightarrow S$ extends the unique homomorphism 
$f:\mathbb{Z}\rightarrow S, n\mapsto n1_{S}$ and sends $x$ to any element of $S$ doesn't 
necessarily consider the commutativity of $S$. The answer is clean here, any element $s\in S$ must commutes with the image 
of $f$ since $s(n1_{S}) = ns = (n1_{S})s$

\subsection*{2.9}
The center of a ring $R$ consists of the elements a such that $ar = ra$ for all $r \in R$. Prove that the center 
is a subring of R. Prove that the center of a division ring is a field.
\begin{proof}
  Denote the center of $R$ as $Z(R)$, then for any $s, t\in Z(R), r\in R$, we have $r(s-t) = rs - rt = sr - tr = (s-t)r$, which indicates 
  that $s-t\in Z(R)$. Thus, $Z(R)$ is an addition subgroup of $R$. \\

  Moreover, $\forall s,t\in Z(R), r\in R$, we have $(st)r = s(tr) = s(rt)=(sr)t=  (rs)t=r(st)$. Thus $rs\in Z(R)$, indicating $Z(R)$ is closed 
  under multiplication. The associativity and distributive law natively holds in $Z(R)$. And $1_{R}\in Z(R)$ obviousl. In conclusion, $Z(R)$
  is a subring of $R$.

  If $R$ is a division ring, for any $s\in Z(R)$, we must prove that $s^{-1}\in Z(R)$. Actually, 
  for any $s\in Z(R), r\in R, sr=rs\Rightarrow rs^{-1} = s^{-1}r$. Thus $s^{-1}\in Z(R)$. And $Z(R)$ 
  is obviously commutative, and therefore a field.
\end{proof}
\subsection*{2.10}
The \textit{centralizer} of an element a of a ring $R$ consists of the elements $r \in R$ such that 
$ar = ra$. Prove that the centralizer of $a$ is a subring of $R$, for every $a \in R$.
Prove that the center of $R$ is the intersection of all its centralizers.
Prove that every centralizer in a division ring is a division ring.
\begin{proof}
  To prove the centralizer of $a\in R$ is a subring of $R$ basically follows the same way as exercise 2.9 does. \\

  For the second part, if $s\in Z(R)$, then $r$ commutes with any element $r\in R$, thus $s\in \mbox{Cen}_{R}(r),r\in R$.
  and $s\in \bigcap_{r\in R}\mbox{Cen}_{R}(r)$, indicating $Z(R)\subseteq \bigcap_{r\in R}\mbox{Cen}_{R}(r)$. On the other 
  hand, any element of $\bigcap_{r\in R}\mbox{Cen}_{R}(r)$ must commute with any element of $R$, thus belongs to 
  $Z(R)$. In conclusion, $Z(R)=\bigcap_{r\in R}\mbox{Cen}_{R}(r)$.\\

  For the third part, it suffices to show that if $r$ commutes with $a$ then so does $r^{-1}$. It is done in exercise 2.9 already.
\end{proof}

\subsection*{2.11}
Let $R$ be a division ring consisting of $p^{2}$ elements, where $p$ is a prime. 
Prove that $R$ is commutative.
\begin{proof}
  Assume that $R$ is not commutative, consider the center of $R$, denoted as $Z(R)$. Then $Z(R)\neq R$. Note that 
  $Z(R)$ is an addition subgroup of $R$, Then it must have $\lvert Z(R)\rvert = p$ since $\lvert Z(R)\rvert$ divides 
  $\lvert R\rvert$, which is $p^2$.\\

  Consider one element $r\in R, r\notin Z(R)$, and its centralizer, denoted as $\mbox{Cen}_{R}(r)$, then since $r\notin Z(R)$,
  it means $\mbox{Cen}_{R}(r)\neq R$. And exercise 2.10 indicates $\mbox{Cen}_{R}(r)$ is a subring of $R$, thus $\lvert \mbox{Cen}_{R}(r)\rvert=p$.\\

  Exercise 2.10 also shows that $Z(R)\subseteq \mbox{Cen}_{R}(r)$, their cardinality equals to each other means $Z(R)=\mbox{Cen}_{R}(r)$. However, it's obvious
  that $r\in \mbox{Cen}_{R}(r)$ but $r\notin Z(R)$, a contradiction.\\

  In conclusion, we must have $Z(R)=R$ and $R$ is therefore commutative, further more, it's a field.
\end{proof}
\noindent
\textbf{NOTE}\quad In fact, any finite division ring is commutative, thus a field. But the proof used here seems hard to extend to more complex condition, i.e. the case 
of arbitary integer. Actually, it's even hard to extend this method to $p^{n}, n\geq 3$ case: $\lvert Z(R)\rvert$ might be 
$p^{3}$ and $\mbox{Cen}_{R}(r)$ might be $p^{2}$ and no contradictions so far.

\subsection*{2.12}
Consider the inclusion map $\iota: \mathbb{Z} \rightarrow \mathbb{Q}$ . Describe the cokernel of $\iota$ in $\mathbf{Ab}$, 
and its cokernel in $\mathbf{Ring}$ (as defined by the appropriate universal property in the style of the one given in § II.8.6)
\begin{proof}
  Before we describe the cokernel requested above, we will review what these concepts(and kernel) means in category 
  conception:\\
  \textbf{Kernel} Let $G, H$ be group and $f: G\rightarrow H$ is a group homomorphism. Then Consider the following 
  category: $\mathscr{K}_{\varphi}$: The object of $\mathscr{K}_{\varphi}$ is one group $S$ associated one morphism $j$, such that 
  the following diagram holds:
  $$
  \begin{tikzcd}[column sep=huge]
  S \arrow[r, "j"'] \arrow[rr, "0", bend left] & G \arrow[r, "f"'] & H
  \end{tikzcd}
  $$
  And the morphism between $(j_1, S_1)$ and $(j_2, S_2)$ is the following diagram:
  $$
  % https://tikzcd.yichuanshen.de/#N4Igdg9gJgpgziAXAbVABwnAlgFyxMJZARgBoAGAXVJADcBDAGwFcYkQBxEAX1PU1z5CKAEwVqdJq3YAJHnxAZseAkXLiaDFm0QgAygH0R8-sqFEyxCVum7DxHhJhQA5vCKgAZgCcIAWyR1EBwIJDJJbXYAHSiGbzQACywQGkZ6ACMYRgAFARVhEG8sFwScExAffyQxYNDEIJsdEAArIxSQNMycvPNdIpKy3i9fAMQAZhoQwM0pJtaHVIys3LNVPuLS8srRidrqmcjdGLjE5KGKkf29xHDMsCgkMYbZ9nJ2zuWetY6YT0HKbhAA
\begin{tikzcd}[column sep=huge]
  S_2 \arrow[r, "j_2"'] \arrow[rr, "0", bend left] & G \arrow[r, "f"']                     & H \\
                                                   & S_1 \arrow[u, "j_1"'] \arrow[lu, "\varphi"] &  
  \end{tikzcd}
  $$
  And $\ker \varphi$ is defined to be the final object of $\mathscr{K}_{\varphi}$. That is, the following diagram holds:
  $$
  % https://tikzcd.yichuanshen.de/#N4Igdg9gJgpgziAXAbVABwnAlgFyxMJZARgBoAGAXVJADcBDAGwFcYkQBxEAX1PU1z5CKAEwVqdJq3YAJHnxAZseAkXLiaDFm0QgAOnoDWMAE4ACAGbz+yoUTLEJW6boDKPCTCgBzeEVAWJhAAtkjqIDgQSGQgjPQARjCMAAoCKsIgJljeABY4IJpSOiBWvAFBoYhiEVGI4XGJKWl2ulm5+YXa7Ab4OPTWJRVIAMw0kWGdLiAAVgWxCUmptqqt2XkDgSFI1eOIMYlgUCPhzsXkcw2LzSuxMBb5ZYNbiKM125PFBjAAHlhwOHAAIQGBgmNA5LAebhAA
  \begin{tikzcd}[column sep=huge, ]
  \ker f \arrow[r, "\iota"'] \arrow[rr, "0", bend left] & G \arrow[r, "f"']                               & H \\
                                                        & S \arrow[u, "j"'] \arrow[lu, "\exists!\varphi"] &  
  \end{tikzcd}
  $$
  And $\ker f$ exists as $\ker f=\{g\in G\mid f(g) = 0\}$. It's easy to verify such set is a subgroup of $G$ and this 
  subgroup associated with the injection homomorphism satisfies the universal property of $\ker$.
  \\
  \\
  \noindent
  \textbf{Cokernel} Conceptually, cokernel just reverse all arrows in the above diagram. Let $G, H$ be groups and $f: G\rightarrow H$ is a 
  group homomorphism, consider the category $\mathscr{C}_{f}$ of which objects and morphisms are following diagrams:
  $$
  % https://tikzcd.yichuanshen.de/#N4Igdg9gJgpgziAXAbVABwnAlgFyxMJZABgBpiBdUkANwEMAbAVxiRAHEQBfU9TXfIRQBGclVqMWbABLdeIDNjwEiAJjHV6zVohABlAPqq5fJYKKjh4rVN2Hh3cTCgBzeEVAAzAE4QAtkhkIDgQSKIS2myeINQMdABGMAwACvzKQiDeWC4AFjgmID7+YdQhSOoRtiAAVkYxIHGJKWnmulm5+TxevgGIQWWIFYlgUEgAzEE2OiDEBUW94QNjmpLTtQ5dhT3jpaGDK5G6ADpH9N5oOVj1jUmpZipt2XmOXEA
\begin{tikzcd}[column sep=huge]
  G \arrow[r, "f"'] \arrow[rr, "0", bend left] & H \arrow[r, "j_2"'] \arrow[d, "j_1"] & S_2 \\
                                               & S_1 \arrow[ru, "\varphi"']           &    
  \end{tikzcd}
  $$
  And $\mbox{coker}f$ is an initial object in this category, that is, the following diagram holds:
  % https://tikzcd.yichuanshen.de/#N4Igdg9gJgpgziAXAbVABwnAlgFyxMJZABgBpiBdUkANwEMAbAVxiRAHEQBfU9TXfIRQBGclVqMWbABLdeIDNjwEiAJjHV6zVohABlOXyWCio4eK1TdAHWsBbAEYQAHsADGEANYwATlwBm3OIwUADm8ESg-j4QdkhkIDgQSKIS2myB1Ax0DjAMAAr8ykIgPlihABY4hiDRsSnUSUjqaVYgAFYgWTl5hcYqumWV1TxRMXGICU2ILblgUEgAzAmWOiDENXUTqdOLmpJrtmhYm+NLjckz++k21jDOWHA4cACEtvQ+aBUn3bkFRSZBuUqkEuEA
  $$
  \begin{tikzcd}[column sep=huge]
  G \arrow[r, "f"'] \arrow[rr, "0", bend left] & H \arrow[r, "j"'] \arrow[d, "\pi"]           & S \\
                                               & \mbox{coker}f \arrow[ru, "\exists!\varphi"'] &  
  \end{tikzcd}
  $$
  As we have proved before, in $\mathbf{Grp}$, $\mbox{coker}f$ is $H/N$, where $N$ is the smallest normal subgroup that 
  contains $\mbox{Im}f$. In particular, $\mbox{coker}f=H/\mbox{Im}f$ in $\mathbf{Ab}$.

  If we replace groups with rings and group homomorphisms with ring homomorphisms, we can naturaly get the definition of kernel and cokernel 
  in $\mathbf{Ring}$.\\

  Now back to the problem itslef,$\mbox{coker}\iota$ in $\mathbf{Ab}$, as stated, is $\mathbb{Q}/\mathbb{Z}$. The associated $\pi$ is $\pi(q) = q+\mathbb{Z}$. 
  And $\mbox{coker}\iota$ in $\mathbf{Ring}$ is $(0, \{0\})$. Actually if $(j, S)$ where $S$ is a ring  and $j$ is a ring homomorphism from $\mathbb{Q}$ to $S$, 
  if it satisfies $j\circ \iota = 0$. Then we have:
  $$
  j(\frac{p}{q}) = j(pq^{-1}) = j(p)j(q)^{-1}=j(\iota(p))j(\iota(q))=0(p)0(q)^{-1}=0
  $$
  Thus $j$ maps each element to be 0 in $S$, thus $S$ could only be $\{0\}$ since $1_{S}=f(1_{\mathbb{Q}})=0$. This indicates there is only one object in this category,
  and $\mbox{coker}\iota$ is this object.
\end{proof}
\subsection*{2.13}
Verify that the ‘componentwise’ product $R_1\times R_2$ of two rings satisfies the universal property for products in a category, given in § I.5.4
\begin{proof}
  $(R_1\times R_2, \pi_1, \pi_2)$ is the product of $R_1$ and $R_2$, where $\pi_1(r_1,r_2)=r_1$ and $\pi_2(r_1, r_2)=r_2$. It's easy to show that $\pi_1, \pi_2$ are 
  ring homomorphisms, we must show that the following diagrams holds:
  % https://tikzcd.yichuanshen.de/#N4Igdg9gJgpgziAXAbVABwnAlgFyxMJZAZgBoAGAXVJADcBDAGwFcYkQAlAfQEYQBfUuky58hFGQBM1Ok1btukgUJAZseAkUmkeMhizaJOvADom8AW3gACRcuHqxRcjr1zDnATJhQA5vCJQADMAJwgLJG0QHAgkF1kDdjM0LF4QGkZ6ACMYRgAFEQ1xEBCsXwALHHsQUPDImhikHhp9eSNk1KVBYLCIxDJo2MQo1o8zGAAPLDgcOABCMwYQtHKsatq+gcbEeNH2ILTumt6kLaHmhLaari7KfiA
  $$
  % https://tikzcd.yichuanshen.de/#N4Igdg9gJgpgziAXAbVABwnAlgFyxMJZAZgBoAGAXVJADcBDAGwFcYkQAlAfQEYQBfUuky58hFGQBM1Ok1btukgUJAZseAkUmkeMhizaJOvADom8AW3gACRcuHqxRcjr1zDnATJhQA5vCJQADMAJwgLJG0QHAgkF1kDdjM0LF4QGkZ6ACMYRgAFEQ1xEBCsXwALHHsQUPDImhikHhp9eSNk1KVBYLCIxDJo2MQo1o8zGAAPLDgcOABCMwYQtHKsatq+gcbEeNH2ILSaHLAoJGJybpres4ah5oS2mq4lI5gTpABac-5KfiA
\begin{tikzcd}
  &  &                                                        & R_1 \\
R \arrow[rr, "\exists!\varphi"] \arrow[rrru, "f_1", bend left] \arrow[rrrd, "f_2", bend right] &  & R_1\times R_2 \arrow[ru, "\pi_1"'] \arrow[rd, "\pi_2"] &     \\
  &  &                                                        & R_2
\end{tikzcd}
$$
For $(R, f_1, f_2)$, defines $\varphi: R\rightarrow R_1\times R_2, r\mapsto(f_1(r), f_2(r))$. Then the diagram is commutative. To prove the uniqueness, 
consider another ring homomorphism $\varphi^{'}:R\rightarrow R_1\times R_2$ makes this diagram commutes, then $\varphi^{'}(r)=(r_1, r_2)$. Further we have $f_1(r) = \pi_1(\varphi(r)) = \pi_1(r_1, r_2)=r_1, f_2(r)=\pi_2(\varphi(r)) = \pi_2(r_1,r_2) = r_2$. 
Thus $\varphi(r) = (f_1(r), f_2(r))$, the uniqueness is proved.\\

In conclusion, $(R_1\times R_2, \pi_1, \pi_2)$ is the product of $R_1$ and $R_2$.
\end{proof}
\subsection*{2.16}
Prove that there is (up to isomorphism) only one structure of ring with identity on the abelian group $(\mathbb{Z}, +)$.
\end{document}