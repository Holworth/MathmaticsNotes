\documentclass[a4paper, pdf, 12pt]{article}
\usepackage{amsthm} %lets us use \begin{proof}
\usepackage{amsmath}
\usepackage{amssymb} %gives us the character \varnothing
\usepackage[dvipsnames]{xcolor}
\usepackage{enumitem}
\usepackage{pst-node}
\usepackage{auto-pst-pdf}
\usepackage{tikz-cd} 
\usepackage{mathrsfs}

\newcommand{\catname}[1]{{\normalfont\textbf{#1}}}
\newcommand{\divides}{\mid}
\newcommand{\notdivides}{\nmid}

\newlist{notes}{enumerate}{1}
\setlist[notes]{label=Note: ,leftmargin=*}
\setlist[itemize]{leftmargin=*}

\title{Chapter3 Rings and Modules}
\begin{document}
\section*{1.Definiton of ring}
\subsection*{1.3} 
Let $R$ be a ring, and let $S$ be any set. Explain how to endow the set $R^{S}$ of
set-functions $S \rightarrow R$ of two operations $+, ·$ so as to make $R^{S}$ into a ring, such that
$R^{S}$ is just a copy of $R$ if $S$ is a sigleton.

\subsection*{1.12}
Just as complex numbers may be viewed as combinations $a + bi$, where $a, b \in \mathbb{R}$, and $i$ satisfies the relation 
$i^2 = −1$ (and commutes with $\mathbb{R}$), we may construct a ring $\mathbb{H}$ by considering linear 
combinations $a + bi + cj + dk$ where $a, b, c, d \in \mathbb{R}$, and $i, j, k$ commute with $\mathbb{R}$ and 
satisfy the following relations:
$$
i^2 = j^2 = k^2 = -1 , ij = -ji = k , jk = -kj = i , ki = -ik = j .
$$
\noindent
Addition in $\mathbb{H}$ is defined componentwise, while multiplication is defined by imposing distributivity 
and applying the relations. For example,
$$
(1+i+j)·(2+k) = 1·2+i·2+j·2+1·k+i·k+j·k = 2+2i+2j+k−j+i = 2+3i+j+k
$$
\noindent
\begin{enumerate}[leftmargin=0cm,itemindent=.2cm,labelwidth=\itemindent,labelsep=0.2cm,align=right,label=(\roman*)]
  \item Verify that this prescription does indeed define a ring.
  \item Compute $(a + bi + cj + dk)(a-bi-cj-dk)$, where $a, b, c, d \in \mathbb{R}$.
  \item Prove that $\mathbb{H}$ is a division ring\\
  Elements of $\mathbb{H}$ are called quaternions. Note that $\mathbb{Q}_8 := \{ \pm1, \pm i, \pm j, \pm k \}$ forms a subgroup of the group 
  of units of $\mathbb{H}$; it is a noncommutative group of order 8, called the quaternionic group.
  \item List all subgroups of $\mathbb{Q}_8$, and prove that they are all normal.
  \item Prove that $\mathbb{Q}_8$, $D_8$ are not isomorphic.
\end{enumerate}

\begin{proof}
  The proof is as follows:
  \begin{enumerate} [leftmargin=0cm,itemindent=.2cm,labelwidth=\itemindent,labelsep=0.2cm,align=right,label=(\roman*)]
    \item It's obviously the set $\mathbb{H}$ forms an abelian group where $0\in \mathbb{R}$ is the identity and each element 
    $a+bi+cj+dk$ has addition inverse $-a-bi-cj-dk$. For multiplication, the operation is close and has identity 1, 
    and distribution law is nativaly true because multiplication is defined in this way.

    \item 
    $$
    \begin{aligned}
      &(a + bi + cj + dk)(a -bi-cj-dk) \\
      &= a^2 - (bi + cj + dk)^2 \\
      & = a^2 - (-b^2-c^2-d^2 + bcij + bdik 
       + cdjk + bcji + bdki + cdkj)\\
      & = a^2+b^2+c^2+d^2
    \end{aligned}
    $$

    \item 
    To prove that $\mathbb{H}$ is a division ring, it suffices to show that each 
    element is an unit.
    According to (i), we have 
    $$(a + bi + cj + dk)(a-bi-cj-dk) = a^2 + b^2+c^2+d^2$$
    and:
    $$(a - bi - cj -dk)(a+bi+cj+dk) = a^2 + (-b)^2 + (-c)^2 + (-d)^2$$
    Thus, the multiplication inverse of $a + bi+cj+dk$ is $(a-bi-cj-dk) / (a^2 + b^2+c^2+d^2)$

    \item 
    Since the order of $\mathbb{Q}_{8}$ is 8, the only possible size of the subgroup of 
    $\mathbb{Q}_{8}$ could only be 2 and 4. For the first case, it's impossible since no element 
    of $\mathbb{Q}_{8}$ has order of 2. For the second case, recall that there are only two possible 
    structure of group with order 4: \\
    The first one is isomorphic to $\mathbb{Z}_{2}\times \mathbb{Z}_{2}$, 
    with means there are four elements of order 2, which is impossible as explained before.\\
    The second one is isomorphic to $\mathbb{Z}_{4}$, generated by an element of order 4. Thus, subgroups 
    of 4 are exactly $\{i, -1, -i, 1\}$ or $\{j, -1, -j, 1\}$, $\{k, -1, -k, 1\}$. For any element $g$ of 
    $\mathbb{Q}_{8}$, we have $gig^{-1}$ is still an element of this subgroup. Thus this
    subgroup is normal.

    \item TODO
  \end{enumerate}
\end{proof}

\subsection*{1.13}
Verify that the multiplication defined in $R[x]$ is associative.
\begin{proof}
  We have to prove for any $f(x), g(x), h(x)\in R[x]$, $(f(x)g(x))h(x) = f(x)(g(x)h(x))$. 
  Suppose that:
  $$
  \begin{aligned}
    f(x) &= \sum_{i=0}^{n}a_{i}x^{i},
    g(x) &= \sum_{i=0}^{m}b_{i}x^{i},
    h(x) &= \sum_{i=0}^{l}c_{i}x^{i}
  \end{aligned}
  $$
  Then for $(f(x)g(x))h(x)$ the coefficient of $x^{p}$ is:
  $$
  \sum_{i+j=p}(fg)_{i}h_{j} = \sum_{i+j=p}(fg)_{i}c_{j}=\sum_{i+j=p}(\sum_{k+l=i}a_{k}b_{l})c_{j}\stackrel{!}{=}\sum_{k+l+j=p}a_kb_lc_j
  $$
  Similarly, for $f(x)(g(x)h(x))$, the coefficient of $x^{p}$ is:
  $$
  \sum_{i+j=p}f_{i}(gh)_{j}=\sum_{i+j=p}f_{i}(\sum_{k+l=j}b_kc_l)\stackrel{!}{=}\sum_{i+k+l=p}a_{i}b_{k}c_{l}
  $$
  Note that the equation labeled with $!$ is induced by the associativity and distributive law of R itself.
\end{proof}

\subsection*{1.14}
Let $R$ be a ring, and let $f(x), g(x) \in R[x]$ be nonzero polynomials. Prove that 
$$
\deg(f(x) + g(x)) \leq \max(\deg(f(x)), \deg(g(x))) .
$$
\noindent
Assuming that R is an integral domain, prove that
$$
\deg(f(x) · g(x)) = \deg(f(x)) + \deg(g(x)).
$$
\begin{proof}
  Let $n=\deg (f(x) + g(x))$, then $\exists f_{i}\neq 0, i\geq n$ or $\exists g_{i}\neq 0, i\geq n$. Thus 
  $\max(\deg(f(x)), \deg(g(x)))\geq \deg (f(x) + g(x))$\\

  For the second part, let $n=\deg f(x), m=\deg g(x)$, then $(fg)_{n+m} = f_{n}g_{m}\neq 0$. And for any $i > n+m$, 
  we must have $(fg)_{i} = 0$ as $f_{i} = 0, i > n$ and $g_{i} = 0, i > m$.
\end{proof}

\subsection*{1.15}
Prove that $R[x]$ is an integral domain if and only if $R$ is an integral domain
\begin{proof}
  If $R[x]$ is an integral domain, then $R$ is an integral domain as $R$ can be viewed as 
  element of $R[x]$. If $R$ is integral domain, then $$\deg (fg) = \deg f + \deg g >= \max (\deg f, \deg g) \geq 0$$ 
  when $\deg f,\deg g\geq 0$. Thus $R[x]$ is an integral domain.
\end{proof}

\end{document}