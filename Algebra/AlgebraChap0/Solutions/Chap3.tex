\documentclass[a4paper, pdf, 12pt]{article}
\usepackage{amsthm} %lets us use \begin{proof}
\usepackage{amsmath}
\usepackage{amssymb} %gives us the character \varnothing
\usepackage[dvipsnames]{xcolor}
\usepackage{enumitem}
\usepackage{pst-node}
\usepackage{auto-pst-pdf}
\usepackage{tikz-cd} 
\usepackage{mathrsfs}

\newcommand{\catname}[1]{{\normalfont\textbf{#1}}}
\newcommand{\divides}{\mid}
\newcommand{\notdivides}{\nmid}
\makeatletter
\renewenvironment{proof}[1][\proofname]{\par
  \pushQED{\qed}%
  \normalfont \topsep6\p@\@plus6\p@\relax
  \trivlist
  \item[%
    \hskip\labelsep
    \normalfont\bfseries % was \itshape
    #1%
    \@addpunct{.}% remove this if you don't want punctuation
  ]\ignorespaces
}{%
  \popQED\endtrivlist\@endpefalse
}
\let\qed\relax % avoid a warning
\DeclareRobustCommand{\qed}{%
  \ifmmode \mathqed
  \else
    \leavevmode\unskip\penalty\@M\hbox{}\nobreak\hspace{.5em minus .1em}% was \hfill
    \hbox{\qedsymbol}%
  \fi
}
\makeatother

\newlist{notes}{enumerate}{1}
\setlist[notes]{label=Note: ,leftmargin=*}
\setlist[itemize]{leftmargin=*}

\title{Chapter3 Rings and Modules}
\begin{document}
\section*{1.Definiton of ring}
\subsection*{1.3} 
Let $R$ be a ring, and let $S$ be any set. Explain how to endow the set $R^{S}$ of
set-functions $S \rightarrow R$ of two operations $+, ·$ so as to make $R^{S}$ into a ring, such that
$R^{S}$ is just a copy of $R$ if $S$ is a sigleton.
\begin{proof}
  The construction is straight forward, for any $f, g\in R^{S}$, let:
  $$
  f+g: S\rightarrow R, s\mapsto f(s) + g(s)
  $$
  $$
  fg: S\rightarrow R, s\mapsto f(s)g(s)
  $$
\end{proof}

\subsection*{1.12}
Just as complex numbers may be viewed as combinations $a + bi$, where $a, b \in \mathbb{R}$, and $i$ satisfies the relation 
$i^2 = −1$ (and commutes with $\mathbb{R}$), we may construct a ring $\mathbb{H}$ by considering linear 
combinations $a + bi + cj + dk$ where $a, b, c, d \in \mathbb{R}$, and $i, j, k$ commute with $\mathbb{R}$ and 
satisfy the following relations:
$$
i^2 = j^2 = k^2 = -1 , ij = -ji = k , jk = -kj = i , ki = -ik = j .
$$
\noindent
Addition in $\mathbb{H}$ is defined componentwise, while multiplication is defined by imposing distributivity 
and applying the relations. For example,
$$
(1+i+j)·(2+k) = 1·2+i·2+j·2+1·k+i·k+j·k = 2+2i+2j+k−j+i = 2+3i+j+k
$$
\noindent
\begin{enumerate}[leftmargin=0cm,itemindent=.2cm,labelwidth=\itemindent,labelsep=0.2cm,align=right,label=(\roman*)]
  \item Verify that this prescription does indeed define a ring.
  \item Compute $(a + bi + cj + dk)(a-bi-cj-dk)$, where $a, b, c, d \in \mathbb{R}$.
  \item Prove that $\mathbb{H}$ is a division ring\\
  Elements of $\mathbb{H}$ are called quaternions. Note that $\mathbb{Q}_8 := \{ \pm1, \pm i, \pm j, \pm k \}$ forms a subgroup of the group 
  of units of $\mathbb{H}$; it is a noncommutative group of order 8, called the quaternionic group.
  \item List all subgroups of $\mathbb{Q}_8$, and prove that they are all normal.
  \item Prove that $\mathbb{Q}_8$, $D_8$ are not isomorphic.
\end{enumerate}

\begin{proof}
  The proof is as follows:
  \begin{enumerate} [leftmargin=0cm,itemindent=.2cm,labelwidth=\itemindent,labelsep=0.2cm,align=right,label=(\roman*)]
    \item It's obviously the set $\mathbb{H}$ forms an abelian group where $0\in \mathbb{R}$ is the identity and each element 
    $a+bi+cj+dk$ has addition inverse $-a-bi-cj-dk$. For multiplication, the operation is close and has identity 1, 
    and distribution law is nativaly true because multiplication is defined in this way.

    \item 
    $$
    \begin{aligned}
      &(a + bi + cj + dk)(a -bi-cj-dk) \\
      &= a^2 - (bi + cj + dk)^2 \\
      & = a^2 - (-b^2-c^2-d^2 + bcij + bdik 
       + cdjk + bcji + bdki + cdkj)\\
      & = a^2+b^2+c^2+d^2
    \end{aligned}
    $$

    \item 
    To prove that $\mathbb{H}$ is a division ring, it suffices to show that each 
    element is an unit.
    According to (i), we have 
    $$(a + bi + cj + dk)(a-bi-cj-dk) = a^2 + b^2+c^2+d^2$$
    and:
    $$(a - bi - cj -dk)(a+bi+cj+dk) = a^2 + (-b)^2 + (-c)^2 + (-d)^2$$
    Thus, the multiplication inverse of $a + bi+cj+dk$ is $(a-bi-cj-dk) / (a^2 + b^2+c^2+d^2)$

    \item 
    Since the order of $\mathbb{Q}_{8}$ is 8, the only possible size of the subgroup of 
    $\mathbb{Q}_{8}$ could only be 2 and 4. For the first case, it's impossible since no element 
    of $\mathbb{Q}_{8}$ has order of 2. For the second case, recall that there are only two possible 
    structure of group with order 4: \\
    The first one is isomorphic to $\mathbb{Z}_{2}\times \mathbb{Z}_{2}$, 
    with means there are four elements of order 2, which is impossible as explained before.\\
    The second one is isomorphic to $\mathbb{Z}_{4}$, generated by an element of order 4. Thus, subgroups 
    of 4 are exactly $\{i, -1, -i, 1\}$ or $\{j, -1, -j, 1\}$, $\{k, -1, -k, 1\}$. For any element $g$ of 
    $\mathbb{Q}_{8}$, we have $gig^{-1}$ is still an element of this subgroup. Thus this
    subgroup is normal.

    \item TODO
  \end{enumerate}
\end{proof}

\subsection*{1.13}
Verify that the multiplication defined in $R[x]$ is associative.
\begin{proof}
  We have to prove for any $f(x), g(x), h(x)\in R[x]$, $(f(x)g(x))h(x) = f(x)(g(x)h(x))$. 
  Suppose that:
  $$
  \begin{aligned}
    f(x) &= \sum_{i=0}^{n}a_{i}x^{i},
    g(x) &= \sum_{i=0}^{m}b_{i}x^{i},
    h(x) &= \sum_{i=0}^{l}c_{i}x^{i}
  \end{aligned}
  $$
  Then for $(f(x)g(x))h(x)$ the coefficient of $x^{p}$ is:
  $$
  \sum_{i+j=p}(fg)_{i}h_{j} = \sum_{i+j=p}(fg)_{i}c_{j}=\sum_{i+j=p}(\sum_{k+l=i}a_{k}b_{l})c_{j}\stackrel{!}{=}\sum_{k+l+j=p}a_kb_lc_j
  $$
  Similarly, for $f(x)(g(x)h(x))$, the coefficient of $x^{p}$ is:
  $$
  \sum_{i+j=p}f_{i}(gh)_{j}=\sum_{i+j=p}f_{i}(\sum_{k+l=j}b_kc_l)\stackrel{!}{=}\sum_{i+k+l=p}a_{i}b_{k}c_{l}
  $$
  Note that the equation labeled with $!$ is induced by the associativity and distributive law of R itself.
\end{proof}

\subsection*{1.14}
Let $R$ be a ring, and let $f(x), g(x) \in R[x]$ be nonzero polynomials. Prove that 
$$
\deg(f(x) + g(x)) \leq \max(\deg(f(x)), \deg(g(x))) .
$$
\noindent
Assuming that R is an integral domain, prove that
$$
\deg(f(x) · g(x)) = \deg(f(x)) + \deg(g(x)).
$$
\begin{proof}
  Let $n=\deg (f(x) + g(x))$, then $\exists f_{i}\neq 0, i\geq n$ or $\exists g_{i}\neq 0, i\geq n$. Thus 
  $\max(\deg(f(x)), \deg(g(x)))\geq \deg (f(x) + g(x))$\\

  For the second part, let $n=\deg f(x), m=\deg g(x)$, then $(fg)_{n+m} = f_{n}g_{m}\neq 0$. And for any $i > n+m$, 
  we must have $(fg)_{i} = 0$ as $f_{i} = 0, i > n$ and $g_{i} = 0, i > m$.
\end{proof}

\subsection*{1.15}
Prove that $R[x]$ is an integral domain if and only if $R$ is an integral domain
\begin{proof}
  If $R[x]$ is an integral domain, then $R$ is an integral domain as $R$ can be viewed as 
  element of $R[x]$. If $R$ is integral domain, then $$\deg (fg) = \deg f + \deg g >= \max (\deg f, \deg g) \geq 0$$ 
  when $\deg f,\deg g\geq 0$. Thus $R[x]$ is an integral domain.
\end{proof}

\subsection*{1.16}
Let $R$ be a ring, and consider the ring of power series $R[[x]]$
\begin{enumerate} [leftmargin=0cm,itemindent=.2cm,labelwidth=\itemindent,labelsep=0.2cm,align=right,label=(\roman*)]
  \item Prove that a power series $a_0+ a_1x + a_2x^2+ \ldots$ is a unit in $R[[x]]$ if and only if
  $a_0$ is a unit in R. What is the inverse of $1−x$ in $R[[x]]$?
  \item Prove that $R[[x]]$ is an integral domain if and only if $R$ is.
\end{enumerate}
\begin{proof}
  The proof is as follows:
  \begin{enumerate} [leftmargin=0cm,itemindent=.2cm,labelwidth=\itemindent,labelsep=0.2cm,align=right,label=(\roman*)]
    \item If $a_0+ a_1x + a_2x^2+ \ldots$ has inverse, let the inverse be $b_0 + b_1x + b_2x^2 + \ldots$, then we have 
    $$
    \begin{aligned}
    1 &= (a_0+ a_1x + a_2x^2+ \ldots)(b_0 + b_1x + b_2x^2 + \ldots)\\
    &= a_0b_0 + (a_0b_1 + a_1b_0)x +(a_0b_2 + a_1b_1 + a_2b_0)x^2+\ldots\\
    \end{aligned}
    $$
    We must have $a_0b_0=1$, similarly we have $b_0a_0=1$. Thus indicates $a_0$ is an unit.\\
    \noindent
    On the other hand, if $a_0$ has inverse, we formally write the inverse of $f$ as:
    $f^{-1} = b_0 + b_1x + b_2x^{2} + \ldots$. Thus $ff^{-1} = 1$ implies the followsing 
    equations:
    $$
    \begin{aligned}
      &a_0b_0 = 1\\
      &a_0b_1 + a_1b_0 = 0\\
      &a_0b_2 + a_1b_1 + a_2b_0 = 0\\
      &a_0b_3 + a_1b_2 + a_2b_1 + a_3b_0 = 0\\
      &\ldots
    \end{aligned}
    $$
    $g$ is constructed by solve these equations:
    $$
    \begin{aligned}
      &b_0 = a_0^{-1}\\
      &b_1 = -a_0^{-1}a_1b_0\\
      &b_2 = -a_0^{-1}(a_1b_1 + a_2b_0)\\
      &\ldots\\
      &b_k = -a_0^{-1}(\sum_{i=1}^{k}a_{i}b_{k-i})
    \end{aligned}
    $$
    This indicates $f$ is an unit.

  \item If $f, g\in R[[x]]$ and $f, g\neq 0$. Then write them in the following form:
  $$
  f = x^{p}(a_p + a_{p+1}x + \ldots), g = x^{q}(b_q + b_{q+1}x + \ldots)
  $$
  Then $fg = x^{p+q}(a_pb_q + \ldots)\neq 0$. In addition, $R$ is Commutative indicates $R[[x]]$ is also
  commutative, thus $R[[x]]$ is an integral domain.
  \end{enumerate}
\end{proof}
\section*{2. Category \textbf{Ring}}
\subsection*{2.3}
Let $S$ be a set, and consider the power set ring $\mathscr{P}(S)$ (Exercise 1.2), and the ring 
$(\mathbb{Z}/2\mathbb{Z})^{S}$ you constructed in Exercise 1.3. Prove that these two rings are isomorphic. 
(Cf. Exercise I.2.11.)
\begin{proof}
  First note that $\mathscr{P}(S)$ and $(\mathbb{Z}/2\mathbb{Z})^{S}$ are isomorphic in 
  \textbf{Set}. For each $f\in (\mathbb{Z}/2\mathbb{Z})$, maps $f$ to $\varphi(f)$ by the 
  following subset of $S$:
  $$
  \varphi(f) = \{s\in S\mid f(s) = [1]_{2}\}
  $$
  Then it's easy to show that $\varphi$ is both bijective and a ring homomorphim, therefore a ring isomorphism.
\end{proof}

\subsection*{2.6}
Let $\alpha: R \rightarrow S$ be a fixed ring homomorphism, and let 
$s \in S$ be an element commuting with $\alpha(r)$ for all $r \in R$. Then there is a unique ring 
homomorphism $\overline{\alpha}: R[x] \rightarrow S$ extending $\alpha$, and sending $x$ to $s$
\begin{proof}
  Define $\overline{\alpha}$ as follows:
  $$
  \overline{\alpha}(\sum_{i\geq 0} a_ix^{i}) = \sum_{i\geq 0}\alpha(a_i)s^{i}
  $$
  To prove this is a ring homomorphism, we need to show that $\overline{\alpha}$ maintains both addition and
  multiplication(and send identity to identity, which is obvious). Addition is easy to verify, for multiplication, 
  it is worthy noted $s$ conmmutes with $\alpha(r), r\in R$ makes it maintains multiplication:
  $$
  \overline{\alpha}((\sum_{i\geq 0}a_i x^i)(\sum_{i\geq 0}b_i x^{i}))=\overline{\alpha}(\sum_{i\geq 0}(\sum_{k+l=i}a_kb_l)x^{i})=\sum_{i\geq 0}\alpha(\sum_{k+l=i}a_kb_l)s^{i}
  $$
  $$
  \begin{aligned}
  \overline{\alpha}(\sum_{i\geq 0}a_ix^{i})\overline{\alpha}(\sum_{i\geq 0}b_ix^{i})&=(\sum_{i\geq 0}\alpha(a_i)s^{i})(\sum_{i\geq 0}\alpha(b_i)s^{i})\\
  &=\sum_{i\geq 0}(\sum_{k+l=i}\alpha(a_k)s^{k}\alpha(b_l)s^{l})\\
  &\stackrel{!}{=}\sum_{i\geq 0}(\sum_{k+l=i}\alpha(a_k)\alpha(b_l)s^{i})\\
  &=\sum_{i\geq 0}(\alpha(\sum_{k+l=i}a_kb_l)s^{i})\\
  &=\overline{\alpha}((\sum_{i\geq 0}a_ix^{i})(\sum_{i\geq 0}b_ix^{i}))
  \end{aligned}
  $$
  Note that $!$ is true because $s$ commutates with all $\alpha(a_k)$ and $\alpha(b_l)$.
  The uniqueness of $\overline{\alpha}$ comes from the fact that $\overline{\alpha}$ is homomorphism, and $\overline{\alpha}(r) = \alpha(r), \overline{\alpha}(x)=s$.
\end{proof}
\noindent
\textbf{NOTE} Example 2.2 asks for particular situation, where a ring homomorphism $\varphi: \mathbb{Z}[x]\rightarrow S$ extends the unique homomorphism 
$f:\mathbb{Z}\rightarrow S, n\mapsto n1_{S}$ and sends $x$ to any element of $S$ doesn't 
necessarily consider the commutativity of $S$. The answer is clean here, any element $s\in S$ must commutes with the image 
of $f$ since $s(n1_{S}) = ns = (n1_{S})s$

\subsection*{2.9}
The center of a ring $R$ consists of the elements a such that $ar = ra$ for all $r \in R$. Prove that the center 
is a subring of R. Prove that the center of a division ring is a field.
\begin{proof}
  Denote the center of $R$ as $Z(R)$, then for any $s, t\in Z(R), r\in R$, we have $r(s-t) = rs - rt = sr - tr = (s-t)r$, which indicates 
  that $s-t\in Z(R)$. Thus, $Z(R)$ is an addition subgroup of $R$. \\

  Moreover, $\forall s,t\in Z(R), r\in R$, we have $(st)r = s(tr) = s(rt)=(sr)t=  (rs)t=r(st)$. Thus $rs\in Z(R)$, indicating $Z(R)$ is closed 
  under multiplication. The associativity and distributive law natively holds in $Z(R)$. And $1_{R}\in Z(R)$ obviousl. In conclusion, $Z(R)$
  is a subring of $R$.

  If $R$ is a division ring, for any $s\in Z(R)$, we must prove that $s^{-1}\in Z(R)$. Actually, 
  for any $s\in Z(R), r\in R, sr=rs\Rightarrow rs^{-1} = s^{-1}r$. Thus $s^{-1}\in Z(R)$. And $Z(R)$ 
  is obviously commutative, and therefore a field.
\end{proof}
\subsection*{2.10}
The \textit{centralizer} of an element a of a ring $R$ consists of the elements $r \in R$ such that 
$ar = ra$. Prove that the centralizer of $a$ is a subring of $R$, for every $a \in R$.
Prove that the center of $R$ is the intersection of all its centralizers.
Prove that every centralizer in a division ring is a division ring.
\begin{proof}
  To prove the centralizer of $a\in R$ is a subring of $R$ basically follows the same way as exercise 2.9 does. \\

  For the second part, if $s\in Z(R)$, then $r$ commutes with any element $r\in R$, thus $s\in \mbox{Cen}_{R}(r),r\in R$.
  and $s\in \bigcap_{r\in R}\mbox{Cen}_{R}(r)$, indicating $Z(R)\subseteq \bigcap_{r\in R}\mbox{Cen}_{R}(r)$. On the other 
  hand, any element of $\bigcap_{r\in R}\mbox{Cen}_{R}(r)$ must commute with any element of $R$, thus belongs to 
  $Z(R)$. In conclusion, $Z(R)=\bigcap_{r\in R}\mbox{Cen}_{R}(r)$.\\

  For the third part, it suffices to show that if $r$ commutes with $a$ then so does $r^{-1}$. It is done in exercise 2.9 already.
\end{proof}

\subsection*{2.11}
Let $R$ be a division ring consisting of $p^{2}$ elements, where $p$ is a prime. 
Prove that $R$ is commutative.
\begin{proof}
  Assume that $R$ is not commutative, consider the center of $R$, denoted as $Z(R)$. Then $Z(R)\neq R$. Note that 
  $Z(R)$ is an addition subgroup of $R$, Then it must have $\lvert Z(R)\rvert = p$ since $\lvert Z(R)\rvert$ divides 
  $\lvert R\rvert$, which is $p^2$.\\

  Consider one element $r\in R, r\notin Z(R)$, and its centralizer, denoted as $\mbox{Cen}_{R}(r)$, then since $r\notin Z(R)$,
  it means $\mbox{Cen}_{R}(r)\neq R$. And exercise 2.10 indicates $\mbox{Cen}_{R}(r)$ is a subring of $R$, thus $\lvert \mbox{Cen}_{R}(r)\rvert=p$.\\

  Exercise 2.10 also shows that $Z(R)\subseteq \mbox{Cen}_{R}(r)$, their cardinality equals to each other means $Z(R)=\mbox{Cen}_{R}(r)$. However, it's obvious
  that $r\in \mbox{Cen}_{R}(r)$ but $r\notin Z(R)$, a contradiction.\\

  In conclusion, we must have $Z(R)=R$ and $R$ is therefore commutative, further more, it's a field.
\end{proof}
\noindent
\textbf{NOTE}\quad In fact, any finite division ring is commutative, thus a field. But the proof used here seems hard to extend to more complex condition, i.e. the case 
of arbitary integer. Actually, it's even hard to extend this method to $p^{n}, n\geq 3$ case: $\lvert Z(R)\rvert$ might be 
$p^{3}$ and $\mbox{Cen}_{R}(r)$ might be $p^{2}$ and no contradictions so far.

\subsection*{2.12}
Consider the inclusion map $\iota: \mathbb{Z} \rightarrow \mathbb{Q}$ . Describe the cokernel of $\iota$ in $\mathbf{Ab}$, 
and its cokernel in $\mathbf{Ring}$ (as defined by the appropriate universal property in the style of the one given in § II.8.6)
\begin{proof}
  Before we describe the cokernel requested above, we will review what these concepts(and kernel) means in category 
  conception:\\
  \textbf{Kernel} Let $G, H$ be group and $f: G\rightarrow H$ is a group homomorphism. Then Consider the following 
  category: $\mathscr{K}_{\varphi}$: The object of $\mathscr{K}_{\varphi}$ is one group $S$ associated one morphism $j$, such that 
  the following diagram holds:
  $$
  \begin{tikzcd}[column sep=huge]
  S \arrow[r, "j"'] \arrow[rr, "0", bend left] & G \arrow[r, "f"'] & H
  \end{tikzcd}
  $$
  And the morphism between $(j_1, S_1)$ and $(j_2, S_2)$ is the following diagram:
  $$
  % https://tikzcd.yichuanshen.de/#N4Igdg9gJgpgziAXAbVABwnAlgFyxMJZARgBoAGAXVJADcBDAGwFcYkQBxEAX1PU1z5CKAEwVqdJq3YAJHnxAZseAkXLiaDFm0QgAygH0R8-sqFEyxCVum7DxHhJhQA5vCKgAZgCcIAWyR1EBwIJDJJbXYAHSiGbzQACywQGkZ6ACMYRgAFARVhEG8sFwScExAffyQxYNDEIJsdEAArIxSQNMycvPNdIpKy3i9fAMQAZhoQwM0pJtaHVIys3LNVPuLS8srRidrqmcjdGLjE5KGKkf29xHDMsCgkMYbZ9nJ2zuWetY6YT0HKbhAA
\begin{tikzcd}[column sep=huge]
  S_2 \arrow[r, "j_2"'] \arrow[rr, "0", bend left] & G \arrow[r, "f"']                     & H \\
                                                   & S_1 \arrow[u, "j_1"'] \arrow[lu, "\varphi"] &  
  \end{tikzcd}
  $$
  And $\ker \varphi$ is defined to be the final object of $\mathscr{K}_{\varphi}$. That is, the following diagram holds:
  $$
  % https://tikzcd.yichuanshen.de/#N4Igdg9gJgpgziAXAbVABwnAlgFyxMJZARgBoAGAXVJADcBDAGwFcYkQBxEAX1PU1z5CKAEwVqdJq3YAJHnxAZseAkXLiaDFm0QgAOnoDWMAE4ACAGbz+yoUTLEJW6boDKPCTCgBzeEVAWJhAAtkjqIDgQSGQgjPQARjCMAAoCKsIgJljeABY4IJpSOiBWvAFBoYhiEVGI4XGJKWl2ulm5+YXa7Ab4OPTWJRVIAMw0kWGdLiAAVgWxCUmptqqt2XkDgSFI1eOIMYlgUCPhzsXkcw2LzSuxMBb5ZYNbiKM125PFBjAAHlhwOHAAIQGBgmNA5LAebhAA
  \begin{tikzcd}[column sep=huge, ]
  \ker f \arrow[r, "\iota"'] \arrow[rr, "0", bend left] & G \arrow[r, "f"']                               & H \\
                                                        & S \arrow[u, "j"'] \arrow[lu, "\exists!\varphi"] &  
  \end{tikzcd}
  $$
  And $\ker f$ exists as $\ker f=\{g\in G\mid f(g) = 0\}$. It's easy to verify such set is a subgroup of $G$ and this 
  subgroup associated with the injection homomorphism satisfies the universal property of $\ker$.
  \\
  \\
  \noindent
  \textbf{Cokernel} Conceptually, cokernel just reverse all arrows in the above diagram. Let $G, H$ be groups and $f: G\rightarrow H$ is a 
  group homomorphism, consider the category $\mathscr{C}_{f}$ of which objects and morphisms are following diagrams:
  $$
  % https://tikzcd.yichuanshen.de/#N4Igdg9gJgpgziAXAbVABwnAlgFyxMJZABgBpiBdUkANwEMAbAVxiRAHEQBfU9TXfIRQBGclVqMWbABLdeIDNjwEiAJjHV6zVohABlAPqq5fJYKKjh4rVN2Hh3cTCgBzeEVAAzAE4QAtkhkIDgQSKIS2myeINQMdABGMAwACvzKQiDeWC4AFjgmID7+YdQhSOoRtiAAVkYxIHGJKWnmulm5+TxevgGIQWWIFYlgUEgAzEE2OiDEBUW94QNjmpLTtQ5dhT3jpaGDK5G6ADpH9N5oOVj1jUmpZipt2XmOXEA
\begin{tikzcd}[column sep=huge]
  G \arrow[r, "f"'] \arrow[rr, "0", bend left] & H \arrow[r, "j_2"'] \arrow[d, "j_1"] & S_2 \\
                                               & S_1 \arrow[ru, "\varphi"']           &    
  \end{tikzcd}
  $$
  And $\mbox{coker}f$ is an initial object in this category, that is, the following diagram holds:
  % https://tikzcd.yichuanshen.de/#N4Igdg9gJgpgziAXAbVABwnAlgFyxMJZABgBpiBdUkANwEMAbAVxiRAHEQBfU9TXfIRQBGclVqMWbABLdeIDNjwEiAJjHV6zVohABlOXyWCio4eK1TdAHWsBbAEYQAHsADGEANYwATlwBm3OIwUADm8ESg-j4QdkhkIDgQSKIS2myB1Ax0DjAMAAr8ykIgPlihABY4hiDRsSnUSUjqaVYgAFYgWTl5hcYqumWV1TxRMXGICU2ILblgUEgAzAmWOiDENXUTqdOLmpJrtmhYm+NLjckz++k21jDOWHA4cACEtvQ+aBUn3bkFRSZBuUqkEuEA
  $$
  \begin{tikzcd}[column sep=huge]
  G \arrow[r, "f"'] \arrow[rr, "0", bend left] & H \arrow[r, "j"'] \arrow[d, "\pi"]           & S \\
                                               & \mbox{coker}f \arrow[ru, "\exists!\varphi"'] &  
  \end{tikzcd}
  $$
  As we have proved before, in $\mathbf{Grp}$, $\mbox{coker}f$ is $H/N$, where $N$ is the smallest normal subgroup that 
  contains $\mbox{Im}f$. In particular, $\mbox{coker}f=H/\mbox{Im}f$ in $\mathbf{Ab}$.

  If we replace groups with rings and group homomorphisms with ring homomorphisms, we can naturaly get the definition of kernel and cokernel 
  in $\mathbf{Ring}$.\\

  Now back to the problem itslef,$\mbox{coker}\iota$ in $\mathbf{Ab}$, as stated, is $\mathbb{Q}/\mathbb{Z}$. The associated $\pi$ is $\pi(q) = q+\mathbb{Z}$. 
  And $\mbox{coker}\iota$ in $\mathbf{Ring}$ is $(0, \{0\})$. Actually if $(j, S)$ where $S$ is a ring  and $j$ is a ring homomorphism from $\mathbb{Q}$ to $S$, 
  if it satisfies $j\circ \iota = 0$. Then we have:
  $$
  j(\frac{p}{q}) = j(pq^{-1}) = j(p)j(q)^{-1}=j(\iota(p))j(\iota(q))=0(p)0(q)^{-1}=0
  $$
  Thus $j$ maps each element to be 0 in $S$, thus $S$ could only be $\{0\}$ since $1_{S}=f(1_{\mathbb{Q}})=0$. This indicates there is only one object in this category,
  and $\mbox{coker}\iota$ is this object.
\end{proof}
\subsection*{2.13}
Verify that the ‘componentwise’ product $R_1\times R_2$ of two rings satisfies the universal property for products in a category, given in § I.5.4
\begin{proof}
  $(R_1\times R_2, \pi_1, \pi_2)$ is the product of $R_1$ and $R_2$, where $\pi_1(r_1,r_2)=r_1$ and $\pi_2(r_1, r_2)=r_2$. It's easy to show that $\pi_1, \pi_2$ are 
  ring homomorphisms, we must show that the following diagrams holds:
  % https://tikzcd.yichuanshen.de/#N4Igdg9gJgpgziAXAbVABwnAlgFyxMJZAZgBoAGAXVJADcBDAGwFcYkQAlAfQEYQBfUuky58hFGQBM1Ok1btukgUJAZseAkUmkeMhizaJOvADom8AW3gACRcuHqxRcjr1zDnATJhQA5vCJQADMAJwgLJG0QHAgkF1kDdjM0LF4QGkZ6ACMYRgAFEQ1xEBCsXwALHHsQUPDImhikHhp9eSNk1KVBYLCIxDJo2MQo1o8zGAAPLDgcOABCMwYQtHKsatq+gcbEeNH2ILTumt6kLaHmhLaari7KfiA
  $$
  % https://tikzcd.yichuanshen.de/#N4Igdg9gJgpgziAXAbVABwnAlgFyxMJZAZgBoAGAXVJADcBDAGwFcYkQAlAfQEYQBfUuky58hFGQBM1Ok1btukgUJAZseAkUmkeMhizaJOvADom8AW3gACRcuHqxRcjr1zDnATJhQA5vCJQADMAJwgLJG0QHAgkF1kDdjM0LF4QGkZ6ACMYRgAFEQ1xEBCsXwALHHsQUPDImhikHhp9eSNk1KVBYLCIxDJo2MQo1o8zGAAPLDgcOABCMwYQtHKsatq+gcbEeNH2ILSaHLAoJGJybpres4ah5oS2mq4lI5gTpABac-5KfiA
\begin{tikzcd}
  &  &                                                        & R_1 \\
R \arrow[rr, "\exists!\varphi"] \arrow[rrru, "f_1", bend left] \arrow[rrrd, "f_2", bend right] &  & R_1\times R_2 \arrow[ru, "\pi_1"'] \arrow[rd, "\pi_2"] &     \\
  &  &                                                        & R_2
\end{tikzcd}
$$
For $(R, f_1, f_2)$, defines $\varphi: R\rightarrow R_1\times R_2, r\mapsto(f_1(r), f_2(r))$. Then the diagram is commutative. To prove the uniqueness, 
consider another ring homomorphism $\varphi^{'}:R\rightarrow R_1\times R_2$ makes this diagram commutes, then $\varphi^{'}(r)=(r_1, r_2)$. Further we have $f_1(r) = \pi_1(\varphi(r)) = \pi_1(r_1, r_2)=r_1, f_2(r)=\pi_2(\varphi(r)) = \pi_2(r_1,r_2) = r_2$. 
Thus $\varphi(r) = (f_1(r), f_2(r))$, the uniqueness is proved.\\

In conclusion, $(R_1\times R_2, \pi_1, \pi_2)$ is the product of $R_1$ and $R_2$.
\end{proof}
\subsection*{2.16}
Prove that there is (up to isomorphism) only one structure of ring with identity on the abelian group $(\mathbb{Z}, +)$.
\section*{3.Ideals and quotient rings}
\subsection*{3.2}
Let $\varphi: R \rightarrow S$ be a ring homomorphism, and let $J$ be an ideal of $S$. 
Prove that $I = \varphi^{-1}(J)$ is an ideal of $R$. Thus, the inverse image of an image is 
also an ideal, is the image of an ideal also an ideal? Prove it or given a counterexample.
\begin{proof}
  For any $s\in \varphi^{-1}(J), r\in R$, we have $\varphi(rs) = \varphi(r)\varphi(s)\in J, \varphi(sr)=\varphi(s)\varphi(r)\in J$ since $\varphi(s)\in J, \varphi(r)\in R$, which indicates 
  that $rs\in \varphi^{-1}(J),sr\in \varphi^{-1}(J)$. Thus $\varphi^{-1}(J)$ is an ideal.

  Then second proposition is false in general, the ring homomorphism image of an ideal is not necessarily
  an ideal. Consider injection: $\iota: \mathbb{Z}\rightarrow \mathbb{Z}[x]$. However, the image 
  of an ideal, say $2\mathbb{Z}$ is still $2\mathbb{Z}\subseteq \mathbb{Z}[x]$ and is not an ideal of 
  $\mathbb{Z}$.

  However, if $\varphi$ is surjective, then $\varphi(I)$ is also an ideal of the target ring.
\end{proof}
\subsection*{3.3}
Let $\varphi: R \rightarrow S$ be a ring homomorphism, and let $J$ be an ideal of $R$.
\begin{enumerate}
  \item Show that $\varphi(J)$ need not be an ideal of $S$.

  \item Assume that $\varphi$ is surjective; then prove that $\varphi(J)$ is an ideal of $S$.
  
  \item Assume that $\varphi$ is surjective, and let $I = \ker \varphi$ ; thus we may identify 
  $S$ with $R/I$. Let $\overline{J} = \varphi(J)$, an ideal of $R/I$ by the previous point. Prove that
  $$
  \cfrac{R/I}{\overline{J}}\cong \frac{R}{I+J}
  $$
  \begin{proof}
    The first proposition are proved in exercise 3.2, and the second 
    one is easy to be proved following the definition. 

    For the third proposition, note that actually we have $\overline{J}\cong (I+J)/J$, then according 
    to proposition 3.14, we have:
    $$
    \frac{R/I}{\overline{J}}\cong \frac{R/J}{(I+J)/J}\cong R/(I+J)
    $$
    The proof is done.
  \end{proof}
  \subsection*{3.4}
  Let $R$ be a ring such that every subgroup of $(R, +)$ is in fact an ideal of $R$. Prove that 
  $R \cong \mathbb{Z}/n\mathbb{Z}$, where $n$ is the characteristic of $R$
  \begin{proof}
    Consider the subset:
    $$
    S=\{n1_{R}\mid n\in \mathbb{Z}\}
    $$
    It is a subgroup of $(R, +)$ because: 
    $$
    (\forall a1_{R}, b1_{R}\in S, a,b\in \mathbb{Z}):\quad a1_{R}-b1_{R}=(a-b)1_{R}\in S
    $$
    According to the assumption, we have $S$ to be an ideal, in particular, we have:
    $$
    (\forall r\in R): \quad r = r1_{R} \in S
    $$
    this indicates that $\forall r\in R, r=m1_{R}$ for some $m\in \mathbb{Z}$. And therefore 
    $R=S$. This indicates $R\cong \mathbb{Z}$ or $R\cong \mathbb{Z}/n\mathbb{Z}$ for $n$ to be 
    the characteristic of $R$.
  \end{proof}
  \subsection*{3.8}
  Prove that a ring $R$ is a division ring if and only if its only left-ideals and right-ideals 
  are $\{ 0 \}$ and $R$.
  In particular, a commutative ring $R$ is a field if and only if the only ideals of $R$ are $\{0\}$ 
  and $R$.
  \begin{proof}
    Let $R$ be a ring, and $I$ be an ideal of $R$. Then if $I$ contains element 
    other than $0_{R}$, we will have $I=R$ since $1_{R}\in I$. Thus the ideal of $R$ can only be 
    $R$ and $\{0\}$ if $R$ is a division ring. 

    On the other hand, if $R$ has only $\{0\}$ and $R$ as ideals, then any element of $R$ must be 
    an unit, otherwise $aR$ where $a$ is a non-unit, could be a right-ideal, a contradiction.

    The second part of this problem is nothing more than a special case of 
    field.
  \end{proof}
  \subsection*{3.9}
  Counterpoint to Exercise 3.8: it is not true that a ring $R$ is a division ring if and only if 
  its only two-sided ideals are $\{0\}$ and $R$. A nonzero ring with this property is said to be 
  simple; by Exercise 3.8, fields are the only simple commutative rings.
  \begin{proof}
    If $R$ is a division ring, then the ideals of $R$ could only be $R$ or $\{0\}$. However, the ideals
    of $R$ are only $\{0\}$ and $R$ doesn't mean both left-ideals and right-ideals of $R$ are only 
    $\{0\}$ and $R$.
  \end{proof}
\end{enumerate}
\subsection*{3.11}
Let $R$ be a ring containing $\mathbb{C}$ as a subring. Prove that there are no ring 
homomorphisms $R \rightarrow \mathbb{R}$
\begin{proof}
  If there exists some ring homomorphism $R\rightarrow \mathbb{R}$, then it induce 
  a ring homomorphism from $\mathbb{C}$ to $\mathbb{R}$. However, this can not be true
  because:
  $$
  -1 = f(-1) = f(\mathbf{i} * \mathbf{i}) = f(\mathbf{i})^{2}
  $$
  There is no such $f(\mathbf{i})\in \mathbb{R}$ satisfies $f(\mathbf{i})^{2}=-1$
\end{proof}
\subsection*{3.12}
Let $R$ be a commutative ring. Prove that the set of nilpotent elements of $R$ is an ideal of $R$. 
(Cf. Exercise 1.6. This ideal is called the \textit{nilradical} of $R$.)
Find a non-commutative ring in which the set of nilpotent elements is not an ideal.
\begin{proof}
  Let $N$ denotes the set of all nilpotent elements of $R$, first to prove that 
  $N$ is a subgroup of $(R, +)$. For any $a, b\in N$, there exists some $m,n\in \mathbb{N}^{+}$
  that $a^{m} = 0, b^{n} = 0$, then we shall have $(a-b)^{m+n+1} = 0$(using binomial theorem). This
  indicates $a-b\in N$, and thus $N$ is a subgroup of $(R, +)$.

  The second part is to prove that for any $r\in R, a\in N, ra\in N$. Note that $(ra)^{m} = r^{m}a^{m} = r^{m}0=0$.
  Thus $ra\in N$. In conclusion, we have $N$ is an ideal of $R$.

  One counterexample for non-commutative case would be matrix ring $M_{n}(\mathbb{R})$. Note that 
  $\begin{pmatrix}
    0 & 0\\
    1 & 0
  \end{pmatrix}$ is a nilpotent element but 
  $\begin{pmatrix}
    0 & 0\\
    1 & 0
  \end{pmatrix}
  \begin{pmatrix}
    1 & 1\\
    0 & 1
  \end{pmatrix}=
  \begin{pmatrix}
    0 & 0\\
    1 & 1
  \end{pmatrix}
  $ is not, which fails to make $N(M_{n}(\mathbb{R}))$ to be 
  an ideal.\\
  \\
  \\
  \noindent
  \textbf{NOTE} There might be some properties of this ideal, one most notable is that the 
  quotient ring $R/N$ has no non-naive nilpotent element:
  $$
  (a+N)^{m} = 0_{R/N}\Rightarrow a^{m} + N = 0_{R/N}\Rightarrow a^{m}\in N\Rightarrow a\in N
  $$
\end{proof}
\subsection*{3.13}
Let $R$ be a commutative ring, and let $N$ be its nilradical (cf. Exercise 3.12). 
Prove that $R/N$ contains no nonzero nilpotent elements.(Such a ring is said to be reduced.)
\begin{proof}
  The proof is done in the "NOTE" section of exercise 3.12
\end{proof}
\subsection*{3.14}
Prove that the characteristic of an integral domain is either $0$ or a prime integer. 
Do you know any ring of characteristic 1?
\begin{proof}
  If the characteristic of $R$ is non-prime, say $\mbox{char}R = mn, m>1, n>1$. Then 
  the definion of characteristic shows that $mn1_{R}=0$, which is $(m1_{R})(n1_{R}) = 0$. 
  Note that $m>1, n>1$ indicates $m < \mbox{char}R, n < \mbox{char}R$, thus $m1_{R}\neq 0, 
  n1_{R}\neq 0$. The equation $(m1_{R})(n1_{R})=0$ implies the multiplication of two non-zero 
  elements is zero, which contradicts the definition of integral domain.

  Ring of characteristic 1 could only be zero ring.
\end{proof}

\subsection*{3.15}
A ring $R$ is \textit{boolean} if $a^2= a$ for all $a \in R$. Prove that $\mathscr{P}(S)$ is boolean, 
for every set $S$ 
(cf. Exercise 1.2). Prove that every boolean ring is commutative, and has characteristic 2. 
Prove that if an integral domain $R$ is boolean, then $R \cong \mathbb{Z}/2\mathbb{Z}$
\begin{proof}
  $\mathscr{P}(S)$ is boolean as for any element $S\in \mathscr{P}(S)$ we have $S^{2} = S\cap S=S$.
  First we prove that if $R$ is \textit{boolean}, then for each element $r\in R$, we have $2r=0$, thus 
  the characteristic of $R$ is 2. Consider the following two equations:
  $$
  \begin{aligned}
  (1+r) = (1+r)^{2} = 1+2r+r^{2} = 1+2r + r
  \end{aligned}
  $$
  This indicates $\forall r\in R, 2r=0$. Further, $\forall a,b\in R$, we have:
  $$
  (a+b) = (a+b)^{2} = a^2 + ab + ba + b^{2} = a + ab + ba + b
  $$ This means $ab+ba = 0$, note that $2ab=0$, these two equations imply $ab=ba, \forall a,b\in R$. Thus $R$ is commutative. The 
  characteristic part is proved already. 

  If $R$ is itself an integral domain, then for any element $r\in R$, we have:
  $$
  r^2 = r \Rightarrow r(r-1_{R}) = 0\Rightarrow r = 1_{R}
  $$
  This implies there are only two elements of $R$ if it is boolean and domain, thus is isomorphic to 
  $\mathbb{Z}/2\mathbb{Z}$.
\end{proof}

\subsection*{3.17}
Let $I, J$ be ideals of a ring $R$. State and prove a precise result relating the
ideals $(I + J)/I$ of $R/I$ and $J/(I \cap J)$ of $R/(I \cap J)$
\begin{proof}
  $(I + J)/I$ is an ideal of quotient ring $R/I$. It's obvious that $I+J$ is an
  ideal that contains $I$. And there is, actually a one-to-one corespondence between 
  the ideal of $R/I$ and the ideal of $R$ that contains $I$. 

  Considering the canonical project: $\pi: R\rightarrow R/I, r\mapsto r+I$. The for each 
  ideal of $R/I$, say $S$, $\pi^{-1}(S)$ is an ideal of $R$ and it contains $I$. This 
  map: $S\mapsto \pi^{-1}(S)$ has one inverse function: $J\mapsto J/I$. Thus the bijection 
  exists.
\end{proof}

\section*{4. Ideals and quotients: remarks and examples}
\subsection*{4.2}
Prove that the homomorphic image of a Noetherian ring is Noetherian. That is, 
prove that if $\varphi: R \rightarrow S$ is a surjective ring homomorphism, and 
$R$ is Noetherian, then $S$ is Noetherian.

\begin{proof}
  Recall that Noetherian ring is a ring where all ideals are finitely generated.
  Let $J$ be an ideal of $S$, then $I=\varphi^{-1}(J)$ is an ideal of $R$. Then $R$ is 
  finitely generated, say $I = (r_1, r_2,\ldots r_n)$. Then for any element $p\in J$, we 
  have $p = \varphi(q), q\in I$, thus $q=\sum_{i=1}^{n}a_ir_i$ and 
  $p = \varphi(q) = \sum_{i=1}^{n}\varphi(a_i)\varphi(r_i)$. Thus, $J\subseteq (\varphi(r_1), \varphi(r_2), \ldots, \varphi(r_n))$
  And is finitely generated.
\end{proof}

\subsection*{4.5}
Let $I, J$ be ideals in a ring $R$, such that $I + J = (1)$. Prove that $IJ = I \cap J$
\begin{proof}
  Recall that $IJ$ denotes the ideal generated by all production $ij, i\in I, j\in J$. And 
  $IJ\subseteq I\cap J$ in general. We have to show $I\cap J\subset IJ$. For any element $r\in I\cap J$, 
  we have $r = r1_{R} = r(i + j) = ri + rj, i\in I, j\in J$. Note that $ri = ir\in IJ, rj\in IJ$, thus 
  $r = ri+rj \in IJ$, and we have $IJ\subseteq I\cap J$ as a result.
\end{proof}

\subsection*{4.6}
Let $I, J$ be ideals in a ring $R$. Assume that $R/(IJ)$ is reduced (that is, it has no 
nonzero nilpotent elements; cf. Exercise 3.13). Prove that $IJ = I \cap J$.
\begin{proof}
  If $IJ\subsetneq I\cap J$, then there is some element $r\in I\cap J, r\notin IJ$. Then 
  consider $r+IJ\in R/(IJ)$. We are gonna to have 
  $$
  (r+IJ)^2 = r^2 + IJ = IJ
  $$ as $r^2 \in IJ(r\in I, r\in J)$, which contradicts the assumption
  that $R/(IJ)$ is reduced. In conclusion, $IJ = I\cap J$.
\end{proof}

\subsection*{4.9}
Generalize the result of Exercise 4.8, as follows. Let $R$ be a ring, and let $f(x)$ be a left-zero-divisor in $R[x]$. 
Prove that $\exists b \in R, b \neq 0$, such that $f(x)b = 0$.
\begin{proof}
  We prove by induction on the degree of $f(x)$. If $\deg f(x) = 0$, then $f(x)$ is simply 
  an element of $R$, written as $r$. Say $f(x)g(x) = 0, g(x)\neq 0$. Then it's easy to see 
  the first coefficient of $g(x)$, say $t$, satisfies $rt = 0$. Thus the proposition 
  is true of $\deg f(x) = 0$.

  Assume that for $\deg f(x) = k$ the proposition is true. Then for $k+1$ case, 
\end{proof}

\subsection*{4.10}
Let $d$ be an integer that is not the square of an integer, and consider the subset of $\mathbb{C}$ defined by:
$$
\mathbb{Q} (\sqrt{d}) := \{ a + b \sqrt{d} \mid a, b \in \mathbb{Q} \} .
$$

\begin{enumerate}
  \item Prove that $\mathbb{Q}(\sqrt{d})$ is a subring of $\mathbb{C}$.
  \item Define a function $N : \mathbb{Q} (\sqrt{d}) \rightarrow \mathbb{Z}$ by $N(a + b\sqrt{d}) := a^2-b^2d$.
Prove that
    $N(zw) = N(z)N(w)$, and that $N(z) \neq 0$ if $z \in \mathbb{Q}(\sqrt{d}), z \neq 0.$

  \item Prove that $Q (\sqrt{d})$ is a field, and in fact the smallest subfield of $\mathbb{C}$ containing
  both $\mathbb{Q}$ and $\sqrt{d}$ (Use $N$).

  \item Prove that $\mathbb{Q}(\sqrt{d}) \cong \mathbb{Q}[t]/(t^2-d)$. (Cf. Example 4.8.)
\end{enumerate}
The function $N$ is a ‘norm’; it is very useful in the study of $Q (\sqrt{d})$ and of its subrings. (Cf. also Exercise 2.5.)
\begin{proof}
  The proof is as follows:
  \begin{enumerate}
  \item $Q(\sqrt{d})$ is indeed a subring of $\mathbb{C}$ because it's a subgroup of $\mathbb{C}$ and closed under 
  multiplication: 
  $$
  (\forall a_1+b_1\sqrt{d}, a_2+b_2\sqrt{d}\in \mathbb{Q}(\sqrt{d})):\quad 
  $$
  $$
  \begin{aligned}
    (a_1+b_1\sqrt{d}) - (a_2+b_2\sqrt{d}) &= (a_1 - a_2) + (b_1-b_2)\sqrt{d}\in \mathbb{Q}(\sqrt{d})\\
    (a_1+b_1\sqrt{d})(a_2 + b_2\sqrt{d}) &= (a_1a_2 + b_1b_2d) + (a_1b_2 + a_2b_1)\sqrt{d}\in \mathbb{Q}(\sqrt{d})
  \end{aligned}
  $$
  Also, $1_{\mathbb{C}}\in \mathbb{Q}(\sqrt{d})$ by setting $a = 1, b = 0$. In conclusion, $\mathbb{Q}(\sqrt{d})$ is a subring of 
  $\mathbb{C}$.

  \item For the second part, let $z = a_1+b_1\sqrt{d}, w = a_2+b_2\sqrt{d}$. Then:
  $$
  \begin{aligned}
    N(zw) &= (a_1a_2+b_1b_2d)^2 - d(a_1b_2 + a_2b_1)^2\\
    &= a_1^2a_2^2 + b_1^2b_2^2d^2 - da_1^2b_2^2-da_2^2b_1^2\\
    &= (a_1^2-b_1^2d)(a_2^2-b_2^2d)\\
    &= N(z)N(w)
  \end{aligned}
  $$
  If $N(z) = 0$, then $a^2-b^2d=0\Rightarrow a/b=\sqrt{d}$, contrdicts the fact that $\sqrt{d}$ is irrational.

  \item $\mathbb{Q}(\sqrt{d})$ is a field since each non-zero element $a+b\sqrt{d}$ has inverse $(a-b\sqrt{d})/(a^2-b^2d)$. Note that 
  we have proved that in (2), $N(z) = a^2-b^2d = 0$ if and only if $z = 0$, thus it's ok to write $a^2-b^2d$ as denominator.
  
  \item Note that $\mathbb{Q}[t]/(t^2 - d)\cong \mathbb{Q}^{\oplus 2}$, there is a one to one corespondence:
  $$
  \mathbb{Q}(\sqrt{d})\rightarrow \mathbb{Q}^{\oplus 2}: a+b\sqrt{d}\mapsto (a, b)
  $$
  And the multiplication defined over $\mathbb{Q}^{\oplus 2}$ is $(a, b)(e, f)=(ae+dbf, af+be)$
\end{enumerate}
The proof is done.
\end{proof}

\subsection*{4.11}
Let $R$ be a commutative ring, $a \in R$, and 
$f_1(x), \ldots , f_r(x) \in R[x]$.
\begin{enumerate}
  \item Prove the equality of ideals
  $$
  (f_1 (x), \ldots, f_r (x), x - a) = (f_1 (a), \ldots , f_r (a), x - a) .
  $$

  \item Prove the useful substitution trick
  $$
  \frac{R[x]}{(f_1(x), \ldots, f_r(x), x-a)}\cong 
  \frac{R}{(f_1(a), \ldots, f_r(a))}
  $$
\end{enumerate}
\begin{proof}
  (1) Since $x-a$ is a mononic polynomial, then for each $f_i(x)$, there exists one 
  $g_i(x), t_i(x)$ such that $\deg t_i(x) < \deg (x-a) = 1$. And:
  $$
  f_i(x) = g_i(x)(x-a) + t_i(x)
  $$
  Let $x=a$, we fill have: $t_i(a) = f_i(a)$. Note that $\deg t_i(x) < 1$, then we must 
  have $t_i(x) = f_i(a)\in R$, which means:
  $$
  f_i(x) = (x-a)g_i(x) + f_i(a)
  $$
  Indicating: $(f_i(x))\subseteq (x-a, f_i(a))$. Thus we have:
  $$
  (f_1(x), \ldots ,f_r(x))\subseteq (f_1(a), \ldots , f_r(a), x-a)
  $$
  Also, $f_i(x) = (x-a)g_i(x) + f_i(a)$ indicates $f_i(a) = f_i(x) - (x-a)g_i(x)$, and $(f_i(a))\subseteq (f_i(x), x-a)$. Similarly 
  we have:
  $$
  (f_1(a), \ldots , f_r(a), x-a)\subseteq(f_1(x), \ldots ,f_r(x))
  $$
  In conclusion, we have:
  $$
  (f_1(a), \ldots , f_r(a), x-a) = (f_1(x), \ldots ,f_r(x))
  $$\\
  (2)Consider the following ring homomorphism:
  $$
  R[x]\longrightarrow R \longrightarrow \frac{R}{(f_1(a),\ldots,f_r(a))}
  $$
  $$
  f(x)\mapsto f(a)\mapsto f(a) + (f_1(a),\ldots, f_r(a))
  $$
  Then it's easy to see that this homomorphism is surjective, since $R[x]\longrightarrow R$ is surjective 
  and $R\longrightarrow \cfrac{R}{(f_1(a),\ldots, f_r(a))}$ is surjective.

  Denote this ring homomorphism as $\varphi$, consider $\ker \varphi$:
  $$
  \ker \varphi = \{f(x)\in R[x]\mid f(a) \in (f_1(a), \ldots, f_r(a))\}
  $$
  Note that for each $f(x)\in (f_1(x),\ldots, f_r(x), x-a)$ we have:
  $$
  f(x) = \sum_{i=1}^{r}r_i(x)f_i(x) + r(x)(x-a)
  $$ and $f(a) = \sum_{i=1}^{r}r_i(a)f_i(a)\in (f_1(a),\ldots,f_r(a))$, this implies that 
  $$
  (f_1(x), \ldots, f_r(x))\subseteq \ker \varphi
  $$
  On the other hand, let $f(x)\in \ker\varphi$, using remainder divison, we have:
  $$
  f(x) = g(x)(x-a) + f(a)
  $$
  Note that $f(a)\in (f_1(a), \ldots, f_r(a))$, thus $f(x)\in (f_1(a),\ldots, f_r(a), x-a)$. Thus we have:
  $$
  f(x)\subseteq (f_1(a), \ldots, f_r(a), x-a) = (f_1(x),\ldots, f_r(x), x-a)
  $$ and $$
  \ker \varphi = (f_1(x),\ldots, f_r(x), x-a)
  $$
  According to the fundamental homomorphism theorem, we have:
  $$
  \cfrac{R[x]}{(f_1(x),\ldots f_r(x), x-a)}\cong \cfrac{R}{(f_1(a), \ldots, f_r(a))}
  $$
  The proof is done.
\end{proof}

\subsection*{4.12}
Let $R$ be a commutative ring, and $a_1,\ldots, a_n$ elements of $R$. Prove that
$$
\cfrac{R[x_1,\ldots x_n]}{(x_1-a_1,\ldots x_n-a_n)}\cong R
$$
\begin{proof}
  Prove by induction, for $n=1$, we have $$
  \cfrac{R[x]}{(x-a)}\cong R
  $$
  Assume for $k$ the proposition is true, then for $k+1$:
  $$
  \begin{aligned}
  \cfrac{R[x_1\ldots x_k, x_{k+1}]}{(x_1-a_1,\ldots x_{k}-a_{k},x_{k+1}-a_{k+1})}&=
  \cfrac{R[x_1,\ldots x_{k}][x_{k+1}]}{(x_1-a_1,\ldots, x_{k+1}-a_{k+1})}\\
    &\cong \cfrac{R[x_1,\ldots x_{k}]}{(x_1-a_1,\ldots x_k-a_k)}\\
    &\cong R
  \end{aligned}
  $$
  The most important step is to view $x_1-a_1,\ldots x_k-a_k$ as constant elements of 
  $R[x_1,\ldots, x_k]$ and using exercise 4.11
\end{proof}

\subsection*{4.13}
Let $R$ be an integral domain. For all $k = 1,\ldots, n$ prove that $(x_1,\ldots, x_k)$
is prime in $R[x_1,\ldots, x_n]$.
\begin{proof}
  For $k=1,\ldots, n$, we have:
  $$
  \cfrac{R[x_1,\ldots x_n]}{(x_1,\ldots x_k)}=\cfrac{R[x_{k+1},\ldots x_{n}][x_1,\ldots x_k]}{(x_1,\ldots, x_k)}
  \cong R[x_{k+1},\ldots x_n]\;(\mbox{exercise4.12})
  $$
  Note that $R[x_{k+1}, \ldots, x_{n}]$ is integral domain, thus $(x_1,\ldots x_{k})$ is prime.
\end{proof}

\subsection*{4.14}
Prove ‘by hand’ that maximal ideals are prime, without using quotient rings.
\begin{proof}
  Let $M$ be one maximal ideal of $R$, and $ab\in M$. We assert that $a\in M$ or $b\in M$, otherwise
  $a\notin R, b\notin R$, consider $(a)$ and $(b)$. We have $M\subsetneqq (a), M\subsetneqq (b)$. Thus we
  must have $(a) = (b) = R$, which implies $a, b$ are units. Thus $ab$ are units and $M = R$,a contradiction.
\end{proof}

\subsection*{4.15}
Let $\varphi: R \rightarrow S$ be a homomorphism of commutative rings, and let $I \subseteq S$ be an ideal. 
Prove that if $I$ is a prime ideal in $S$, then $\varphi^{-1} (I)$ is a prime ideal in $R$. 
Show that $\varphi^{-1}(I)$ is not necessarily maximal if $I$ is maximal.
\begin{proof}
  If $ab\in \varphi^{-1}(I)$, then $\varphi(ab)=\varphi(a)\varphi(b)\in I$. Note that $I$ is prime, then 
  $\varphi(a)\in I$ or $\varphi(b)\in I$. Thus $a\in \varphi^{-1}(I)$ or $b\in \varphi^{-1}(I)$. Indicating 
  $\varphi^{-1}(I)$ is prime.
  One counterexample is consider $\varphi: \mathbb{R}\rightarrow \mathbb{R}[x], r\mapsto r$. Then the 
  ideal $(x^{2} + 1)$ is maximal in $\mathbb{R}[x]$, but the inverse image $\varphi^{-1}((x^{2} + 1))$
  is $\{0\}$, which is not maximal in $\mathbb{R}$.
\end{proof}

\subsection*{4.16}
Let $R$ be a commutative ring, and let $P$ be a prime ideal of $R$. Suppose 0 is the only zero-divisor of $R$ contained in $P$. 
Prove that $R$ is an integral domain.
\begin{proof}
  If $a,b\neq 0$ but $ab=0$, then $ab\in P$ and $a\in P$ or $b\in P$, say $a\in P$, then $a=0$ since 
  $0$ is the only zero-divisor contained in $P$, a contradiction. Thus there is no non-zero zero divisor, 
  and $R$ is an integral domain.
\end{proof}

\subsection*{4.18}
Let $R$ be a commutative ring, and let $N$ be its nilradical (Exercise 3.12). 
Prove that $N$ is contained in every prime ideal of $R$.
\begin{proof}
  Recall that the nilradical of $R$ is the set consists of all nilponent elements of $R$
  (and is an ideal). Note that for each $P$, where $P$ is a prime ideal, and any element 
  $r\in N$, $\exists n\in \mathbb{N}^{+}$, s.t. $r^{n} = 0$. We have 
  $r^{n}=0\in N $, thus $r\in N$ or $r^{n-1}\in N$. If $r\in N$, we're done, otherwise 
  $r^{n-1}\in N$ indicates $r\in N$ or $r^{n-2}\in N$. Repeate the process we will have $r\in N$
  at last. Thus, $N\subseteq P$ for any prime ideal $P$.
\end{proof}
\noindent
\textbf{NOTE} Actually $N=\bigcap_{P is prime}P$

\subsection*{4.19}
Let $R$ be a commutative ring, let $P$ be a prime ideal in $R$, and let $I_j$ be ideals 
of $R$.
\begin{enumerate} [leftmargin=0.6cm,itemindent=.2cm,labelwidth=\itemindent,labelsep=0.2cm,align=right,label=(\roman*)]
\item Assume that $I_1\cdots I_r \subseteq P$; prove that $I_j \subseteq P$ for some $j$.
\item By (i), if $\cap_{j=1}^{r}I_j\subseteq P$ , then $P$ contains one of the ideals $I_j$ . Prove or disprove:
if $\cap_{j=1}^{\infty}I_j\subseteq P$ , then $P$ contains one of the ideals $I_j$ .
\end{enumerate}

\begin{proof}
  Here are proofs:
\begin{enumerate} [leftmargin=0.6cm,itemindent=.2cm,labelwidth=\itemindent,labelsep=0.2cm,align=right,label=(\roman*)]
  \item If for any $I_{i}, i=1,\ldots, r$, there is some element $a_{i}$ such that $a_{i}\in I_{i}$ but $a_{i}\notin P$
  ,then $a_1a_2\cdots a_{r}\notin P$(otherwise some $a_{i}\in P$ since $P$ is prime). Thus 
  $a_1a_2\cdots a_{r}\in I_{1}\cdots I_{r}$ but $a_1a_2\cdots a_{r}\notin P$, a contradiction. Thus we must 
  have some $I_{j}$ such that $I_{j}\subseteq P$.
  \item The proposition is false, consider $\mathbb{Z}$, and $P = (p)$ for some prime number $p$. Then it's 
  easy to prove $(p)$ is prime ideal: $ab\in (p)\Rightarrow p\mid ab\Rightarrow p\mid a$ or $p\mid b\Rightarrow a\in (p)$
  or $b\in (p)$.

  Consider $I_j = (p_j)$ where $p_{j}$ is the $j^{th}$ prime number(except for $p$). Then $\cap_{j=1}^{\infty}I_j=\{0\}$ satisfies the 
  condition $\cap_{j=1}^{\infty}I_{j}\subseteq P$ but none of $I_{j}$ makes $I_{j}\subseteq P$ true.
\end{enumerate}
\end{proof}

\subsection*{4.23}
A ring $R$ has Krull dimension 0 if every prime ideal in $R$ is maximal. Prove that fields and boolean rings 
(Exercise 3.15) have Krull dimension 0
\begin{proof}
  If $R$ is a ring, then $R$ has only two ideals: $(0)$ and $R$, thus the only prime chain would be 
  $(0)$ and Krull dimension is 0. (Note that $(0)$ is a prime ideal if $R$ is integral domain).
\end{proof}

\subsection*{4.24}
Prove that the ring $\mathbb{Z}[x]$ has Krull dimension $\geq 2$. 
(It is in fact exactly 2; thus it corresponds to a surface from the point of view of algebraic geometry.)
\begin{proof}
  Consider the following prime ideal chain:
  $$
  (0)\subsetneqq (x) \subsetneqq (2, x)
  $$
  First, $(0)$ is prime ideal in $\mathbb{Z}[x]$. $(x)$ is prime ideal according to exercise
  4.13; $(2, x)$ is prime since:
  $$
  \cfrac{\mathbb{Z}[x]}{(2, x)}\cong \cfrac{\mathbb{Z}}{(2)}\cong \mathbb{Z}/2\mathbb{Z}
  $$
  Second, each prime ideal is a proper subset of the following one:
  $$
  2\notin x \Rightarrow (x)\neq (2,x) 
  $$
  Thus we construct a prime ideal chain which has length 2, and the Krull dimension of $\mathbb{Z}[x]$ is greater or 
  equal to 2.
\end{proof}
\end{document}