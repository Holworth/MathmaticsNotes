\documentclass[a4paper, 12pt]{article}
\usepackage{amsthm} %lets us use \begin{proof}
\usepackage{amsmath}
\usepackage{amssymb} %gives us the character \varnothing
\usepackage[dvipsnames]{xcolor}
\usepackage{enumitem}
\newcommand{\catname}[1]{{\normalfont\textbf{#1}}}
\newlist{notes}{enumerate}{1}
\setlist[notes]{label=Note: ,leftmargin=*}
\title{Chapter2 Solutions}

\begin{document}
  \section*{Definition of Group}
  \subsection*{1.1} Write a careful proof that every group is the group of isomorphisms of a
groupoid. In particular, every group is the group of automorphisms of some object
in some category
\begin{proof}
  Let $G$ be a group, we define a category \catname{C} as follows:
  \begin{itemize}
    \item Obj(\catname{C}) = $\{*\}$
    \item Hom$(*,*)$ = $\{g \mid g \in G\}$
  \end{itemize}
  We prove the fore-defined structure does form a category:
  \begin{itemize}
    \item \textbf{Composition of Morphisms}\quad  
    There is a function as follows:\\
    $$
    \mbox{Hom}(*,*) \times \mbox{Hom}(*,*) \rightarrow \mbox{Hom}(*,*)
    $$
    $$
    (g,h)\mapsto gh
    $$
    This composition law explicitly satisfies associativity.
    
    \item \textbf{Identity} $1_G\in \mbox{Hom}(*,*)$ is the identity.
  \end{itemize}
  Also, for any $g\in \mbox{Hom}(*,*)$, there exists $g^{-1}\in \mbox{Hom}(*,*)$ such that
  $gg^{-1} = g^{-1}g = 1_G$. Thus, every morphism in $\mbox{Hom}(*,*)$ is an isomorphism and 
  $\catname{C}$ is a groupoid.
\end{proof}

\subsection*{1.4} Suppose that $g^2 = e$ for all elements $g$ of a group $G$; prove that G is commutative.
\begin{proof}
  For any $g,h\in G$, we have:
  $$
  gh = g^{-1}h^{-1} = (hg)^{-1} = hg 
  $$
  Which indicates $G$ is commutative
\end{proof}

\subsection*{1.7} Prove Corollary 1.11: \\
$$
\textit{Let $g$ be an element of finite order, and let $N\in\mathbb{Z}$. Then:}
$$
$$
\textit{$g^{N}=e\Leftrightarrow N$ is a multiple of $\mid g\mid$}
$$

\begin{proof}
  $(\Rightarrow)$ According to Lemma1.10\\
  \indent\indent $(\Leftarrow)$ $$g^N=(g^{\mid g\mid})^{\frac{N}{\mid g\mid}}=(e_G)^{\frac{N}{\mid g\mid}}=e_G$$
\end{proof}

\subsection*{1.8} Let $G$ be a finite {\color{red}\textbf{abelian}} group, with exactly one element f of order 2. Prove that $\prod_{g\in G}g=f$
\begin{proof}
  Since $G$ is abelian, the product of all elements of $G$ is well-defined, that is to say, the results is irrelevant to the multiplication order.

  Thus, we have:
  $$
  \prod_{g\in G}g = (a_1a_1^{-1})(a_2a_2^{-1})\cdots(a_na_n^{-1})fe_G=f
  $$
\end{proof}
\noindent \textbf{Note} The original problem has no abelian condition, which is a false proposition: Consider 
$Q_8=\{\pm 1, \pm i, \pm j, \pm k\}$, which is a non-commutative group and only $-1$ has an order of 2. However, the 
product of all elements in $Q_8$ may generate different results:
$$
1ijk(-1)(-i)(-j)(-k) = 1
$$
$$
1i(-i)j(-j)k(-k)(-1) = -1
$$

\subsection*{1.9} Let $G$ be a finite group, of order $n$, and let $m$ be the number of 
elements $g \in G$ of order exactly 2. Prove that $n-m$ is odd. Deduce that if $n$ is even then
$G$ necessarily contains elements of order 2.
\begin{proof}
  All elements can be make pair with its inverse, thus:
  $$
  G=\bigcup \{a_i, a_i^{-1}\}
  $$
  For those elements which have order greater than 2, $a_i$ and $a_i^{-1}$ are different. Thus we have:
  $n = m + 2k + 1$ where $k$ is the number of pair where element has order greate than 2.

  This shows that $n-m=2k+1$ is an odd value. If $n$ is even, then $m$ is certainly greater than 0, meaning there 
  are elements has order equals to 2.
\end{proof}

\subsection*{1.11} Prove that for all $g,h$ in a group $G$, $|gh| = |hg|$
\begin{proof}
  We prove that for $n\in \mathbb{N}^{+}$, $(gh)^{n}=e\Longleftrightarrow (hg)^{n}=e$
  $$
  \begin{aligned}
  (gh)^n=e  &\Longleftrightarrow (gh)(gh)\cdots(gh)=e\\
            &\Longleftrightarrow g(hg)^{n-1}h = e \\
            &\Longleftrightarrow(hg)^{n-1}h=g^{-1}\\
            &\Longleftrightarrow(hg)^n = e
  \end{aligned}
  $$
  Thus we have: $|hg| \mid |gh|$ and $|gh| \mid |hg|$, indicating $|gh| = |hg|$
\end{proof}

\subsection*{1.12}
In the group of invertible $2\times 2$ matrices, consider
$$
g = 
\begin{pmatrix}
  0 & -1\\
  1 &  0
\end{pmatrix}
\quad
,
\quad
h = 
\begin{pmatrix}
  0  & 1\\
  -1 & -1
\end{pmatrix}
$$

\noindent Verify that $|g| = 4,|h| = 3,$ and $|gh| = \infty$

\begin{proof}
  It is easy to show that $g^{2}=-I$, thus $|g| = 4$.
  For $h$ we have:
  $$
  h^2 = \begin{pmatrix}
    0 & -1\\
    1 & 0
  \end{pmatrix}\quad, \quad
  h^3 = \begin{pmatrix}
    1 & 0 \\
    0 & 1
  \end{pmatrix}
  $$
  Thus, $|h| = 3$.
  $gh = \begin{pmatrix}
    1 & 1\\
    0 & 1
  \end{pmatrix}
  $, it's not hard to verify that $(gh)^n
    = \begin{pmatrix}
      1 & n\\
      0 & 1
    \end{pmatrix}
  $(By induction), which indicates $gh$ has no finite order.
\end{proof}
\noindent \textbf{Note} If $g$ and $h$ are commutative, then $|gh|\leq lcm(|g|,|h|)$. However, for a non-commutative group, 
there is no general result for the order of $gh$.

\subsection*{1.14} prove that if $g$ and $h$ commute, and
$gcd(|g|, |h|) = 1$, then $|gh| = |g||h|$
\begin{proof}
  If $(gh)^t = e, t\in \mathbb{N}^{+}$ then: $g^t = h^{-t}$. We have:
  $$
  g^{t|h| } = h^{-t|h| } = e \Rightarrow |g| \mid t|h| \Rightarrow |g| \mid t
  $$ since $gcd(|g|, |h|)=1$. Also, $|h| \mid t$ and $|g||h| \mid t$ because $gcd(|g|, |h|)=1$.
  Note that $(gh)^{|g||h|}=e$ we have: $|gh| \mid |g||h|$. By the above fact, we have $|g||h|\mid |gh|$. Thus 
  we have: $|gh| = |g||h|$.
\end{proof}

\end{document}
