\documentclass[a4paper, pdf, 12pt]{article}
\usepackage{amsthm} %lets us use \begin{proof}
\usepackage{amsmath}
\usepackage{amssymb} %gives us the character \varnothing
\usepackage[dvipsnames]{xcolor}
\usepackage{enumitem}
\usepackage{pst-node}
\usepackage{auto-pst-pdf}
\usepackage{tikz-cd} 

\newcommand{\catname}[1]{{\normalfont\textbf{#1}}}
\newlist{notes}{enumerate}{1}
\setlist[notes]{label=Note: ,leftmargin=*}
\setlist[itemize]{leftmargin=*}

\title{Chapter2 Solutions}
\begin{document}
\section*{Definition of Group}
\subsection*{1.1} Write a careful proof that every group is the group of isomorphisms of a
groupoid. In particular, every group is the group of automorphisms of some object
in some category
\begin{proof}
  Let $G$ be a group, we define a category \catname{C} as follows:
  \begin{itemize}
    \item Obj(\catname{C}) = $\{*\}$
    \item Hom$(*,*)$ = $\{g \mid g \in G\}$
  \end{itemize}
  We prove the fore-defined structure does form a category:
  \begin{itemize}
    \item \textbf{Composition of Morphisms}\quad
          There is a function as follows:\\
          $$
            \mbox{Hom}(*,*) \times \mbox{Hom}(*,*) \rightarrow \mbox{Hom}(*,*)
          $$
          $$
            (g,h)\mapsto gh
          $$
          This composition law explicitly satisfies associativity.

    \item \textbf{Identity} $1_G\in \mbox{Hom}(*,*)$ is the identity.
  \end{itemize}
  Also, for any $g\in \mbox{Hom}(*,*)$, there exists $g^{-1}\in \mbox{Hom}(*,*)$ such that
  $gg^{-1} = g^{-1}g = 1_G$. Thus, every morphism in $\mbox{Hom}(*,*)$ is an isomorphism and
  $\catname{C}$ is a groupoid.
\end{proof}

\subsection*{1.4} Suppose that $g^2 = e$ for all elements $g$ of a group $G$; prove that G is commutative.
\begin{proof}
  For any $g,h\in G$, we have:
  $$
    gh = g^{-1}h^{-1} = (hg)^{-1} = hg
  $$
  Which indicates $G$ is commutative
\end{proof}

\subsection*{1.7} Prove Corollary 1.11: \\
$$
  \textit{Let $g$ be an element of finite order, and let $N\in\mathbb{Z}$. Then:}
$$
$$
  \textit{$g^{N}=e\Leftrightarrow N$ is a multiple of $\mid g\mid$}
$$

\begin{proof}
  $(\Rightarrow)$ According to Lemma1.10\\
  \indent\indent $(\Leftarrow)$ $$g^N=(g^{\mid g\mid})^{\frac{N}{\mid g\mid}}=(e_G)^{\frac{N}{\mid g\mid}}=e_G$$
\end{proof}

\subsection*{1.8} Let $G$ be a finite {\color{red}\textbf{abelian}} group, with exactly one element f of order 2. Prove that $\prod_{g\in G}g=f$
\begin{proof}
  Since $G$ is abelian, the product of all elements of $G$ is well-defined, that is to say, the results is irrelevant to the multiplication order.

  Thus, we have:
  $$
    \prod_{g\in G}g = (a_1a_1^{-1})(a_2a_2^{-1})\cdots(a_na_n^{-1})fe_G=f
  $$
\end{proof}
\noindent \textbf{Note} The original problem has no abelian condition, which is a false proposition: Consider
$Q_8=\{\pm 1, \pm i, \pm j, \pm k\}$, which is a non-commutative group and only $-1$ has an order of 2. However, the
product of all elements in $Q_8$ may generate different results:
$$
  1ijk(-1)(-i)(-j)(-k) = 1
$$
$$
  1i(-i)j(-j)k(-k)(-1) = -1
$$

\subsection*{1.9} Let $G$ be a finite group, of order $n$, and let $m$ be the number of
elements $g \in G$ of order exactly 2. Prove that $n-m$ is odd. Deduce that if $n$ is even then
$G$ necessarily contains elements of order 2.
\begin{proof}
  All elements can be make pair with its inverse, thus:
  $$
    G=\bigcup \{a_i, a_i^{-1}\}
  $$
  For those elements which have order greater than 2, $a_i$ and $a_i^{-1}$ are different. Thus we have:
  $n = m + 2k + 1$ where $k$ is the number of pair where element has order greate than 2.

  This shows that $n-m=2k+1$ is an odd value. If $n$ is even, then $m$ is certainly greater than 0, meaning there
  are elements has order equals to 2.
\end{proof}

\subsection*{1.11} Prove that for all $g,h$ in a group $G$, $|gh| = |hg|$
\begin{proof}
  We prove that for $n\in \mathbb{N}^{+}$, $(gh)^{n}=e\Longleftrightarrow (hg)^{n}=e$
  $$
    \begin{aligned}
      (gh)^n=e & \Longleftrightarrow (gh)(gh)\cdots(gh)=e \\
               & \Longleftrightarrow g(hg)^{n-1}h = e     \\
               & \Longleftrightarrow(hg)^{n-1}h=g^{-1}    \\
               & \Longleftrightarrow(hg)^n = e
    \end{aligned}
  $$
  Thus we have: $|hg| \mid |gh|$ and $|gh| \mid |hg|$, indicating $|gh| = |hg|$
\end{proof}

\subsection*{1.12}
In the group of invertible $2\times 2$ matrices, consider
$$
  g =
  \begin{pmatrix}
    0 & -1 \\
    1 & 0
  \end{pmatrix}
  \quad
  ,
  \quad
  h =
  \begin{pmatrix}
    0  & 1  \\
    -1 & -1
  \end{pmatrix}
$$

\noindent Verify that $|g| = 4,|h| = 3,$ and $|gh| = \infty$

\begin{proof}
  It is easy to show that $g^{2}=-I$, thus $|g| = 4$.
  For $h$ we have:
  $$
    h^2 = \begin{pmatrix}
      0 & -1 \\
      1 & 0
    \end{pmatrix}\quad, \quad
    h^3 = \begin{pmatrix}
      1 & 0 \\
      0 & 1
    \end{pmatrix}
  $$
  Thus, $|h| = 3$.
  $gh = \begin{pmatrix}
      1 & 1 \\
      0 & 1
    \end{pmatrix}
  $, it's not hard to verify that $(gh)^n
    = \begin{pmatrix}
      1 & n \\
      0 & 1
    \end{pmatrix}
  $(By induction), which indicates $gh$ has no finite order.
\end{proof}
\noindent \textbf{Note} If $g$ and $h$ are commutative, then $|gh|\leq lcm(|g|,|h|)$. However, for a non-commutative group,
there is no general result for the order of $gh$.

\subsection*{1.14} prove that if $g$ and $h$ commute, and
$gcd(|g|, |h|) = 1$, then $|gh| = |g||h|$
\begin{proof}
  If $(gh)^t = e, t\in \mathbb{N}^{+}$ then: $g^t = h^{-t}$. We have:
  $$
    g^{t|h| } = h^{-t|h| } = e \Rightarrow |g| \mid t|h| \Rightarrow |g| \mid t
  $$ since $gcd(|g|, |h|)=1$. Also, $|h| \mid t$ and $|g||h| \mid t$ because $gcd(|g|, |h|)=1$.
  Note that $(gh)^{|g||h|}=e$ we have: $|gh| \mid |g||h|$. By the above fact, we have $|g||h|\mid |gh|$. Thus
  we have: $|gh| = |g||h|$.
\end{proof}

\section*{Examples of groups}

\subsection*{2.1}
One can associate an $n \times n$ matrix $M_{\sigma}$ with a permutation $\sigma \in S_n$, by
letting the entry at $(i,\sigma(i))$ be 1, and letting all other entries be 0. For example,
the matrix corresponding to the permutation
$$
  \sigma = \begin{pmatrix}
    1 & 2 & 3 \\
    3 & 1 & 2
  \end{pmatrix} \in S_3
$$

\noindent would be
$$
  M_{\sigma} = \begin{pmatrix}
    0 & 0 & 1 \\
    1 & 0 & 0 \\
    0 & 1 & 0
  \end{pmatrix}
$$

\noindent Prove that, with this notation,
$$
  M_{\sigma\tau } = M_{\sigma}M_{\tau}
$$

\noindent for all $\sigma,\tau \in S_n$,where the product on the right is the ordinary product of matrices.

\begin{proof}
  $$
    \begin{aligned}
      M_\sigma M_\tau(i,j) & = \sum_{k = 1}^{n}M_{\sigma}(i,k)M_{\tau}(k,j) \\
                           & = \sum_{\substack{1\leq k\leq n                \\\sigma(i)=k, \tau(k)=j}}1
    \end{aligned}
  $$
  Only when $\tau\circ\sigma(i)=j$ would makes this item equals to 1, thus $M_{\sigma}M_{\tau}(i,j)=M_{\sigma\tau}(i,j)$. It's done.
\end{proof}

\subsection*{2.2}
Prove that if $d \leq n$, then $S_n$ contains elements of order $d$.
\begin{proof}
  The permutation
  $$
    \sigma = \begin{pmatrix}
      1 & 2 & 3 & \cdots & d-1 & d & d+1 & \cdots & n \\
      2 & 3 & 4 & \cdots & d   & 1 & d+1 & \cdots & n
    \end{pmatrix}
  $$
  is obviously an element has an order of $d$.
\end{proof}

\subsection*{2.6} For every positive integer $n$ construct a group containing two elements $g, h$
such that $|g| = 2, |h| = 2$, and $|gh| = n$.
\begin{proof}
  $D_{2n}$ satisfies this condition.
\end{proof}

\subsection*{2.7} Find all elements of $D_{2n}$ that commute with every other element.

\subsection*{2.12} Prove that there are no integers $a,b,c$ such that $a^2+b^2=3c^2$.
\begin{proof}
  Let $(a,b,c)$ be the smallest tuple that satisfies $a^2+b^2=3c^2$ then we have:
  $$
    a^2+b^2 = [0]_{3}
  $$
  There is only one possible way to achive this: $a = [0]_{3}, b = [0]_3$. Let $a = 3a^{'}, b = 3b^{'}$ then we have:
  $3(a'^{2} + b'^{2}) = c^{2}$, indicating $c = [0]_{3}$. Let $c = 3c'$ would incur $a'^{2} + b'^{2} = 3c'^{2}$ and we have a
  solution $(a', b', c')$ which is smaller than $(a, b, c)$, a contradiction.

\end{proof}

\subsection*{2.13} Prove that if $\mbox{gcd}(m, n)$ = 1, then there exist integers $a$ and $b$ such that
$$
  am + bn = 1
$$ Conversely, prove that if $am+ bn = 1$ for some integers $a$ and
$b$, then $\mbox{gcd}(m, n) = 1$
\begin{proof}
  $[m]_{n}$ is an generator of $\mathbb{Z}/n\mathbb{Z}$. Thus, there exists some positive integer $a$ such that:
  $a[m]_{n}=[1]_{n}$, i.e $[am]_{n} = [1]_{n}$. Further, we have: $am - 1 = b'n$ for some $b' \in \mathbb{N}$. which is:
  $am - b'n = 1$, Let $b = -b'$, the equation holds.\\

  If there are $a,b$ such that $am+bn=1$ then $\mbox{gcd}(m,n)$ is a divisor of left side, thus a divisor of 1. Then $\mbox{gcd}(m,n)$ has to be 1.
\end{proof}

\subsection*{2.15} Let $n>0$ be an odd integer.
\begin{itemize}
  \item Prove that if $\mbox{gcd}(m,n) = 1$, then $\mbox{gcd}(2m+ n, 2n) = 1$.
  \item Prove that if $\mbox{gcd}(r, 2n) = 1$, then $gcd(\frac{r+n}{2}, n) = 1$
  \item Conclude that the function $[m]_n \rightarrow [2m + n]_{2n}$ is a bijection between $(\mathbb{Z}/n\mathbb{Z})^{*}$
        and $(\mathbb{Z}/2n\mathbb{Z})^{*}$
\end{itemize}
The number $\phi(n)$ of elements of $(\mathbb{Z}/n\mathbb{Z})^{*}$ is Euler’s $\phi$-function. The reader has just
proved that if n is odd, then $\phi(2n) = \phi(n)$. Much more general formulas will be
given later on (cf. Exercise V.6.8)

\begin{proof}
  \noindent (1) Let $d = \mbox{gcd}(2m+n,2n)$ then $d \mid 2(2m+n) - 2n$, which is $d \mid 4m$. Thus:
  $d \mid \mbox{gcd}(4m, 2n)$. Note that $\mbox{gcd}(m,n)=1$, then $\mbox{gcd}(4m,2n) = 2\mbox{gcd}(2m,n)=2$.
  Thus $d=1$ or $d=2$. Note that $2m+n$ is odd, then $d = 1$.\\

  \noindent (2) Let $d = \mbox{gcd}(\frac{r+n}{2},n)$, then $d \mid 2\times \frac{r+n}{2} - n$, that is $d\mid r$. Then $d\mid n$ indicates
  $d\mid \mbox{r,n}$. Thus $d=1$.\\

  \noindent (3)
  According to (1), $\mbox{gcd}(m,n)=1$ indicates $mbox{gcd}(2m+n, 2n)=1$, thus the element $[2m+n]_{2n}\in (\mathbb{Z}/2n\mathbb{Z})^{*}$.
  Next we will verify that this function is well-defined.

  If $[m_1]_{n} = [m_2]_{n}$ then $n\mid (m_2-m_1)\Rightarrow 2n\mid (2m_2-2m_1)\Rightarrow 2n\mid ((2m_2+n)-(2m_1+n))$.
  Thus, $[2m_2+n]_{2n} = [2m_1+n]_{2n}$. This indicates the function is well-defined.

  If $[2m_1+n]_{2n} = [2m_2+n]_{2n}$ then we have $2n\mid ((2m_2+n)-(2m_1+n))$, which is
  $2n\mid 2(m_2-m_1)$, and further $n\mid (m_2-m_1)$, indicating $[m_2]_{n} = [m_1]_{n}$. Thus, this function
  is injective.

  For any $[2m+n]_{2n}\in (\mathbb{Z}/2n\mathbb{Z})^{*}$, we have $f([m]_{n})=[2m+n]_{2n}$. According to (2),
  $\mbox{gcd}(\frac{2m+n+n}{2}, n) = 1$, which is $\mbox{gcd}(m+n,n)=1\Rightarrow\mbox{gcd}(m,n)=1$. Thus, $[m]_{n}\in (\mathbb{Z}/n\mathbb{Z})^{*}$ and
  $f$ is surjective.

  In conclusion, $f$ is both injective and surjective, thus bijective.

\end{proof}

\section*{The Category $\textbf{Grp}$}
\subsection*{3.3} Show that if $G, H$ are abelian groups, then $G \times H$ satisfies the universal
property for coproducts in $\textbf{Ab}$
\begin{proof}
  Let $\tau_{G}$ and $\tau_{H}$ satisfies $\tau_{G}(g) = (g, 0_{H})$ and $\tau_{H}(h) = (0_{G}, h)$. We have to show that the following 
  commutative graph exists:
  $$
  \begin{tikzcd}[row sep=huge]
      & A & \\
  G \ar[r, "\tau_{G}"] \ar[ur, "f_{G}"] & \ar[u, dashed, "\exists!f"] G\times H & \ar[ul, "f_{H}"] \ar[l, "\tau_{H}"] H
  \end{tikzcd}
  $$
  We define $f$ as follows:
  $$
  f: G\times H\rightarrow A, \quad (g,h)\mapsto f_G(g) + f_H(h)
  $$
  We show that $f$ is an homomorphism: 
  $$
  \begin{aligned}
    f((g_1, h_1) + (g_2, h_2)) = f((g_1 + g_2, h_1 + h_2)) &= f_G(g_1+g_2) + f_H(h_1+h_2) \\
            &= f_G(g_1) + f_G(g_2) + f_H(h_1) + f_H(h_2) \\
            &= (f_G{g_1}+f_{H}(h_1)) + (f_G{g_2} + f_H(h_2)) \\
            &= f(g_1, h_1) + f(g_2, h_2)
  \end{aligned}
  $$
  And we show that $f$ is unique. if $f'$ satisfies the above commutative diagram, then we have:
  $$
  \begin{aligned}
    f'(g, h) = f'(g, 0_{H}) + f'(0_{G}, h) &= f'(\tau_{G}(g)) + f'(\tau_{H}(h)) \\
      &= (f'\tau_{G})(g) + (f'\tau_{H})(h) \\
      &= f_{G}(g) + f_{H}(h) = f(g,h)
  \end{aligned}
  $$ Thus, $f$ is unique. And by the definition of coproduct, $G\times H$ is the coproduct of $G$
  and $H$ in category $\textbf{Ab}$.

\end{proof}

\subsection*{3.4} Let $G, H$ be groups, and assume that $G \cong H \times G$. Can you conclude that $H$
is trivial.

\vspace{0.3cm}
\noindent
\textit{Solution} \quad No, $H$ might be non-trivial group. The following example:
$$
  2\mathbb{Z}\times \mathbb{Z}_2 \cong \mathbb{Z}\cong \mathbb{Z}_2
$$
indicates that $H=\mathbb{Z}_2$ is not a trivial group. We construct homomorphims as follows:
$$
  f: 2\mathbb{Z}\times \mathbb{Z}_2\longrightarrow \mathbb{Z}
$$
$$
  (\left[a\right],2k)\mapsto 2k + a, a=0,1
$$ Then it is easy to verify that $f$ is bijective.
$
  \forall x = (\left[a\right], 2k_1), y = (\left[b\right], 2k_2).
$
$$
  f(x+y) = f(\left[a+b\right],2k_1+2k_2) = 2k_1 + 2k_2 + (a+b) = f(x)+f(y)
$$ Thus, $f$ is an homomorphim, therefore, $2\mathbb{Z}\times \mathbb{Z}_2\cong \mathbb{Z}$. The
right part, $2\mathbb{Z}\cong \mathbb{Z}$ is trivial.


\subsection*{3.5}
Prove that $\mathbb{Q}$ is not the direct product of two nontrivial groups
\begin{proof}
  Proof by contradiction, say $\mathbb{Q}$ is the direct product of two groups $\mathbb{Q}\cong G\times H$, say that
  $G$ is nontrivial. We prove that $\pi_{G}$ is injective by proving no other element is mapped to be $0_{G}$ except for $0\in \mathbb{Q}$

  Suppose that $\pi_{G}\left(\cfrac{m}{n}\right)=0_{G}$. We have: $\pi_{G}(m)=n\pi_{G}(m)=nm\pi_{G}(1) = 0_G$. Thus $\pi_{G}(1)=0_{G}$.
  Which means $\pi_{G}(\mathbb{Z})=\left\{0_G\right\}$.

  Thus, for any $\cfrac{a}{b}\in \mathbb{Q}$, we have: $0_{G}=\pi_{G}(a)=b\pi_{G}(\cfrac{a}{b})\Rightarrow \pi_{G}(\cfrac{a}{b})=0_{G}$, which means
  $\pi_{G}(\mathbb{Q})=\left\{0_G\right\}$. Note that $\pi_{G}$ is surjective and $G$ is nontrivial, we have above assumption failed, that is to say,
  no element $\cfrac{a}{b}$ satisfies $\pi_{G}(\cfrac{a}{b})=0_{G}$, which means $\pi_{G}$ is injective.

  Thus $H$ must be trivial, otherwise, $\pi_{G}(g_1, h_1) = g_1 = \pi_{G}(g_1, h_2)$ indicates that $\pi_{G}$ is not injective.
\end{proof}

\subsection*{3.6}
Consider the product of the cyclic groups $C_2, C_3$: $C_2 \times C_3$. By
Exercise $3.3$, this group is a coproduct of $C_2$ and $C_3$ in $\mathbf{Ab}$. Show that it is not a
coproduct of $C_2$ and $C_3$ in $\catname{Grp}$, as follows:
\begin{itemize}
  \item find injective homomorphisms $C_2 \rightarrow S_3$, $C_3 \rightarrow S_3$;
  \item arguing by contradiction, assume that $C_2 \times C_3$ is a coproduct of $C_2, C_3$, and
        deduce that there would be a group homomorphism $C_2 \times C_3 \rightarrow S_3$ with certain
        properties;
  \item show that there is no such homomorphism
\end{itemize}

\begin{proof}
  The injective homomorphism is:
  $$
    f_{C_2}: C_2\rightarrow S_3
  $$
  $$
    \left[0\right]_{2}\mapsto \begin{pmatrix}
      1 & 2 & 3 \\
      1 & 2 & 3
    \end{pmatrix},
    \left[1\right]_{2}\mapsto \begin{pmatrix}
      1 & 2 & 3 \\
      2 & 1 & 3
    \end{pmatrix}
  $$ and
  $$
    f_{C_3}: C_3\rightarrow S_3
  $$
  $$
    \left[0\right]_{3}\mapsto \begin{pmatrix}
      1 & 2 & 3 \\
      1 & 2 & 3
    \end{pmatrix},
    \left[1\right]_{3}\mapsto \begin{pmatrix}
      1 & 2 & 3 \\
      2 & 3 & 1
    \end{pmatrix},
    \left[2\right]_{3}\mapsto \begin{pmatrix}
      1 & 2 & 3 \\
      3 & 1 & 2
    \end{pmatrix},
  $$
  According to the definition of coproduct, the following diagram holds

  $$
  \begin{tikzcd}[row sep=huge]
    & S_3 & \\
    C_2 \ar[r,"\tau_{C_2}"]  \ar[ur, "f_{C_2}"] & C_2\times C_3 \ar[u, dashed, "\exists!f"]& \ar[ul, "f_{C_3}"]\ar[l,"\tau_{C_3}",swap] C_3 
  \end{tikzcd}
  $$
  \\
  \noindent
  The homomorphism $f: C_2\times C_3\rightarrow S_3$ satisfies $f\tau_{C_2}=f_{C_2}$ and $f\tau_{C_3}=f_{C_3}$.
  We prove that such $f$ does not exist:
  We write $
  \begin{pmatrix}
    1 & 2 & 3 \\
    2 & 1 & 3
  \end{pmatrix}
  $ and $
  \begin{pmatrix}
    1 & 2 & 3 \\
    2 & 3 & 1
  \end{pmatrix}
  $ as $a$ and $b$ for simplicity: thus we must have:
  $$
  \begin{aligned}
    f([0]_{2}, [0]_{3}) = \mathbf{1}_{S_{3}}, f([1]_{2}, [0]_{3}) = a, f([0]_{2}, [1]_{3}) = b, f([0]_{2}, [1]_{3}) = b^2
  \end{aligned}
  $$
  And we have:
  $$
  ab = f([1]_{2}, [0]_{3}) + f([0]_{2}, [1]_{3}) = f([1]_{2}, [1]_{3})
  $$
  and 
  $$
  (ab)(ab) = f([1]_{2}, [1]_{3})f([1]_{2}, [1]_{3}) = f([0]_{2}, [2]_{3}) = b^2
  $$
  This indicates $abab=b^2\Rightarrow ba=a^{-1}b=ab$. However, $ab = \begin{pmatrix}
   1 & 2 & 3\\
   3 & 2 & 1 
  \end{pmatrix}
  $, $ba = \begin{pmatrix}
    1 & 2 & 3 \\
    1 & 3 & 2
  \end{pmatrix}$ thus $ab \neq ba$. Then such $f$ does not exist. We assert that $C_2\times C_3$ is not the coproduct of $C_2$ and $C_3$ in 
  category $\textbf{Grp}$.
\end{proof}
\end{document}
