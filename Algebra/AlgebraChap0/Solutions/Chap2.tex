\documentclass[a4paper, pdf, 12pt]{article}
\usepackage{amsthm} %lets us use \begin{proof}
\usepackage{amsmath}
\usepackage{amssymb} %gives us the character \varnothing
\usepackage[dvipsnames]{xcolor}
\usepackage{enumitem}
\usepackage{pst-node}
\usepackage{auto-pst-pdf}
\usepackage{tikz-cd} 
\usepackage{mathrsfs}

\newcommand{\catname}[1]{{\normalfont\textbf{#1}}}
\newcommand{\divides}{\mid}
\newcommand{\notdivides}{\nmid}

\newlist{notes}{enumerate}{1}
\setlist[notes]{label=Note: ,leftmargin=*}
\setlist[itemize]{leftmargin=*}

\title{Chapter2 Solutions}
\begin{document}
\section*{Definition of Group}
\subsection*{1.1} Write a careful proof that every group is the group of isomorphisms of a
groupoid. In particular, every group is the group of automorphisms of some object
in some category
\begin{proof}
  Let $G$ be a group, we define a category \catname{C} as follows:
  \begin{itemize}
    \item Obj(\catname{C}) = $\{*\}$
    \item Hom$(*,*)$ = $\{g \mid g \in G\}$
  \end{itemize}
  We prove the fore-defined structure does form a category:
  \begin{itemize}
    \item \textbf{Composition of Morphisms}\quad
          There is a function as follows:\\
          $$
            \mbox{Hom}(*,*) \times \mbox{Hom}(*,*) \rightarrow \mbox{Hom}(*,*)
          $$
          $$
            (g,h)\mapsto gh
          $$
          This composition law explicitly satisfies associativity.

    \item \textbf{Identity} $1_G\in \mbox{Hom}(*,*)$ is the identity.
  \end{itemize}
  Also, for any $g\in \mbox{Hom}(*,*)$, there exists $g^{-1}\in \mbox{Hom}(*,*)$ such that
  $gg^{-1} = g^{-1}g = 1_G$. Thus, every morphism in $\mbox{Hom}(*,*)$ is an isomorphism and
  $\catname{C}$ is a groupoid.
\end{proof}

\subsection*{1.4} Suppose that $g^2 = e$ for all elements $g$ of a group $G$; prove that G is commutative.
\begin{proof}
  For any $g,h\in G$, we have:
  $$
    gh = g^{-1}h^{-1} = (hg)^{-1} = hg
  $$
  Which indicates $G$ is commutative
\end{proof}

\subsection*{1.7} Prove Corollary 1.11: \\
$$
  \textit{Let $g$ be an element of finite order, and let $N\in\mathbb{Z}$. Then:}
$$
$$
  \textit{$g^{N}=e\Leftrightarrow N$ is a multiple of $\mid g\mid$}
$$

\begin{proof}
  $(\Rightarrow)$ According to Lemma1.10\\
  \indent\indent $(\Leftarrow)$ $$g^N=(g^{\mid g\mid})^{\frac{N}{\mid g\mid}}=(e_G)^{\frac{N}{\mid g\mid}}=e_G$$
\end{proof}

\subsection*{1.8} Let $G$ be a finite {\color{red}\textbf{abelian}} group, with exactly one element f of order 2. Prove that $\prod_{g\in G}g=f$
\begin{proof}
  Since $G$ is abelian, the product of all elements of $G$ is well-defined, that is to say, the results is irrelevant to the multiplication order.

  Thus, we have:
  $$
    \prod_{g\in G}g = (a_1a_1^{-1})(a_2a_2^{-1})\cdots(a_na_n^{-1})fe_G=f
  $$
\end{proof}
\noindent \textbf{Note} The original problem has no abelian condition, which is a false proposition: Consider
$Q_8=\{\pm 1, \pm i, \pm j, \pm k\}$, which is a non-commutative group and only $-1$ has an order of 2. However, the
product of all elements in $Q_8$ may generate different results:
$$
  1ijk(-1)(-i)(-j)(-k) = 1
$$
$$
  1i(-i)j(-j)k(-k)(-1) = -1
$$

\subsection*{1.9} Let $G$ be a finite group, of order $n$, and let $m$ be the number of
elements $g \in G$ of order exactly 2. Prove that $n-m$ is odd. Deduce that if $n$ is even then
$G$ necessarily contains elements of order 2.
\begin{proof}
  All elements can be make pair with its inverse, thus:
  $$
    G=\bigcup \{a_i, a_i^{-1}\}
  $$
  For those elements which have order greater than 2, $a_i$ and $a_i^{-1}$ are different. Thus we have:
  $n = m + 2k + 1$ where $k$ is the number of pair where element has order greate than 2.

  This shows that $n-m=2k+1$ is an odd value. If $n$ is even, then $m$ is certainly greater than 0, meaning there
  are elements has order equals to 2.
\end{proof}

\subsection*{1.11} Prove that for all $g,h$ in a group $G$, $|gh| = |hg|$
\begin{proof}
  We prove that for $n\in \mathbb{N}^{+}$, $(gh)^{n}=e\Longleftrightarrow (hg)^{n}=e$
  $$
    \begin{aligned}
      (gh)^n=e & \Longleftrightarrow (gh)(gh)\cdots(gh)=e \\
               & \Longleftrightarrow g(hg)^{n-1}h = e     \\
               & \Longleftrightarrow(hg)^{n-1}h=g^{-1}    \\
               & \Longleftrightarrow(hg)^n = e
    \end{aligned}
  $$
  Thus we have: $|hg| \mid |gh|$ and $|gh| \mid |hg|$, indicating $|gh| = |hg|$
\end{proof}

\subsection*{1.12}
In the group of invertible $2\times 2$ matrices, consider
$$
  g =
  \begin{pmatrix}
    0 & -1 \\
    1 & 0
  \end{pmatrix}
  \quad
  ,
  \quad
  h =
  \begin{pmatrix}
    0  & 1  \\
    -1 & -1
  \end{pmatrix}
$$

\noindent Verify that $|g| = 4,|h| = 3,$ and $|gh| = \infty$

\begin{proof}
  It is easy to show that $g^{2}=-I$, thus $|g| = 4$.
  For $h$ we have:
  $$
    h^2 = \begin{pmatrix}
      0 & -1 \\
      1 & 0
    \end{pmatrix}\quad, \quad
    h^3 = \begin{pmatrix}
      1 & 0 \\
      0 & 1
    \end{pmatrix}
  $$
  Thus, $|h| = 3$.
  $gh = \begin{pmatrix}
      1 & 1 \\
      0 & 1
    \end{pmatrix}
  $, it's not hard to verify that $(gh)^n
    = \begin{pmatrix}
      1 & n \\
      0 & 1
    \end{pmatrix}
  $(By induction), which indicates $gh$ has no finite order.
\end{proof}
\noindent \textbf{Note} If $g$ and $h$ are commutative, then $|gh|\leq lcm(|g|,|h|)$. However, for a non-commutative group,
there is no general result for the order of $gh$.

\subsection*{1.14} prove that if $g$ and $h$ commute, and
$gcd(|g|, |h|) = 1$, then $|gh| = |g||h|$
\begin{proof}
  If $(gh)^t = e, t\in \mathbb{N}^{+}$ then: $g^t = h^{-t}$. We have:
  $$
    g^{t|h| } = h^{-t|h| } = e \Rightarrow |g| \mid t|h| \Rightarrow |g| \mid t
  $$ since $gcd(|g|, |h|)=1$. Also, $|h| \mid t$ and $|g||h| \mid t$ because $gcd(|g|, |h|)=1$.
  Note that $(gh)^{|g||h|}=e$ we have: $|gh| \mid |g||h|$. By the above fact, we have $|g||h|\mid |gh|$. Thus
  we have: $|gh| = |g||h|$.
\end{proof}

\section*{Examples of groups}

\subsection*{2.1}
One can associate an $n \times n$ matrix $M_{\sigma}$ with a permutation $\sigma \in S_n$, by
letting the entry at $(i,\sigma(i))$ be 1, and letting all other entries be 0. For example,
the matrix corresponding to the permutation
$$
  \sigma = \begin{pmatrix}
    1 & 2 & 3 \\
    3 & 1 & 2
  \end{pmatrix} \in S_3
$$

\noindent would be
$$
  M_{\sigma} = \begin{pmatrix}
    0 & 0 & 1 \\
    1 & 0 & 0 \\
    0 & 1 & 0
  \end{pmatrix}
$$

\noindent Prove that, with this notation,
$$
  M_{\sigma\tau } = M_{\sigma}M_{\tau}
$$

\noindent for all $\sigma,\tau \in S_n$,where the product on the right is the ordinary product of matrices.

\begin{proof}
  $$
    \begin{aligned}
      M_\sigma M_\tau(i,j) & = \sum_{k = 1}^{n}M_{\sigma}(i,k)M_{\tau}(k,j) \\
                           & = \sum_{\substack{1\leq k\leq n                \\\sigma(i)=k, \tau(k)=j}}1
    \end{aligned}
  $$
  Only when $\tau\circ\sigma(i)=j$ would makes this item equals to 1, thus $M_{\sigma}M_{\tau}(i,j)=M_{\sigma\tau}(i,j)$. It's done.
\end{proof}

\subsection*{2.2}
Prove that if $d \leq n$, then $S_n$ contains elements of order $d$.
\begin{proof}
  The permutation
  $$
    \sigma = \begin{pmatrix}
      1 & 2 & 3 & \cdots & d-1 & d & d+1 & \cdots & n \\
      2 & 3 & 4 & \cdots & d   & 1 & d+1 & \cdots & n
    \end{pmatrix}
  $$
  is obviously an element has an order of $d$.
\end{proof}

\subsection*{2.6} For every positive integer $n$ construct a group containing two elements $g, h$
such that $|g| = 2, |h| = 2$, and $|gh| = n$.
\begin{proof}
  $D_{2n}$ satisfies this condition.
\end{proof}

\subsection*{2.7} Find all elements of $D_{2n}$ that commute with every other element.

\subsection*{2.12} Prove that there are no integers $a,b,c$ such that $a^2+b^2=3c^2$.
\begin{proof}
  Let $(a,b,c)$ be the smallest tuple that satisfies $a^2+b^2=3c^2$ then we have:
  $$
    a^2+b^2 = [0]_{3}
  $$
  There is only one possible way to achive this: $a = [0]_{3}, b = [0]_3$. Let $a = 3a^{'}, b = 3b^{'}$ then we have:
  $3(a'^{2} + b'^{2}) = c^{2}$, indicating $c = [0]_{3}$. Let $c = 3c'$ would incur $a'^{2} + b'^{2} = 3c'^{2}$ and we have a
  solution $(a', b', c')$ which is smaller than $(a, b, c)$, a contradiction.

\end{proof}

\subsection*{2.13} Prove that if $\mbox{gcd}(m, n)$ = 1, then there exist integers $a$ and $b$ such that
$$
  am + bn = 1
$$ Conversely, prove that if $am+ bn = 1$ for some integers $a$ and
$b$, then $\mbox{gcd}(m, n) = 1$
\begin{proof}
  $[m]_{n}$ is an generator of $\mathbb{Z}/n\mathbb{Z}$. Thus, there exists some positive integer $a$ such that:
  $a[m]_{n}=[1]_{n}$, i.e $[am]_{n} = [1]_{n}$. Further, we have: $am - 1 = b'n$ for some $b' \in \mathbb{N}$. which is:
  $am - b'n = 1$, Let $b = -b'$, the equation holds.\\

  If there are $a,b$ such that $am+bn=1$ then $\mbox{gcd}(m,n)$ is a divisor of left side, thus a divisor of 1. Then $\mbox{gcd}(m,n)$ has to be 1.
\end{proof}

\subsection*{2.15} Let $n>0$ be an odd integer.
\begin{itemize}
  \item Prove that if $\mbox{gcd}(m,n) = 1$, then $\mbox{gcd}(2m+ n, 2n) = 1$.
  \item Prove that if $\mbox{gcd}(r, 2n) = 1$, then $gcd(\frac{r+n}{2}, n) = 1$
  \item Conclude that the function $[m]_n \rightarrow [2m + n]_{2n}$ is a bijection between $(\mathbb{Z}/n\mathbb{Z})^{*}$
        and $(\mathbb{Z}/2n\mathbb{Z})^{*}$
\end{itemize}
The number $\phi(n)$ of elements of $(\mathbb{Z}/n\mathbb{Z})^{*}$ is Euler’s $\phi$-function. The reader has just
proved that if n is odd, then $\phi(2n) = \phi(n)$. Much more general formulas will be
given later on (cf. Exercise V.6.8)

\begin{proof}
  \noindent (1) Let $d = \mbox{gcd}(2m+n,2n)$ then $d \mid 2(2m+n) - 2n$, which is $d \mid 4m$. Thus:
  $d \mid \mbox{gcd}(4m, 2n)$. Note that $\mbox{gcd}(m,n)=1$, then $\mbox{gcd}(4m,2n) = 2\mbox{gcd}(2m,n)=2$.
  Thus $d=1$ or $d=2$. Note that $2m+n$ is odd, then $d = 1$.\\

  \noindent (2) Let $d = \mbox{gcd}(\frac{r+n}{2},n)$, then $d \mid 2\times \frac{r+n}{2} - n$, that is $d\mid r$. Then $d\mid n$ indicates
  $d\mid \mbox{r,n}$. Thus $d=1$.\\

  \noindent (3)
  According to (1), $\mbox{gcd}(m,n)=1$ indicates $mbox{gcd}(2m+n, 2n)=1$, thus the element $[2m+n]_{2n}\in (\mathbb{Z}/2n\mathbb{Z})^{*}$.
  Next we will verify that this function is well-defined.

  If $[m_1]_{n} = [m_2]_{n}$ then $n\mid (m_2-m_1)\Rightarrow 2n\mid (2m_2-2m_1)\Rightarrow 2n\mid ((2m_2+n)-(2m_1+n))$.
  Thus, $[2m_2+n]_{2n} = [2m_1+n]_{2n}$. This indicates the function is well-defined.

  If $[2m_1+n]_{2n} = [2m_2+n]_{2n}$ then we have $2n\mid ((2m_2+n)-(2m_1+n))$, which is
  $2n\mid 2(m_2-m_1)$, and further $n\mid (m_2-m_1)$, indicating $[m_2]_{n} = [m_1]_{n}$. Thus, this function
  is injective.

  For any $[2m+n]_{2n}\in (\mathbb{Z}/2n\mathbb{Z})^{*}$, we have $f([m]_{n})=[2m+n]_{2n}$. According to (2),
  $\mbox{gcd}(\frac{2m+n+n}{2}, n) = 1$, which is $\mbox{gcd}(m+n,n)=1\Rightarrow\mbox{gcd}(m,n)=1$. Thus, $[m]_{n}\in (\mathbb{Z}/n\mathbb{Z})^{*}$ and
  $f$ is surjective.

  In conclusion, $f$ is both injective and surjective, thus bijective.

\end{proof}

\section*{The Category $\textbf{Grp}$}
\subsection*{3.3} Show that if $G, H$ are abelian groups, then $G \times H$ satisfies the universal
property for coproducts in $\textbf{Ab}$
\begin{proof}
  Let $\tau_{G}$ and $\tau_{H}$ satisfies $\tau_{G}(g) = (g, 0_{H})$ and $\tau_{H}(h) = (0_{G}, h)$. We have to show that the following
  commutative graph exists:
  $$
    \begin{tikzcd}[row sep=huge]
      & A & \\
      G \ar[r, "\tau_{G}"] \ar[ur, "f_{G}"] & \ar[u, dashed, "\exists!f"] G\times H & \ar[ul, "f_{H}"] \ar[l, "\tau_{H}"] H
    \end{tikzcd}
  $$
  We define $f$ as follows:
  $$
    f: G\times H\rightarrow A, \quad (g,h)\mapsto f_G(g) + f_H(h)
  $$
  We show that $f$ is an homomorphism:
  $$
    \begin{aligned}
      f((g_1, h_1) + (g_2, h_2)) = f((g_1 + g_2, h_1 + h_2)) & = f_G(g_1+g_2) + f_H(h_1+h_2)                   \\
                                                             & = f_G(g_1) + f_G(g_2) + f_H(h_1) + f_H(h_2)     \\
                                                             & = (f_G{g_1}+f_{H}(h_1)) + (f_G{g_2} + f_H(h_2)) \\
                                                             & = f(g_1, h_1) + f(g_2, h_2)
    \end{aligned}
  $$
  And we show that $f$ is unique. if $f'$ satisfies the above commutative diagram, then we have:
  $$
    \begin{aligned}
      f'(g, h) = f'(g, 0_{H}) + f'(0_{G}, h) & = f'(\tau_{G}(g)) + f'(\tau_{H}(h)) \\
                                             & = (f'\tau_{G})(g) + (f'\tau_{H})(h) \\
                                             & = f_{G}(g) + f_{H}(h) = f(g,h)
    \end{aligned}
  $$ Thus, $f$ is unique. And by the definition of coproduct, $G\times H$ is the coproduct of $G$
  and $H$ in category $\textbf{Ab}$.

\end{proof}

\subsection*{3.4} Let $G, H$ be groups, and assume that $G \cong H \times G$. Can you conclude that $H$
is trivial.

\vspace{0.3cm}
\noindent
\textit{Solution} \quad No, $H$ might be non-trivial group. The following example:
$$
  2\mathbb{Z}\times \mathbb{Z}_2 \cong \mathbb{Z}\cong \mathbb{Z}_2
$$
indicates that $H=\mathbb{Z}_2$ is not a trivial group. We construct homomorphims as follows:
$$
  f: 2\mathbb{Z}\times \mathbb{Z}_2\longrightarrow \mathbb{Z}
$$
$$
  (\left[a\right],2k)\mapsto 2k + a, a=0,1
$$ Then it is easy to verify that $f$ is bijective.
$
  \forall x = (\left[a\right], 2k_1), y = (\left[b\right], 2k_2).
$
$$
  f(x+y) = f(\left[a+b\right],2k_1+2k_2) = 2k_1 + 2k_2 + (a+b) = f(x)+f(y)
$$ Thus, $f$ is an homomorphim, therefore, $2\mathbb{Z}\times \mathbb{Z}_2\cong \mathbb{Z}$. The
right part, $2\mathbb{Z}\cong \mathbb{Z}$ is trivial.


\subsection*{3.5}
Prove that $\mathbb{Q}$ is not the direct product of two nontrivial groups
\begin{proof}
  Proof by contradiction, say $\mathbb{Q}$ is the direct product of two groups $\mathbb{Q}\cong G\times H$, say that
  $G$ is nontrivial. We prove that $\pi_{G}$ is injective by proving no other element is mapped to be $0_{G}$ except for $0\in \mathbb{Q}$

  Suppose that $\pi_{G}\left(\cfrac{m}{n}\right)=0_{G}$. We have: $\pi_{G}(m)=n\pi_{G}(m)=nm\pi_{G}(1) = 0_G$. Thus $\pi_{G}(1)=0_{G}$.
  Which means $\pi_{G}(\mathbb{Z})=\left\{0_G\right\}$.

  Thus, for any $\cfrac{a}{b}\in \mathbb{Q}$, we have: $0_{G}=\pi_{G}(a)=b\pi_{G}(\cfrac{a}{b})\Rightarrow \pi_{G}(\cfrac{a}{b})=0_{G}$, which means
  $\pi_{G}(\mathbb{Q})=\left\{0_G\right\}$. Note that $\pi_{G}$ is surjective and $G$ is nontrivial, we have above assumption failed, that is to say,
  no element $\cfrac{a}{b}$ satisfies $\pi_{G}(\cfrac{a}{b})=0_{G}$, which means $\pi_{G}$ is injective.

  Thus $H$ must be trivial, otherwise, $\pi_{G}(g_1, h_1) = g_1 = \pi_{G}(g_1, h_2)$ indicates that $\pi_{G}$ is not injective.
\end{proof}

\subsection*{3.6}
Consider the product of the cyclic groups $C_2, C_3$: $C_2 \times C_3$. By
Exercise $3.3$, this group is a coproduct of $C_2$ and $C_3$ in $\mathbf{Ab}$. Show that it is not a
coproduct of $C_2$ and $C_3$ in $\catname{Grp}$, as follows:
\begin{itemize}
  \item find injective homomorphisms $C_2 \rightarrow S_3$, $C_3 \rightarrow S_3$;
  \item arguing by contradiction, assume that $C_2 \times C_3$ is a coproduct of $C_2, C_3$, and
        deduce that there would be a group homomorphism $C_2 \times C_3 \rightarrow S_3$ with certain
        properties;
  \item show that there is no such homomorphism
\end{itemize}

\begin{proof}
  The injective homomorphism is:
  $$
    f_{C_2}: C_2\rightarrow S_3
  $$
  $$
    \left[0\right]_{2}\mapsto \begin{pmatrix}
      1 & 2 & 3 \\
      1 & 2 & 3
    \end{pmatrix},
    \left[1\right]_{2}\mapsto \begin{pmatrix}
      1 & 2 & 3 \\
      2 & 1 & 3
    \end{pmatrix}
  $$ and
  $$
    f_{C_3}: C_3\rightarrow S_3
  $$
  $$
    \left[0\right]_{3}\mapsto \begin{pmatrix}
      1 & 2 & 3 \\
      1 & 2 & 3
    \end{pmatrix},
    \left[1\right]_{3}\mapsto \begin{pmatrix}
      1 & 2 & 3 \\
      2 & 3 & 1
    \end{pmatrix},
    \left[2\right]_{3}\mapsto \begin{pmatrix}
      1 & 2 & 3 \\
      3 & 1 & 2
    \end{pmatrix},
  $$
  According to the definition of coproduct, the following diagram holds

  $$
    \begin{tikzcd}[row sep=huge]
      & S_3 & \\
      C_2 \ar[r,"\tau_{C_2}"]  \ar[ur, "f_{C_2}"] & C_2\times C_3 \ar[u, dashed, "\exists!f"]& \ar[ul, "f_{C_3}"]\ar[l,"\tau_{C_3}",swap] C_3
    \end{tikzcd}
  $$
  \\
  \noindent
  The homomorphism $f: C_2\times C_3\rightarrow S_3$ satisfies $f\tau_{C_2}=f_{C_2}$ and $f\tau_{C_3}=f_{C_3}$.
  We prove that such $f$ does not exist:
  We write $
    \begin{pmatrix}
      1 & 2 & 3 \\
      2 & 1 & 3
    \end{pmatrix}
  $ and $
    \begin{pmatrix}
      1 & 2 & 3 \\
      2 & 3 & 1
    \end{pmatrix}
  $ as $a$ and $b$ for simplicity: thus we must have:
  $$
    \begin{aligned}
      f([0]_{2}, [0]_{3}) = \mathbf{1}_{S_{3}}, f([1]_{2}, [0]_{3}) = a, f([0]_{2}, [1]_{3}) = b, f([0]_{2}, [1]_{3}) = b^2
    \end{aligned}
  $$
  And we have:
  $$
    ab = f([1]_{2}, [0]_{3}) + f([0]_{2}, [1]_{3}) = f([1]_{2}, [1]_{3})
  $$
  and
  $$
    (ab)(ab) = f([1]_{2}, [1]_{3})f([1]_{2}, [1]_{3}) = f([0]_{2}, [2]_{3}) = b^2
  $$
  This indicates $abab=b^2\Rightarrow ba=a^{-1}b=ab$. However, $ab = \begin{pmatrix}
      1 & 2 & 3 \\
      3 & 2 & 1
    \end{pmatrix}
  $, $ba = \begin{pmatrix}
      1 & 2 & 3 \\
      1 & 3 & 2
    \end{pmatrix}$ thus $ab \neq ba$. Then such $f$ does not exist. We assert that $C_2\times C_3$ is not the coproduct of $C_2$ and $C_3$ in
  category $\textbf{Grp}$.
\end{proof}

\section*{Group Homomorphisms}
\subsection*{4.1}
Check that the function $\pi_{m}^{n}$
defined in 4.1 is well-defined, and makes the
diagram commute.Verify that it is a group homomorphism.Why is the hypothesis
$m \divides n$ necessary?
\begin{proof}
  $\pi_{m}^{n}$ is well-defined: if $[a_1]_{n}=[a_2]_{n}$ then $n\divides a_1-a_2$, thus
  $m\divides a_1-a_2$ as $m\divides n$. We have $[a_1]_{m}=[a_2]_{m}$ and
  $\pi_{m}^{n}([a_1]_{n}) = \pi_{m}^{n}([a_2]_{n})$. The function has nothing to do with the
  representators.
  This is a homomorphism becuase:
  $$
    \pi_{m}^{n}([a]_{n}+[b]_{n}) = \pi_{m}^{n}([a+b]_{n}) = [a+b]_{m} = [a]_{m} + [b]_{m} = \pi_{m}^{n}([a]_{n}) + \pi_{m}^{n}([b]_{n})
  $$

  The hypothesis $m\divides n$ is necessary because if $m\notdivides n$ we may fail to show
  that $pi_{m}^{n}$ is well-defined.
  One example is to use $m = 4, n = 3$. Then $\pi_{m}^{n}$ is not well-defined, we have:
  $$
    \pi_{3}^{4}([12]_{4}) = [12]_{3} = [0]_{3};
  $$
  $$
    \pi_{3}^{4}([8]_{4}) = [8]_{3} = [2]_{3} \neq [0]_{3}
  $$
\end{proof}

\subsection*{4.2}
Show that the homomorphism $\pi_{2}^{4}\times \pi_{2}^{4}$
: $C_4\rightarrow C_2\times C_2$ is not an isomorphism.
In fact, is there any nontrivial isomorphism $C_4\rightarrow C_2\times C_2$?\\

\noindent
\textit{Solution}\quad No, there is no such isomorphism between $C_4$ and $C_2\times C_2$.
The reason is that $C_4$ has one element of order 4, which is $[1]_{4}$, however, each element of $C_2\times C_2$ has order 1 or 2.

\subsection*{4.3}
Prove that a group of order $n$ is isomorphic to $\mathbb{Z}/n \mathbb{Z}$ if and only if it contains
an element of order $n$.
\begin{proof}
  $(\Rightarrow)$ If group $G$ with order of $n$ is isomorphic to $\mathbb{Z}/n\mathbb{Z}$ then $G$ must have
  an element of order $n$, which is $f^{-1}([1]_{n})$. Here $f$ is the isomorphism from $G$ to $\mathbb{Z}/n\mathbb{Z}$.
  \\
  $(\Leftarrow)$ If group $G$ with order $n$ has an element with order of $n$,say $g$ Then $\langle g\rangle = G$. We define the
  homomorphism $f$: $G\rightarrow \mathbb{Z}/n\mathbb{Z}$ as follows: $g^{k}\mapsto [k]_{n}$.

  \noindent
  It is obvious to see that $f$ is an isomorphism.
\end{proof}

\subsection*{4.4}
Prove that no two of the groups $(\mathbb{Z}, +), (\mathbb{Q}, +), (\mathbb{R},+)$ are isomorphic to one
another. Can you decide whether $(\mathbb{R}, +), (\mathbb{C}, +)$ are isomorphic to one another.
\begin{proof}
  $(\mathbb{Z, +})$ and $(\mathbb{Q, +})$ are not isomorphic to $(\mathbb{R}, +)$ because they even
  do not have the same cardinality. \\
  \\
  \noindent
  $(\mathbb{Z}, +)\ncong (\mathbb{Q}, +)$: \\
  Suppose $f$ is an isomorphism from $(\mathbb{Z}, +)$ to $(\mathbb{Q}, +)$, let $f(1) = g\in \mathbb{Q}$. Then we have $\mathbb{Q}$ is generated by
  $g$ as $\cfrac{a}{b}=f(n) = nf(1)=ng$ for some $n$. Let $g = \cfrac{a}{b}$ and $a, b$ relatively prime, then have:
  $\cfrac{na}{b}=\cfrac{1}{p}$. We have: $pna=b$. note that $\mbox{gcd}(a,b)=1$, then we must have $a=1$. And $np=b$. We pick $p$ a prime that
  is relatively prime to $b$. Then $np=b$ can not be true.
\end{proof}

\subsection*{4.5}
Prove that the groups $(\mathbb{R}\setminus\{0\}, \times)$ and $(\mathbb{C}\setminus\{0\}, \times)$ are not isomorphic.

\begin{proof}
  If $(\mathbb{R}\setminus\{0\},\times)$ is isomorphic to $(\mathbb{C}\setminus\{0\},\times)$ let the isomorphism be $f$.
  and let $f(1) = 1$ and let $f(\mathbf{i})=g$ Consider $f(-1)$, we have:
  $$
    f(-1)^{2} = f((-1)^{2}) = f(1) = 1
  $$ Then we have $f(-1) = 1$ or $f(-1) = -1$, note that $f$ is an isomorphism, we must have $f(-1) = -1$.
  Further we have:$f(\mathbf{i})^{2} = f(\mathbf{i}^{2}) = f(-1) = -1$. However, no such element in $\mathbb{R}$ makes this true.
  Thus, we have show that $(\mathbb{R}\setminus\{0\},\times)\ncong (\mathbb{C}\setminus\{0\},\times)$.
\end{proof}

\subsection*{4.6}
We have seen that $(\mathbb{R}, +)$ and $(\mathbb{R}^{>0}, \times)$ are isomorphic (Example 4.4). Are the
groups $(\mathbb{Q}, +)$ and $(\mathbb{Q}^{>0}, \times)$ isomorphic?\\

\noindent
\textit{Solution}\quad

\subsection*{4.7}
Let $G$ be a group. Prove that the function $G \rightarrow G$ defined by $g \mapsto g^{-1}$ is a
homomorphism if and only if $G$ is abelian. Prove that $g \mapsto g^2$ is a homomorphism
if and only if $G$ is abelian.
\begin{proof}
  $g\mapsto g^{-1}$ is an homomorphism iff $f(ab)=f(a)f(b)$ holds for any $a,b\in G$.
  This is true if and only if $a^{-1}b^{-1}=b^{-1}a^{-1}$ for any $a,b\in G$. And
  $a^{-1}b^{-1}=b^{-1}a^{-1}\Longleftrightarrow ba=ab$ by taking inverse at both sides. Thus we
  have $g\mapsto g^{-1}$ if and only if $G$ is abelian.

  $g\mapsto g^{2}$ is an homomorphism iff $f(ab) = f(a)f(b)$ holds for any $a,b\in G$.
  This is true if and only if $(ab)(ab) = a^2b^2\Longleftrightarrow ab=ba$ for any $a,b\in G$.
\end{proof}

\subsection*{4.8}
Let $G$ be a group, and $g \in G$. Prove that the function
$\gamma_{g} : G \rightarrow G$ defined
by ($\forall a\in G$) : $\gamma_{g}(a) = gag^{-1}$ is an automorphism of G.
(The automorphisms $\gamma_{g}$ are
called ‘inner’ automorphisms of G.) Prove that the function $G \rightarrow \mbox{Aut}(G)$ defined
by $g \rightarrow \gamma_{g}$ is a homomorphism. Prove that this homomorphism is trivial if and
only if G is abelian.

\begin{proof}
  First we show that $\gamma_{g}$ is an homomorphism: for any $a,b\in G$ we have:
  $$
    \gamma_{g}(ab) = g(ab)g^{-1} = (gag^{-1})(gbg^{-1}) = \gamma_{g}(a)\gamma_{g}(b)
  $$
  Thus $\gamma_{g}$ is an homomorphism. $\gamma_{g}$ has an inverse: $\gamma_{g^{-1}}$.
  We have:$\gamma_{g}\gamma_{g^{-1}}(a) = \gamma_{g}(g^{-1}ag)=g(g^{-1}ag)g^{-1} = a$ for
  any $a\in G$. Thus, $\gamma_{g}\gamma_{g^{-1}}=I_{G}$. Similarly, $\gamma_{g^{-1}}\gamma_{g}=I_{G}$.
  Thus $\gamma_{g}$ has inverse and therefore a bijection, this indicates $\gamma_{g}$ is an isomorphism.\\

  Let $f: G\rightarrow \mbox{Aut}(G), g\rightarrow \gamma_{g}$ be the function mentioned above. We shall prove that this
  function is actually an homomorphism:
  $f(ab) = \gamma_{ab}$ and we have:$\gamma_{ab}(g) = (ab)^{-1}gab = b^{-1}(a^{-1}ga)b = \gamma_{a}\circ\gamma_{b}(g)$ for all
  $g\in G$. Thus we have $f(ab) = \gamma_{ab}=\gamma_{a}\circ\gamma_{b}=f(a)f(b)$. Therefore $f$ is an homomorphism.
  If $G$ is abelian then all $f(g) = \gamma_{g}=I_{G}$, thus is trivial.
\end{proof}

\subsection*{4.9}
Prove that if $m, n$ are positive integers such that $\mbox{gcd}(m, n) = 1$, then
$C_{mn}\cong C_{m}\times C_{n}$.

\begin{proof}
  The homomorphism $\pi_{m}^{mn}\times \pi_{n}^{mn}:C_{mn}\rightarrow C_{m}\times C_{n}$ is defined as follows:
  $$
    [a]_{mn}\mapsto ([a]_{m}, [a]_{n})
  $$ and is an homomorphism as $\pi_{m}^{mn}$ and $\pi_{n}^{mn}$ are homomorphisms.
  We shall show that this function is bijection. First it is injective:
  if $f([a]_{mn}) = f([b]_{mn})$ then $([a]_{m}, [a]_{n}) = ([b]_{m}, [b]_{n})$ which means: $m\divides a - b$ and
  $n\divides a - b$. Further we have $mn \divides a - b$ because $\mbox{gcd}(m,n) = 1$. Thus $[a]_{mn} = [b]_{mn}$ and
  this indicates $f$ is injective.

  For surjective property, note that $\mbox{gcd}(m,n)=1$ indicates there exist some $x, y$ such that $xm - ny = 1$.
  Then we have $x$ satisfies $xm = ny + 1$, we call $\mathbf{x}=[xm]_{mn}$, we have $f(\mathbf{x}) = ([0]_{m}, [1]_{n})$.
  Similarly, we will have such $\mathbf{y}$ satisfying $f(\mathbf{y}) = ([1]_{m}, [0]_{n})$.
  For any $([a]_{m}, [b]_{n})\in C_{m}\times C_{n}$ we have: $([a]_{m}, [b]_{n}) = ([a]_{m}, [0]_{n}) + ([0]_{m}, [b]_{n}) = af(\mathbf{x}) + bf(\mathbf{y})=f(a\mathbf{x}+b\mathbf{y})$.
  Thus $f$ is surjective and $f$ is bijective.

  In conclusion, we have $f$ to be group homomorphism and bijection. Thus $f$ is a group isomorphism.
\end{proof}

\subsection*{4.10}
Let $p \neq q$ be odd prime integers; show that $(\mathbb{Z}/pq\mathbb{Z})^{*}$ is not cyclic.

\begin{proof}
  Suppose that $(\mathbb{Z}/pq\mathbb{Z})^{*}$ is cyclic. Then we have the order of
\end{proof}

\subsection*{4.11}
In due time we will prove the easy fact that if $p$ is a prime integer then
the equation $x^d = 1$ can have at most $d$ solutions in $\mathbb{Z}/p\mathbb{Z}$. Assume this fact, and
prove that the multiplicative group $G = (\mathbb{Z}/p\mathbb{Z})^{*}$ is cyclic

\begin{proof}
  Let the maximum order of elements in $(\mathbb{Z}/p\mathbb{Z})^{*}$ be $d$, we show that $d$ must be $p$.

  If $d\leq p-2$, say $g$ has order $d$, then for every element $h\in (\mathbb{Z}/p\mathbb{Z})^{*}$ we have $|h|\divides d$. Otherwise,
  the element $gh$ will have order of $\mbox{lcm}(|h|, d) > d$, contradicts the assumption that $d$ is the maximum order.

  Thus we have $g^{d} = 1$ for every element in $\mathbb{Z}/p\mathbb{Z}$, which means $x^{d} = 1$ has $p-1$ solutions, controdicts the
  assumption. Thus, we have $d = p-1$ and $\mathbb{Z}/p\mathbb{Z}$ is cyclic.
\end{proof}

\noindent
\textbf{NOTE} This proof can not be used to proof a general $(\mathbb{Z}/n\mathbb{Z})^{*}, n\in \mathbb{N}^{+}$  is cyclic(though this
proposition is false). The assumption $x^{d}=1$ has at most $d$ solutions is constrainted within $\mathbb{Z}/p\mathbb{Z}$, not generalized
group.

\subsection*{4.14}
Prove that the order of the group of automorphisms of a cyclic group $C_n$ is
the number of positive integers $r < n$ that are \textit{relatively prime to} $n$.
\begin{proof}
  $C_n$ is generated by $[1]_{n}$, so any automorphism from $C_n$ to $C_n$ is determined by the image of
  $[1]_{n}$. To make this homomorphim $f$ bijective, we must make $f([1]_{n})$ also be a generator. Thus the
  number of elements in $\mbox{Aut}_{\mathbf{Grp}}(C_n)$ is determined by the number of generators in $C_{n}$, which is
  the number of positive number that is relatively prime to $n$. We formally prove this as followed:

  Let $f\in \mbox{Aut}_{\mathbf{Grp}}(C_n)$, consider $f([1]_{n})$. Notice that $f$ is isomorphism, thus we have $|f([1]_{n})|$ has order
  $n$(proposition4.8), thus $|f([1]_{n})|$ is relatively prime to $n$(The representator of $f([1]_{n})$).

  On the contrary, if $[m]_{n}, \mbox{gcd}(m, n)=1$, we define $f([1]_{n})=[m]_{n}$, it derives an isomorphism from $C_{n}$ to $C_{n}$. Thus, we have
  established a map from $\mbox{Aut}_{\mathbf{Grp}}(C_n)$ to the set of numbers that are relatively prime to $n$, denoted as $S$. This map is injective as each $f$ maps
  $[1]_{n}$ to different elements in $S$, and is surjective by the construction described above. Thus, it is bijection and they have the
  same cardinality.
\end{proof}

\subsection*{4.15}
Compute the group of automorphisms of $(\mathbb{Z}, +)$. Prove that if $p$ is prime,
then $\mbox{Aut}_{\mathbf{Grp}}(C_p) \cong C_{p-1}$. (Use Exercise 4.11.)

\begin{proof}
  There are only two elements in $\mbox{Aut}_{\mathbf{Grp}}(\mathbb{Z}, +)$: The identity and the isomorphism that maps $1$ to $-1$.

  \noindent To prove $\mbox{Aut}_{\mathbf{Grp}}(C_{p})\cong C_{p-1}$, we show that
  $\mbox{Aut}_{\mathbf{Grp}}(C_{p})\cong (\mathbb{Z}/p\mathbb{Z})^{*}$ and leverage the result of
  exercise 4.11.

  The proof of exercise 4.14 shows that there is a bijection from $\mbox{Aut}_{\mathbf{Grp}}(C_p)$ to $(\mathbb{Z}/p\mathbb{Z})^{*}$ by $[m]_{n}\mapsto f_{[m]_{n}}, \mbox{gcd}(m,n)=1$, where
  $f_{[m]_{n}}$ is the automorphism derived by $f_{[m]_{n}}([1]_{n}) = [m]_{n}$. We show that this map, namely $\phi$ is an homomorphim:
  $$
    \phi([m_1]_{n}\times[m_2]_{n})=\phi([m_1m_2]_{n}) = f_{[m_1m_2]_{n}}=f_{[m_1]_{n}}\circ f_{[m_2]_{n}}
  $$
  The last $=$ is true by checking the image of $[1]_{n}$ under $f_{[m_1m_2]_{n}}$ and $f_{[m_1]_{n}}\circ f_{[m_2]_{n}}$
  In conclusion, we have the map $\phi$ is both a homomorphim and bijection. Thus, $\mbox{Aut}_{\mathbf{Grp}}(C_p)\cong (\mathbb{Z}/p\mathbb{Z})^{*}\cong C_{p-1}$.
\end{proof}

\subsection*{4.16}
Prove \textit{Wilson's theorem}: a positive integer $p$ is prime if and only if
$$(p-1)! \equiv -1 \mod{p} $$

\begin{proof}
  $(\Rightarrow)$ If $p$ is a prime, then $(\mathbb{Z}/p\mathbb{Z})^{*}$ is cyclic, let $g\in \mathbb{Z}/p\mathbb{Z})^{*}$ be the elements with order $p-1$, then we have:
  $$
    (p-1)! \equiv gg^{2}\dots g^{p-1}\equiv g^{\frac{p(p-1)}{2}} \mod{p}
  $$
  Note that we have $g^{p-1}\equiv 1\mod{p}$ and $g^{\frac{p-1}{2}}\equiv -1\mod{n}$ because the order of $g$ is exactly $p-1$. We have:
  $$
    g^{\frac{p(p-1)}{2}} = g^{\frac{(p-1)^{2}}{2}}g^{\frac{p-1}{2}}\equiv g^{\frac{p-1}{2}}\equiv -1\mod{p}
  $$
  The proof is done.\\

  \noindent
  $(\Leftarrow)$ Suppose $p$ is not a prime and $d$ is a divisor of $p$. Then we have:$(p-1)!\equiv -1\mod{d}$. However,
  $d < p$ indicates $d \divides d!$ and $d! \divides (p-1)!$, thus we have:$(p-1)!\equiv 0\mod{d}$, a contradiction.
\end{proof}

\section*{5. Free Group}
\subsection*{5.1}
Does the category $\mathscr{F}^{A}$ defined in 5.2 have final objects? If so, what are they.\\

\noindent
\textit{Solution} \quad It has, the object $(G, j)$ where $G$ is trivial group and $j$ is a set-function satisfies: $a\mapsto 1_{G}, \forall a\in A$ is a final object in
$\mathscr{F}^{A}$. It's obvious that any other object in this category has a morphism to this object, namely the trivial homomorphim. Note that final object in a category is
the same up to isomorphism, thus, these are all possible final objects.

\subsection*{5.2}
\subsection*{5.3}
Use the universal property of free groups to prove that the map $j : A \rightarrow F(A)$
is injective, for all sets $A$.
\begin{proof}
  The universal property indicates that the following commutative diagram holds for any objects $(G, j_2)$:
  $$
    \begin{tikzcd}[row sep=huge]
      F(A) \ar[r, "\varphi"] & G\\
      A \ar[u, "j"]  \ar[ur, "j_1"] &
    \end{tikzcd}
  $$
  Specifically, let $j_1$ be injective set-function, we must have $j_1=\varphi\circ j$, the fact that
  $j_1$ is injective indicates $j$ is injective. The difficulty is to show that such $j_1$ and $G$ exists.

\end{proof}

\subsection*{5.5}
Verify explicitly that $H^{\oplus A}$ is a group.
\begin{proof}
  $H^{\oplus A}$ is a subset of $H^{A}$ that consists of set-functions only has finitely many ``non-zero" images.
  For $\alpha_1, \alpha_2 \in H^{\oplus A}$, we have $\alpha_1 + \alpha_2 \in H^{A}$ by defining:
  $$
    (\alpha_1 + \alpha_2)(a) = \alpha_1(a) + \alpha_2(a)
  $$
  Note that $\alpha_1$ and $\alpha_2$ has at most finitely many non-zero images, thus $\alpha_1 + \alpha_2$ has only finitely many
  non-zero images. Further, we have the zero element: $\mathbf{0}: a\mapsto 0_{H}$ and addition inverse:$-\alpha: a\mapsto -\alpha(a)$.
  Thus $H^{\oplus A}$ is a group. The commutativaty of $H$ also indicates that $H^{\oplus A}$ is an abelian group.
\end{proof}

\subsection*{5.6}
Prove that the group $F(\{x, y\})$ (visualized in Example 5.3) is a coproduct
$\mathbb{Z} * \mathbb{Z}$ of $\mathbb{Z}$ by itself in the category \textbf{Grp}.

\begin{proof}
  There is a explicit proof to show that $F(\{x, y\})$ is the coproduct of $\mathbb{Z}$ and $\mathbb{Z}$:
  We have the following diagram:
  $$
    \begin{tikzcd}
      \mathbb{Z} \ar[rd, "\iota_{1}"] & \\
      & F(\{x, y\}) \\
      \mathbb{Z} \ar[ru, "\iota_{2}"]
    \end{tikzcd}
  $$
  $\iota_1$ and $\iota_2$ are homomorphims derived by defining $\iota_1(1) = x$ and $\iota_2(1) = y$.
  Then for any other group $G$ and $f_1, f_2$ we have to prove the next diagram holds:
  \begin{center}
    $$
      \begin{tikzcd}[row sep=huge]
        & G &  \\
        \mathbb{Z} \ar[ru, "f_1"] \ar[r, "\iota_{1}"] & F(\{x, y\}) \ar[u, dashed, "\exists!\varphi"]& \ar[lu, "f_2"] \ar[l, "\iota_{2}"] \mathbb{Z}
      \end{tikzcd}
    $$
  \end{center}
  Define $\varphi$ such that $\varphi(x) = f_1(1)$ and $\varphi(y) = f_2(1)$. Then we have such 
  $\varphi$ is a homomorphim and is unique. Thus, the free group on $\{x, y\}$ is a coproduct of $\mathbb{Z}$ and $\mathbb{Z}$.
\end{proof}

\subsection*{5.7}
Extend the result of Exercise 5.6 to free groups $F(\{x_1,\ldots, x_n\})$ and to free
abelian groups $F^{ab}(\{x_1,\ldots, x_n\})$\\

\noindent
\textit{Solution}
  The Extended result is that: $F(\{x_1,\ldots, x_n\})$ is the coproduct of $n$ $\mathbb{Z}$ in category $\mathbf{Grp}$ and is 
  a coproduct of $n$ $\mathbb{Z}$ in category Ab.

\subsection*{5.8}
Still more generally, prove that $F(A\sqcup B) = F(A)*F(B)$ and that $F^{ab}(A\sqcup B) =
F^{ab}(A) \oplus F^{ab}(B)$ for all sets $A, B$. 
\begin{proof}
  We will only prove the fact that $F(A\sqcup B) = F(A) * F(B)$. In this question, we can only use the universal property.
  To prove that $F(A\sqcup B)$ is the coproduct of $F(A)$ and $F(B)$, we first construct the ``injection" homomorphim:
  Here is the diagram:
  $$
  \begin{tikzcd}
    A \ar[r, "i_{A}"] \ar[d, "\iota_{A}"] & F(A) \ar[d, dashed, "I_{F(A)}"]\\
    A\sqcup B \ar[r, "i_{A\sqcup B}"] & F(A\sqcup B) \\
    B \ar[r, "i_{B}"] \ar[u, "\iota_{B}"] & F(B) \ar[u, dashed, "I_{F(B)}"]\\
  \end{tikzcd}
  $$
  Note that the set-function $i_{A\sqcup B}\circ\iota_{A}$(or $i_{A\sqcup B}\iota_{B}$) is a function from $A$(or $B$) to $F(A\sqcup B)$, 
  according to the universal property of $F(A)$, there exists a unique homomorphim $I_{F(A)}$(or $I_{F(B)}$) such that 
  $I_{F(A)}\circ i_{A} = i_{A\sqcup B}\circ \iota_{A}$ and $I_{F(B)}\circ i_{B} = i_{A\sqcup B}\circ \iota_{B}$. We prove that 
  $(F(A\sqcup B), I_{F(A)}, I_{F(B)})$ is a coproduct of $F(A)$ and $F(B)$.

  Say $G$ is another group with homomorphim$f_{F(A)}: F(A)\rightarrow G$ and $f_{F(B)}: F(B)\rightarrow G$. Then we have:
  $$
  \begin{tikzcd}
    A \ar[r, "i_{A}"] \ar[d, "\iota_{A}"] & F(A) \ar[d, "f_{F(A)}"]\\
    A\sqcup B \ar[r, dashed, "f"] & G \\
    B \ar[r, "i_{B}"] \ar[u, "\iota_{B}"] & F(B) \ar[u, "f_{F(B)}"]\\
  \end{tikzcd}
  $$
  Note that $A\sqcup B$ is a coproduct of $A$ and $B$, then there is a set function $f$ such that $f\circ \iota_{A} = f_{F(A)}\circ i_{A}$ and 
  $f\circ \iota_{B} = f_{F(B)}\circ i_{B}$.

  According to the universal property of $F(A\sqcup B)$, there exists some $\varphi$ such that the following diagram commutes:
  $$
  \begin{tikzcd}
    A\sqcup B \ar[r, "i_{A\sqcup B}"] \ar[rd, "f"] & F(A\sqcup B) \ar[d, dashed, "\varphi"]\\
    & G
  \end{tikzcd}
  $$
  We have to prove that $f_{F(A)} = \varphi \circ I_{F(A)}$ and $f_{F(A)} = \varphi \circ I_{F(B)}$ and such $\varphi$ is unique.
  We only prove that $f_{F(A)} = \varphi \circ I_{F(A)}$ due to similarity.

  Note that $I_{F(A)}\circ i_{A} = i_{A\sqcup B}\circ \iota_{A}$, we have:$\varphi \circ I_{F(A)}\circ i_{A} = \varphi \circ i_{A\sqcup B} \circ \iota_{A} = (\varphi\circ i_{A\sqcup B})\circ \iota_{A} 
    = f\circ \iota_{A} = f_{F(A)} \circ i_{A}
  $ that is $(\varphi \circ I_{F(A)})\circ i_{A} = f_{F(A)}\circ i_{A}$.

  In the following diagram:
  $$
  \begin{tikzcd}[column sep=huge]
     & \ar[ld, dashed, "\varphi \circ I_{F(A)}"]F(A) \ar[rd, "f_{F(A)}"]& \\
  G  &  \ar[l, "(\varphi \circ I_{F(A)})\circ i_{A}"]A \ar[u, "i_{A}"] \ar[r, "f_{F(A)}\circ i_{A}"] & G
  \end{tikzcd}
  $$
  According to the universal property of $F(A)$, we must have:$\varphi \circ I_{F(A)} = f_{F(A)}$ due to the uniqueness.
  To prove the uniqueness of $\varphi$, we assume that $\varphi^{'}$ satisfies $\varphi^{'}\circ I_{F(A)} = f_{F(A)}$(same for $B$), we have
  $\varphi^{'}\circ I_{F(A)}\circ i_{A} = f_{F(A)}\circ i_{A}$. The left side equals to 
  $\varphi^{'}\circ(i_{A\sqcup B}\circ \iota_{A})$, thus we have:$(\varphi^{'}\circ i_{A\sqcup B})\circ \iota_{A} = f_{F(A)}\circ i_{A}$. According to the universal property of $A\sqcup B$, we have $f = \varphi^{'}\circ i_{A\sqcup B} \Rightarrow \varphi \circ i_{A\sqcup B} = \varphi^{'}\circ i_{A\sqcup B}$.
  And we are done.

\end{proof}

  \section*{6. Subgroups}
  \subsection*{6.2}
  Prove that the set of $2 \times 2$ matrices 
  $$
  \begin{pmatrix}
    a & b\\
    0 & d
  \end{pmatrix}
  $$ with $a,b,d$ in $\mathbb{C}$ is a subgroup of $\mbox{GL}_{2}(\mathbb{C})$. 
  More generally, prove that the set of $n\times n$ complex matrices $(a_{ij})_{1\leq i,j\leq n}$ with $a_{ij} = 0$ for $i > j$, and $a_{11}\cdots a_{nn} \neq 0$,
  is a subgroup of $\mbox{GL}_n(\mathbb{C})$. (These matrices are called `upper triangular', for evident
  reasons.)

  \begin{proof}
    Let $A$ denote the set compries matrix described in this question, then for any $a,b\in A$, we have:
    $$
    ab^{-1} = \begin{pmatrix}
      a_{1} & b_{1}\\
      0 & d_{1}\\
    \end{pmatrix}
    \times
    \frac{1}{ad}
    \begin{pmatrix}
      d_{2} & -b_{2}\\
      0 & a_{2}
    \end{pmatrix}
    =\frac{1}{ad} \begin{pmatrix}
      a_1d_2 & b_1a_2 - a_1b_2 \\
      0 & d_1a_2
    \end{pmatrix}
    $$
    And $(a_1d_2)(d_1a_2) = (a_1d_1)(a_2d_2)\neq 0$. Thus we have $ab^{-1}\in A$ and $A$ is 
    a subgroup of $\mbox{GL}_{2}(\mathbb{C})$.

    For a more general case, we show that the multiplication of two `upper triangular' matrix is still 
    `upper triangular' and the inverse of an `upper trivial' matrix is still upper trivial.

    If $A$ and $B$ are `upper triangular' matrixes, then for $AB$ we have:
    $$
    (AB)_{ij} = \sum_{k = 1}^{n}a_{ik}b_{kj}
    $$ 
    For $i > j$, note that:
    $$
      a_{ik}b_{kj} = 
    \begin{cases}
      0, a_{ik} = 0, i > k  \\
      0, b_{kj} = 0, k \geq i > j
    \end{cases}
    $$
    Thus, we have $(AB)_{ij} = 0$ for $i > j$. This indicates that $AB$ is still `upper triangular'.

    For the second proposition, we induct on $n$: for $n=2$, the case is proved above; Let's assume this proposition is held
    for $n=k$, and for $n=k+1$, for any `upper triangular' matrix, it could be written as:
    $$
    B = 
    \begin{pmatrix}
      a_{11} & B_{1\times k}\\
      \mathbf{0}_{k\times 1} & T_{k\times k}
    \end{pmatrix}
    $$ where $a_{11}\neq 0$ and $T_{k\times k}$ is an `upper triangular' matrix of order $n$.
    We have its inverse as:
    $$
    B^{-1} = 
    \begin{pmatrix}
      a_{11}^{-1} & -a_{11}^{-1}B_{1\times k}T_{k\times k}^{-1} \\
      \mathbf{0}_{k\times 1} & T_{k\times k}^{-1}
    \end{pmatrix}
    $$
    According to the assumption that $T_{k\times k}^{-1}$ is an `upper triangular', we have $B^{-1}$ is also `upper triangular'.

    With above two propositions, for any $a,b\in A_{n}$, we have $ab^{-1}$ is still an `upper triangular' matrix, thus $ab^{-1}\in A_{n}$ and the 
    proof is done.
  \end{proof}

  \subsection*{6.3}
  Prove that every matrix in $\mbox{SU}_2(\mathbb{C})$ may be written in the form
  $$
  \begin{pmatrix}
    a + bi & c + di \\
    -c + di & a - bi\\
  \end{pmatrix}
  $$
where $a, b, c, d \in \mathbb{R}$ and $a^2 + b^2 + c^2 + d^2 = 1$. (Thus, $\mbox{SU}_2(\mathbb{C})$ may be realized as
a three-dimensional sphere embedded in $\mathbb{R}^{4}$; in particular, it is simply connected.)

\begin{proof}
  Let $M\in \mbox{SU}_{2}(\mathbb{C})$ and 
  $$
  M = \begin{pmatrix}
    x & y \\
    z & w
  \end{pmatrix}
  $$. We have $$
    \begin{pmatrix}
      x & y\\
      z & w
    \end{pmatrix}
    \begin{pmatrix}
      \overline{x} & \overline{z}\\
      \overline{y} & \overline{w}
    \end{pmatrix} = \begin{pmatrix}
      1 & 0\\
      0 & 1
    \end{pmatrix}
    = \begin{pmatrix}
      \overline{x} & \overline{z}\\
      \overline{y} & \overline{w}
    \end{pmatrix}
    \begin{pmatrix}
      x & y\\
      z & w
    \end{pmatrix}
  $$
  That means: $$
  \begin{cases}
    x\overline{z} + y\overline{w} = 0 \\
    z\overline{x} + w\overline{y} = 0 \\
    \overline{x}y + \overline{z}w = 0 \\
    \overline{y}x + \overline{w}z = 0 \\
  \end{cases}
  $$
\end{proof}

\subsection*{6.5}
Let $G$ be a commutative group, and let $n > 0$ be an integer. Prove that
$\{g^{n} \mid g \in G\}$ is a subgroup of $G$. Prove that this is not necessarily the case if $G$ is
not commutative
\begin{proof}
  For any $a, b\in G$, we have $a = g^{n}, b = h^{n}$ for some $g,h\in G$, and $b^{-1} = (h^{-1})^{n}$. Thus:
  $$
  ab^{-1} = g^{n} (h^{-1})^{n} = (gh^{-1})^n
  $$
  Note that $gh^{-1}\in G$, thus $ab^{-1}\in \{g^{n}\mid g\in G\}$, which means this group is a subgroup of $G$.
  An counter example of the latter assertion would be the permutation group $S_{4}$ and let $n = 2$.
\end{proof}

\subsection*{6.7}
Show that inner automorphisms (cf. Exercise 4.8) form a subgroup of
$\mbox{Aut}(G)$; this subgroup is denoted $\mbox{Inn}(G)$. Prove that $\mbox{Inn}(G)$ 
is cyclic if and only if $\mbox{Inn}(G)$ is trivial if and only if $G$ is abelian
\begin{proof}
  For $\gamma_{a}, \gamma_{b}\in \mbox{Inn}(G)$, we have $\gamma_{a}\gamma_{b}^{-1} = \gamma_{ab^{-1}}\in \mbox{Inn}(G)$. Thus 
  it is a subgroup of $\mbox{Aut}(G)$. 
  
  $\mbox{Inn}(G)$ is trivial is obviously equavialent to the fact that $G$ is abelian. If $\mbox{Inn}(G)$ is cyclic, then there exists 
  some $a\in G$ such that for any $g\in G$, there exists some $n\in \mathbb{N}^{+}$ such that $\gamma_{a^{n}} = \gamma_{g}$, this indicates
  $gag^{-1} = a^{n}aa^{-n} = a$ and thus $ga = ag, \forall g\in G$. Thus we have $\forall \gamma_{g}\in \mbox{Inn}(G)$, $\gamma_{g}=\gamma_{a^{m}}$
  and $\forall x\in G, \gamma_{a^{m}}(x) = x$, thus $\gamma_{g} = \mbox{Id}_{G}$. The proof is done.
\end{proof}

\subsection*{6.9}
Prove that every finitely generated subgroup of $\mathbb{Q}$ is cyclic. Prove that $\mathbb{Q}$ is
not finitely generated

\begin{proof}
  Let $H<G$ be a finitely generated subgroup and $H=\langle a_1,a_2,...,a_n\rangle$. We induct on $n$ to 
  prove that $H$ is cyclic: \\
  \noindent
  (1) If $n=1$ then we have $F(\{a_1\})$ to be cyclic, thus $H=\varphi(F(\{a_1\}))$ is also cyclic\\
  (2) Assume for $n$ this holds, consider $n+1$. Since $H^{'}=\langle a_1,a_2,...,a_n\rangle$ is cyclic, there 
  exits some $q\in \mathbb{Q}$ such that $H^{'} = \langle q\rangle$, Consider $a_{n+1}$ and $q$, let's say $a_{n+1}$
  and $q$ both has the form: $q=\cfrac{s}{t}, a_{n+1}=\cfrac{s^{'}}{t}$. Consider $q^{'} = \mbox{gcd}(s, s^{'})$ and we will 
  have both $q$ and $a_{n+1}$ be multiple $\cfrac{q^{'}}{t}$. Note that $\mbox{gcd}(\cfrac{s}{q^{'}}, \cfrac{s^{'}}{q^{'}}) = 1$.
  We will have $x, y\in \mathbb{N}$ such that $\cfrac{xs}{q^{'}} + \cfrac{ys^{'}}{q^{'}} = 1$, by multiplying $\cfrac{q^{'}}{t}$ at both sides:
  $$
  \cfrac{q^{'}}{t} = \cfrac{xs}{t} + \cfrac{ys^{'}}{t}
  $$
  This means: $\cfrac{q^{'}}{t}\in \langle a_1, a_2,..., a_{n+1} \rangle$ and it's obviously that each element can be 
  expressed as multiple of $\cfrac{q^{'}}{t}$. Thus the proposition is true for the case of $n+1$. 
  
  In conclusion, we have proved that any finitely generated subgroup of $\mathbb{Q}$ is cyclic.

  $\mathbb{Q}$ is not finitely generated as $\mathbb{Q}$ is not cyclic.
\end{proof}

\subsection*{6.10}
The set of $2\times2$ matrices with integer entries and determinant 1 is denoted
$\mbox{SL}_2(\mathbb{Z})$:
$$
\mbox{SL}_2(\mathbb{Z}) = \bigg\{
  \begin{pmatrix}
    a & b \\
    c & d
  \end{pmatrix}
  \mid
  \;
  \mbox{such that}\;a, b, c, d\in \mathbb{Z},ad-bc=1
\bigg\}
$$
Prove that $\mbox{SL}_{2}(\mathbb{Z})$ is generated by the matrices:
$$
  s = \begin{pmatrix}
      0 & -1\\
      1 & 0
  \end{pmatrix}
  ,
  \quad
  t = \begin{pmatrix}
    1 & 1\\
    0 & 1
  \end{pmatrix}
$$

\begin{proof}
  Using induction, we have 
  $t^{a} = \begin{pmatrix}
    1 & a\\
    0 & 1
  \end{pmatrix}, a\in \mathbb{N}$ and 
  $
  s^{2} = \begin{pmatrix}
    -1 & 0\\
    0 & -1
  \end{pmatrix},
  s^{3} = \begin{pmatrix}
    1 & 0 \\
    0 & 1
  \end{pmatrix}
  $
\end{proof}

\subsection*{6.12}
Let $m, n$ be positive integers, and consider the subgroup $\langle m, n\rangle$ of $\mathbb{Z}$ 
they generate. By Proposition 6.9,
$\langle m, n\rangle = d\mathbb{Z}$
for some positive integer $d$. What is $d$, in relation to $m, n$?

\begin{proof}
  Since $\langle m, n\rangle = d\mathbb{Z}$, there exits some $x, y\in \mathbb{N}$ such that 
  $xm + yn = d$. Thus we have $\mbox{gcd}(m, n) \mid d$. On the contrary, note that $m\in d\mathbb{Z}$ and 
  $n\in d\mathbb{Z}$, thus we have $d\mid m$ and $d\mid n$, which indicates $d\mid \mbox{gcd}(m, n)$. Thus 
  we have $\mbox{gcd}(m, n) = d\mathbb{Z}$.
\end{proof}

\end{document}

