\documentclass[a4paper, 11pt]{article}
\usepackage{amsmath, amsthm, amssymb, amsfonts}
\usepackage{graphics}
\usepackage{tikz-cd}
\usepackage{enumerate}
\usepackage[a4paper,top=3cm,bottom=2cm,left=3cm,right=3cm,marginparwidth=1.75cm]{geometry}

\newenvironment{myprf}
{\renewcommand\qedsymbol{$ $}\begin{proof}[$\textbf{Proof}$]}
  {\end{proof}}

\newenvironment{solution}
{\renewcommand\qedsymbol{$ $}\begin{proof}[$\textbf{Solution}$]}
  {\end{proof}}
\title{Chapter5 Fields and Galois theory Solutions}
\date{October 5, 2020}
\begin{document}
       \maketitle

\section*{Field Extensions}
\subsection*{1.}
\noindent
(a) $[F:K]=1$ if and only if $F=K$.\\
(b) If $[F:K]$ is a prime, then there are no intermediate fields between $F$ and $K$\\
(c) If $u\in F$ has degree $n$ over $K$, then $n$ divides $[F:K]$\\
\begin{myprf}
        (a) $\Rightarrow$: If $[F:K]=1$ then let $\{u\},u\in F$ be the basis of $F$. If $u=0$ then $F=0$ as 
        every elements in $F$ has the form $ku$ for some $k\in K$.Let $f$ be a map,
        which is $f:K\rightarrow F, k\mapsto ku$. Then it's easy to see that $f$ is injective.
        By the fact that every element in $F$ has form $ku$ for some $k\in K$, we have
        $f$ is surjective, hence $f$ is bijective. Therefore $F=K$.\\
        \indent
        $\Leftarrow$ If $F=K$ then any nonzero element could be the basis of $F$ over $K$
        \\
        (b)If there is some intermediate field $E$ between $F$ and $K$ then we have
        $$
        [F:K]=[F:E][E:K]
        $$
        which means $[F:E]=1$ or $[E:K]$=1 as $[F:K]$ is prime. Therefore we have $F=E$ 
        or $E=K$ by (a).
        \\
        (c) By the condition, let $f$ be the minimal polynomial of $u$ over $K$, we have
        that $1,u,u^2,...,u^{n-1}$ is a basis of $K(u)$(\textbf{Theorem1.6}) Notice that $K(u)$ is an 
        intermediate field between $K$ and $F$, we have $n$ divides $[F:K]$ by \textbf{Theorem1.2}
\end{myprf}

\subsection*{2.}
\noindent Give an example of a finitely generation field extension, which is not finite 
dimensional.
\begin{solution}
        Consider $\mathbb{Q}(e)$, it's obvious that $\mathbb{Q}(e)$ is a finitly 
        generated extension but $\mathbb{Q}(e)$ is not finite dimensional over 
        $\mathbb{Q}$, otherwise $e$ is algebraic over $\mathbb{Q}$, which is false.
\end{solution}

\subsection*{3.}
\noindent If $u_1,u_2,...,u_n\in F$ then the field $F(u_1,...,u_n)$ is isomorphic to the 
quotient field of the ring $K[u_1,...,u_n]$.
\begin{myprf}
        Define map between $F(u_1,...,u_2)$ and the quotient field of $F[u_1,...,u_n]$
        as follows:
        $$
        f: h(u_1,...,u_n)/k(u_1,...,u_n) \mapsto (h(u_1,...,u_n), k(u_1,...,u_n))
        $$It's easy to see that $f$ is an isomorphism.
\end{myprf}

\subsection*{4.}
\noindent
(a) For any $u_1,...,u_n\in F$ and any permutation $\sigma \in S_n,K(u_1,...,u_n)
    = K(u_{\sigma(1)},...,u_{\sigma(n)})$\\
    (b) $K(u_1,...,u_{n-1})(u_n)=K(u_1,...,u_{n-1}, u_n)$\\
    (c) State and prove the analogues of (a) and (b) for $K[u_1,...,u_n]$.\\
    (d) If each $u_i$ is algebraic over $K$, then $K(u_1,...,u_n)=K[u_1,...,u_n]$\\

\begin{myprf}
        \noindent
        (a) According to the definition and remark after $\textbf{Theorem1.2}$, 
        $K(u_1,...,u_n)$ is the subfield generated by $F\cup \{u_1,...,u_n\}$ and 
        $K(u_{\sigma(1)},...,u_{\sigma(n)})$ is the subfield generated by $F\cup
        \{u_{\sigma(1)},...,u_{\sigma(n)}\}$. These two sets are equal as $\sigma$ is 
        bijective.\\
        \\
        (b)$K(u_1,...,u_{n-1})(u_n)$ is a subfield (of $F$) that contains $u_1,..,u_{n-1}
        ,u_n$, therefore according to the difinition of $K(u_1,...,u_n)$, we have:
        $$
        K(u_1,...,u_n)\subset K(u_1,...,u_{n-1})(u_n)
        $$
        On the other hand, $K(u_1,...,u_{n-1})(u_n)$ is the subfield generated by 
        $K(u_1,...,u_{n-1})\cup \{u_n\}$ Notice that $K(u_1,...,u_n)$ contains 
        $K(u_1,...,u_{n-1})$ and $u_n$, we have:
        $$
        K(u_1,...,u_{n-1})(u_n)\subset K(u_1,...,u_n)
        $$ therefore these two subfield are equal.\\
        \\
        (c) The analogues of $K[u_1,...,u_n]$ are easy to write and prove as long as 
        we replace "subfield"  with "subring".\\
        \\
        (d) We prove by induction: when $n=1$ this holds as $K(u)=K[u]$, which is showed  
        in $\textbf{Theorem1.6}$. Let's assume $K(u_1,...,u_{n-1})=K[u_1,...,u_{n-1}]$,
        then $u_n$ is algebraic over $K$ implies $u_n$ is also algebraic over
        $K(u_1,...,u_{n-1})$. We have:
        $$
        K(u_1,...,u_n)=K(u_1,...,u_{n-1})(u_n)=K[u_1,...,u_{n-1}](u_n)=K[u_1,...,u_{n-1}][u_n]=K[u_1,...,u_n]
        $$ The count-down-2 equation follows from the conclusion of adding one algebraic 
        element.
\end{myprf}

\subsection*{5.}
\noindent
Let $L$ and $M$ be subfields of $F$ and $LM$ their composite.\\
\indent (a) If $K\subset L\cap M$ and $M=K(S)$ for some $S\subset M$, then $LM=L(S)$.\\
\indent (b) When is it true that $LM$ is the set theoretic union $L\cup M$\\
\indent (c) If $E_1,...,E_n$ are subfields of $F$, show that
$$
E_{1}E_{2}...E_{n}=E_1(E_2(...(E_{n-1}(E_{n}))...)).
$$
\begin{myprf}
       PASS
\end{myprf}
        
\subsection*{6.}
Every element of $K(x_1,...,x_n)$ which is not in $K$ is transcendental over $K$.
\begin{myprf}
       PASS: I feel this question is incorrect
\end{myprf}
\subsection*{7.}
If $v$ is algebraic over $K(u)$ for some $u\in F$ and $v$ is transcendental over $K$,
then $u$ is algebraic over $K(v)$.
\begin{myprf}
    $v$ is algebraic over $K(u)$ means there is some polynomial $f\in K(u)[x]$ such that
    $f(u)=0$. We can write this in the following form:
    $$
    \sum_{i=0}^{n}\cfrac{h_i(u)}{k_i(u)} v^{i} = 0, h_i(x),k_i(x)\in K[x]
    $$. By multiplying $\displaystyle \prod_{i=0}^n h_i(u)$ we have:
    $$
    \sum_{i=0}^{n}F_i(u)v^{i}=0, F_i(u)=\prod_{j\neq i}k_j(u)h_i(u)
    $$ If we combine all coefficiences of each $u^i$ together, we will have:
    $$
    \sum_{i=0}^{m}G_i(v)u^i=0, G_i(x)\in K[x]
    $$ Notice that $G_i(v)\neq 0,\forall i=0,...,m$ as $v$ is transcendental over $K$. 
    We have $u$ is algebraic over $K(v)$.
\end{myprf}

\subsection*{8.}
If $u\in F$ is algebraic of odd degree over $K$, then so is $u^2$ and $K(u)=K(u^2)$
\begin{myprf}
        If $u$ is algebraic over $K$ then $[F(u):F]$ is finite and equals to the degree
        of the minimal polynomial of $u$. It's easy to see that $K(u^2)$ is an 
        intermediate between $K$ and $K(u)$, according to $\textbf{Theorem1.2}$ we have
        $[K(u^2):K] \mid [K(u):K]$. Now that $[K(u):K]$ is odd, so is $[K(u^2):K]$ and 
        $u^2$ has odd degree, which shows $u^2$ is also algebraic over $K$\\
        \noindent 
       
\end{myprf}


\end{document}

