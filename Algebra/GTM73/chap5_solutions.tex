\documentclass[a4paper, 11pt]{article}
\usepackage{amsmath, amsthm, amssymb, amsfonts}
\usepackage{graphics}
\usepackage{tikz-cd}
\usepackage{enumerate}
\usepackage[a4paper,top=3cm,bottom=2cm,left=3cm,right=3cm,marginparwidth=1.75cm]{geometry}

\newenvironment{myprf}
{\renewcommand\qedsymbol{$ $}\begin{proof}[$\textbf{Proof}$]}
  {\end{proof}}

\newenvironment{myremark}
{\renewcommand\qedsymbol{$ $}\begin{proof}[$\textbf{REMARK}$]}
  {\end{proof}}

\newenvironment{solution}
{\renewcommand\qedsymbol{$ $}\begin{proof}[$\textbf{Solution}$]}
  {\end{proof}}
\title{Chapter5 Fields and Galois theory Solutions}
\date{October 5, 2020}
\begin{document}
       \maketitle

\section*{Field Extensions}
\subsection*{1.}
\noindent
(a) $[F:K]=1$ if and only if $F=K$.\\
(b) If $[F:K]$ is a prime, then there are no intermediate fields between $F$ and $K$\\
(c) If $u\in F$ has degree $n$ over $K$, then $n$ divides $[F:K]$\\
\begin{myprf}
        (a) $\Rightarrow$: If $[F:K]=1$ then let $\{u\},u\in F$ be the basis of $F$. If $u=0$ then $F=0$ as 
        every elements in $F$ has the form $ku$ for some $k\in K$.Let $f$ be a map,
        which is $f:K\rightarrow F, k\mapsto ku$. Then it's easy to see that $f$ is injective.
        By the fact that every element in $F$ has form $ku$ for some $k\in K$, we have
        $f$ is surjective, hence $f$ is bijective. Therefore $F=K$.\\
        \indent
        $\Leftarrow$ If $F=K$ then any nonzero element could be the basis of $F$ over $K$
        \\
        (b)If there is some intermediate field $E$ between $F$ and $K$ then we have
        $$
        [F:K]=[F:E][E:K]
        $$
        which means $[F:E]=1$ or $[E:K]$=1 as $[F:K]$ is prime. Therefore we have $F=E$ 
        or $E=K$ by (a).
        \\
        (c) By the condition, let $f$ be the minimal polynomial of $u$ over $K$, we have
        that $1,u,u^2,...,u^{n-1}$ is a basis of $K(u)$(\textbf{Theorem1.6}) Notice that $K(u)$ is an 
        intermediate field between $K$ and $F$, we have $n$ divides $[F:K]$ by \textbf{Theorem1.2}
\end{myprf}

\subsection*{2.}
\noindent Give an example of a finitely generation field extension, which is not finite 
dimensional.
\begin{solution}
        Consider $\mathbb{Q}(e)$, it's obvious that $\mathbb{Q}(e)$ is a finitly 
        generated extension but $\mathbb{Q}(e)$ is not finite dimensional over 
        $\mathbb{Q}$, otherwise $e$ is algebraic over $\mathbb{Q}$, which is false.
\end{solution}

\subsection*{3.}
\noindent If $u_1,u_2,...,u_n\in F$ then the field $F(u_1,...,u_n)$ is isomorphic to the 
quotient field of the ring $K[u_1,...,u_n]$.
\begin{myprf}
        Define map between $F(u_1,...,u_2)$ and the quotient field of $F[u_1,...,u_n]$
        as follows:
        $$
        f: h(u_1,...,u_n)/k(u_1,...,u_n) \mapsto (h(u_1,...,u_n), k(u_1,...,u_n))
        $$It's easy to see that $f$ is an isomorphism.
\end{myprf}

\subsection*{4.}
\noindent
(a) For any $u_1,...,u_n\in F$ and any permutation $\sigma \in S_n,K(u_1,...,u_n)
    = K(u_{\sigma(1)},...,u_{\sigma(n)})$\\
    (b) $K(u_1,...,u_{n-1})(u_n)=K(u_1,...,u_{n-1}, u_n)$\\
    (c) State and prove the analogues of (a) and (b) for $K[u_1,...,u_n]$.\\
    (d) If each $u_i$ is algebraic over $K$, then $K(u_1,...,u_n)=K[u_1,...,u_n]$\\

\begin{myprf}
        \noindent
        (a) According to the definition and remark after $\textbf{Theorem1.2}$, 
        $K(u_1,...,u_n)$ is the subfield generated by $F\cup \{u_1,...,u_n\}$ and 
        $K(u_{\sigma(1)},...,u_{\sigma(n)})$ is the subfield generated by $F\cup
        \{u_{\sigma(1)},...,u_{\sigma(n)}\}$. These two sets are equal as $\sigma$ is 
        bijective.\\
        \\
        (b)$K(u_1,...,u_{n-1})(u_n)$ is a subfield (of $F$) that contains $u_1,..,u_{n-1}
        ,u_n$, therefore according to the difinition of $K(u_1,...,u_n)$, we have:
        $$
        K(u_1,...,u_n)\subset K(u_1,...,u_{n-1})(u_n)
        $$
        On the other hand, $K(u_1,...,u_{n-1})(u_n)$ is the subfield generated by 
        $K(u_1,...,u_{n-1})\cup \{u_n\}$ Notice that $K(u_1,...,u_n)$ contains 
        $K(u_1,...,u_{n-1})$ and $u_n$, we have:
        $$
        K(u_1,...,u_{n-1})(u_n)\subset K(u_1,...,u_n)
        $$ therefore these two subfield are equal.\\
        \\
        (c) The analogues of $K[u_1,...,u_n]$ are easy to write and prove as long as 
        we replace "subfield"  with "subring".\\
        \\
        (d) We prove by induction: when $n=1$ this holds as $K(u)=K[u]$, which is showed  
        in $\textbf{Theorem1.6}$. Let's assume $K(u_1,...,u_{n-1})=K[u_1,...,u_{n-1}]$,
        then $u_n$ is algebraic over $K$ implies $u_n$ is also algebraic over
        $K(u_1,...,u_{n-1})$. We have:
        $$
        K(u_1,...,u_n)=K(u_1,...,u_{n-1})(u_n)=K[u_1,...,u_{n-1}](u_n)=K[u_1,...,u_{n-1}][u_n]=K[u_1,...,u_n]
        $$ The count-down-2 equation follows from the conclusion of adding one algebraic 
        element.
\end{myprf}

\subsection*{5.}
\noindent
Let $L$ and $M$ be subfields of $F$ and $LM$ their composite.\\
\indent (a) If $K\subset L\cap M$ and $M=K(S)$ for some $S\subset M$, then $LM=L(S)$.\\
\indent (b) When is it true that $LM$ is the set theoretic union $L\cup M$\\
\indent (c) If $E_1,...,E_n$ are subfields of $F$, show that
$$
E_{1}E_{2}...E_{n}=E_1(E_2(...(E_{n-1}(E_{n}))...)).
$$
\begin{myprf}
       PASS
\end{myprf}
        
\subsection*{6.}
Every element of $K(x_1,...,x_n)$ which is not in $K$ is transcendental over $K$.
\begin{myprf}
       PASS: I feel this question is incorrect
\end{myprf}
\subsection*{7.}
If $v$ is algebraic over $K(u)$ for some $u\in F$ and $v$ is transcendental over $K$,
then $u$ is algebraic over $K(v)$.
\begin{myprf}
    $v$ is algebraic over $K(u)$ means there is some polynomial $f\in K(u)[x]$ such that
    $f(u)=0$. We can write this in the following form:
    $$
    \sum_{i=0}^{n}\cfrac{h_i(u)}{k_i(u)} v^{i} = 0, h_i(x),k_i(x)\in K[x]
    $$. By multiplying $\displaystyle \prod_{i=0}^n h_i(u)$ we have:
    $$
    \sum_{i=0}^{n}F_i(u)v^{i}=0, F_i(u)=\prod_{j\neq i}k_j(u)h_i(u)
    $$ If we combine all coefficiences of each $u^i$ together, we will have:
    $$
    \sum_{i=0}^{m}G_i(v)u^i=0, G_i(x)\in K[x]
    $$ Notice that $G_i(v)\neq 0,\forall i=0,...,m$ as $v$ is transcendental over $K$. 
    We have $u$ is algebraic over $K(v)$.
\end{myprf}

\subsection*{8.}
If $u\in F$ is algebraic of odd degree over $K$, then so is $u^2$ and $K(u)=K(u^2)$
\begin{myprf}
        If $u$ is algebraic over $K$ then $[F(u):F]$ is finite and equals to the degree
        of the minimal polynomial of $u$. It's easy to see that $K(u^2)$ is an 
        intermediate between $K$ and $K(u)$, according to $\textbf{Theorem1.2}$ we have
        $[K(u^2):K] \mid [K(u):K]$. Now that $[K(u):K]$ is odd, so is $[K(u^2):K]$ and 
        $u^2$ has odd degree, which shows $u^2$ is also algebraic over $K$\\
        \indent
         Let $f(x)=\displaystyle \sum_{i=0}^{p}k_ix^i$ be the minimal polynomial
        of $u$ over $K$, then we have:$\displaystyle \sum_{i=0}^p k_iu^i=0$. Do the 
        following transmission:
        $$
        u\sum_{i\;is\; odd}k_i(u^2)^{\frac{i-1}{2}}+\sum_{i\; is\; even}k_i(u^2)^
        {\frac{i}{2}} = 0
        $$ Let $\displaystyle h(x)=\sum_{i\;is\;odd}k_ix^{\frac{i-1}{2}},
        g(x)=\sum_{i\;is\;even}k_ix^{\frac{i}{2}}$ Then we have: $u=-\cfrac{g(u^2)}{h(u^2)}$($p$ is odd guarrantees $h(x)$ exists and not equals to 0, the minimal of $p$ guarrantees $h(u)\neq 0$),
        \\\indent Therefore we have $u\in K(u^2)$, hence $K(u)\subset K(u^2)$. It's obvious that
        $K(u^2)\subset K(u)$, then we have $K(u^2)=K(u)$
       
\end{myprf}
\subsection*{9.}
If $x^n-a\in K[x]$ is irreducible and $u\in F$ is a root of $x^n-a$ and $m$ divides $n$,
then prove that the degree of $u^m$ over $K$ is $n/m$. What is the irreducible polynomial
for $u^m$ over $K$?
\begin{myprf}
        $u$ is a root of $f(x)=x^n-a$ means $u^n-a=0$, we have $(u^m)^{\frac{n}{m}}-a=0$
        Let $g(x)=x^{\frac{n}{m}}-a$, we claim that $g(x)$ is the minimal polynomial
        of $u^m$. \\
        \indent If there is another $g^{'}(x)$ such that deg$g^{'}$ $<$ deg$g$ and $g^{'}(u^m)=0$.
        Then there is a polynomial $f^{'}(x)$ with degree of deg$g^{'}\times m$, which
        is less that deg$f$
        such that $f^{'}(u)=0$, which contradicts the defintion of minimal polynomial.
        Therefore we hace degree of $u^m$ over $K$ is $\frac{n}{m}$\\
        \indent The fact that there is only one minimal polynomial(let $f,g$ be minimal
        polynomials, then $f\mid g$ and $g\mid f$) shows that $x^{\frac{n}{m}}-a$ is the 
        minimal polynomial of $u^m$

 \end{myprf}
 
 \subsection*{10.}
 If $F$ is algebraic over $K$ and $D$ is an integral domain such that $K\subset D\subset F$
 ,then $D$ is a field

 \begin{myprf}
         For any element $d\in D\subset F$, consider $d^{-1} \in F$. Let $f(x)$ be the 
         minimal polynomial of $d^{-1}$ over $K$($F$ is algebraic over $K$ by condition)
         Then we have: $f(d^{-1})=0$, write it as :
         $$
         \sum_{i=0}^n k_i d^{-i} = 0\Rightarrow d^{-1}=(k_n)^{-1}d^{n-1}\sum_{i=0}^{n-1}
         k_i d^{-i} = (k_n)^{-1}\sum_{i=0}^{n-1}k_id^{n-1-i}
         $$ Therefore $d^{-1}\in D$ and $D$ is a subgroup of $F$ under multiplication
    
 \end{myprf}

 \subsection*{12.}
 If $d\ge 0$ is an integer that is not a square describe the field $\mathbb{Q}(\sqrt{d})$
 and find a set of elements that generate the whole field.
 \begin{solution}
         
         PASS: I don't understand what it means.
 \end{solution}
 
 \subsection*{13.}
 $\textbf{(a)}$ Consider the extension $\mathbb{Q}(u)$ of $\mathbb{Q}$ generated by a real root u
 of $x^3-6x^2+9x+3$.(Why is this irreducible?) Express each of the following elements
 in terms of the basis $\{1,u,u^2\}$:$u^4, u^5, 3u^5-u^4+2;(u+1)^{-1};(u^2-6u+8)^{-1}.$\\
 $\textbf{(b)}$ Do the same with respect to the basis $\{1,u,u^2,u^3,u^4\}$ of $\mathbb{Q}(u)$ where
 $u$ is a real root of $x^{5}+2x+2$ and the elements in question are: 
 $(u^2+2)(u^3+3u);u^{-1};u^{4}(u^{4}+3u^{2}+7u+5);(u+2)(u^2+3)^{-1}$
 
 \begin{solution}
         (a) $u^4=27u^2-57u-18$, $u^5=105u^2-261u-81$.\\
         By using Euclidean Algorithm, we calculated that $\gcd(x^3-6x^2+9x+3,x+1)=1$ and:
         $$
         -(x^3-6x^2+9x+3)+(x+1)\times\frac{1}{14}(x^2-8x+17)=1
         $$ Therefore $(u+1)^{-1}=\frac{1}{14}(u^2-8u+17)$.\\
         (b) The same as (a), but there are more calculations\\
 \end{solution}

 \subsection*{14.}
 $\textbf{(a)}$ If $F=\mathbb{Q}(\sqrt{2}, \sqrt{3})$,find $[F:\mathbb{Q}]$ and a basis of $F$ over $\mathbb{Q}$.\\
 $\textbf{(b)}$ Do the same for $F=\mathbb{Q}(i,\sqrt{3}, \omega)$,where $i\in \mathbb{C}, i^2=-1,$and $\omega$
 is a complex (nonreal) cube root of 1.

 \begin{solution}
         (a) $[\mathbb{Q}(\sqrt{2}, \sqrt{3}):\mathbb{Q}]=[\mathbb{Q}(\sqrt{2},\sqrt{3}):\mathbb{Q}(\sqrt{2})]
         \times [\mathbb{Q}(\sqrt{2}):\mathbb{Q}]$ It's easy to see that these two components are both 2, hence 
         we have : $[\mathbb{Q}(\sqrt{2}, \sqrt{3}):\mathbb{Q}]=4$.The basis are $\{1,\sqrt{2}, \sqrt{3},\sqrt{6} \}$.\\
         (b) Notice that $\omega=-\cfrac{1}{2}+\cfrac{\sqrt{3}}{2}i$, we have: $\omega\in \mathbb{Q}(i,\sqrt{3})$.
         Then we have:$[\mathbb{Q}(i,\sqrt{3}):\mathbb{Q}]=[\mathbb{Q}(i,\sqrt{3}):\mathbb{Q}(i)]\times 
         [\mathbb{Q}(i):\mathbb{Q}]$, which equals to $2\times2=4$. The basis is $\{1, i,\sqrt{3}, \sqrt{3}i\}$

 \end{solution}

 \subsection*{15.}
 In the field $K(x)$, let $u=x^3/(x+1)$. Show that $K(x)$ is a simple extension of the field $K(u)$. What is
 $[K(x):K(u)]$
 \begin{myprf}
         Let $v=x^2/(x+1)$, we will show that $K(u)(v)=K(x)$, which means $K(x)$ is simple extension of $K(u)$.
         $K(u)(v)\subset K(x)$ is obvious. Notice that $x=(x^3/(x+1))/(x^2/(x+1))=u/v$ We have:$x\in K(u)(v)$,
         moreover, any $f/g\in K(x)$ could be written as the combination of $x^i$, thus an element of $K(u)(v)$.
         Therefore we have $K(x)\subset K(u)(v)$ and $K(u)(v)=K(x)$.\\
 \end{myprf}

 \subsection*{16.}
 In the field $\mathbb{C}$, $\mathbb{Q}(i)$ and $\mathbb{Q}(\sqrt{2})$ are isomorphic as vector spaces, but not
 as fields.
 \begin{myprf}
         i and $\sqrt{2}$ is algebraic over $\mathbb{Q}$, thus by $\textbf{Theorem1.6}$ we have:
         $$
         \mathbb{Q}(i)=
         \mathbb{Q}[i]=\{f(i)\mid f(x)\in \mathbb{Q}[x]\}=\{a+bi\mid a,b\in \mathbb{Q}\} \\
         $$
         $$
         \mathbb{Q}(\sqrt{2})=
         \mathbb{Q}[\sqrt{2}]=\{f(i)\mid f(x)\in \mathbb{Q}[x]\}=\{a+b\sqrt{2}\mid a,b\in \mathbb{Q}\}\\
         $$ 
         Consider $\mathbb{Q}$-module homorphism:
         $$
         f:
         \mathbb{Q}(\sqrt{2})\rightarrow \mathbb{Q}(i):a+b\sqrt{2}
         \mapsto a+bi
         $$ Then $f$ is easy to be seen as $\mathbb{Q}$-module isomorphism, thus a vector space isomorphism.\\
         We will show that there is no filed-isomorphism between $\mathbb{Q}(i)$ and $\mathbb{Q}(\sqrt{2})$.
         If there is some field-isomorphism $f$ between these two fields of $\mathbb{C}$. Then we have:
         $$
         f(a+bi)=f(a)+f(b)f(i), \forall a,b\in \mathbb{Q}
         $$ Then we have:
         $$
         \begin{aligned}
                 f(a)^2+f(b)^2&=f(a^2+b^2)=f((a+bi)(a-bi))=f(a+bi)f(a-bi)\\&=(f(a)+f(b)f(i))(f(a)-f(b)f(i))=f(a)^2-f(b)^2f(i)^2
         \end{aligned}
         $$ This means $f(i)^2=-1$ which is impossible in $\mathbb{Q}(\sqrt{2})$
         \begin{myremark}
                 $\mathbb{Q}(i)$ and $\mathbb{Q}(\sqrt{2})$ as vector space is isomorphic is because they have
                 the same dimension. And the difference between vector space and fields($mathbb{Q}(\sqrt{2})$ and 
                 $\mathbb{Q}(i)$ is that 
                 $f((a+bi)(a-bi))=f(a+bi)f(a-bi)$ is not always true in vector space.
         \end{myremark}

        
         
 \end{myprf}


 


\end{document}

