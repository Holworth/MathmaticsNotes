\documentclass[a4paper]{article}
\usepackage{amsmath,amsthm,amssymb,amsfonts}
\usepackage{graphics}
\usepackage{tikz-cd} 
\usepackage{enumerate}
\newtheorem{myDef}{Definition} 
\newtheorem{myTheo}{Theorem}
\newtheorem{myCol}{Corollary}
%% Sets page size and margins
\usepackage[a4paper,top=3cm,bottom=2cm,left=3cm,right=3cm,marginparwidth=1.75cm]{geometry}
\newcommand{\R}{\mathbb{R}}
\newcommand{\N}{\mathbb{N}}
\newcommand{\Z}{\mathbb{Z}}
\providecommand{\C}{\mathbb{C}}
\title{Modules and Homorphism}
\begin{document}
    \maketitle
    \begin{myDef}
        Let $\mathrm{R}$ be a ring, an (left) $\mathbf{R-module}$ (denoted by $\mathrm{A}$) is an abelian group
        with a function $R\times A\rightarrow A$ satifies $\forall r,s \in \mathrm{R}, \forall a,b\in \mathrm{A}$,
        the following conditons holds:
        $$
        \begin{aligned}
        &r(a+b)=ra+rb\\
        &(r+s)a=ra+sa\\
        &(rs)a =r(sa)
        \end{aligned}
        $$
    \end{myDef}
    \noindent
    \textbf{note}:
    \begin{enumerate}[(i)]
        \item Let $\mathrm{R}$ be a ring with identity, and $\mathrm{A}$ satifies:$1_Ra=a, \forall a\in \mathrm{A}$,then $\mathrm{A}$ is called \textbf{unitary $\mathbf{R}$-module}
        \item If $\mathrm{R}$ is a division ring and $\mathrm{A}$ is an \textbf{unitary $\mathbf{R}$-module}, then $\mathrm{A}$ is called \textbf{vector space}\\
    \end{enumerate}
    
    \begin{myCol}
        $\forall r \in \mathrm{R},a\in\mathrm{A}$,we have:
        \begin{enumerate}[(i)]
            \item $r0_A = 0_A,0_ra=0_A$
            \item $-ra=(-r)a=r(-a)$
            \item n(ra)=(nr)a=r(na)
        \end{enumerate}
        \vspace{0.1cm}
        \begin{proof}
            the proof is trivial
        \end{proof}
        \vspace{0.3cm}
    \end{myCol}
    
    \begin{myDef}
        Let $\mathrm{R}$ be a ring and $\mathrm{A,B}$ be $\mathrm{R}$-module. A $\mathbf{R-module \;homorphism}$ f is an abelian group homorphism
        $A\rightarrow B$ satifies:$\forall a,b\in \mathrm{A},r\in \mathrm{R}$:\\
        $$
        f(a+b)=f(a)+f(b),f(ra)=rf(a)
        $$ 
        if $f$ is an abelian group $\mathbf{monomorphism}(resp.\mathbf{epimorphism,isomorphism})$ then $f$ is called an $\mathrm{R}$-module $\mathbf{monomorphism}(resp.\mathbf{epimorphism,isomorphism})$.
    The kernel of $f$ is the kernel of $f$ as an abelian group homorphism: $\ker f=\{a\in A|f(a)=0_B\}$
    \end{myDef}
    \noindent
    \textbf{note}:
    \begin{enumerate}[(i)]
        \item $f$ is monomorphism if and only if $\ker f$={$0_A$}
        \item $f$ is isomorphism if and only if there is an R-module $g:B\rightarrow A$ such that:$fg=1_B,gf=1_A$
        \item $f(0_A)=0_B$
    \end{enumerate}
    \vspace{0.5cm}
    \begin{myDef}
        Let $\mathrm{R}$ be a ring and $\mathrm{A}$ be an $\mathrm{R}$-module. A submodule of $\mathrm{A}$, say $\mathrm{B}$, is a subset of $\mathrm{A}$,satifies:
        $\forall a,b\in \mathrm{A}, r\in \mathrm{R}$:
        $$
        a-b\in \mathrm{B},ra\in \mathrm{B}
        $$
        In other words, $\mathrm{B}$ is a subgroup of $\mathrm{A}$ and is closed under the map. It's obviously that $\mathrm{B}$ is an $\mathrm{R}$-module itself. A submodule of a 
        vector space is called a subspace.
    \end{myDef}
    \textbf{EXAMPLES}
    \noindent
    \begin{enumerate}[(i)]
        \item Let $f:A\rightarrow B$ be an $\mathrm{R}$-module homorphism, then $\ker f$ is a submodule of $\mathrm{A}$ and $\mathrm{Im} f$ is a submodule of $\mathrm{B}$
        \item Let $I$ be a left ideal of $R$, $A$ an $\mathrm{R}$-module, $S$ a nonempty subset of $A$.Define $IS$ as follows:\\
        $$
        IS=\{\sum_{i=1}^nr_is_i|r_i\in I, s_i\in S, n\in \N^{*}\}
        $$
        then $IS$ is a submodule of $A$
        \item Let $A$ be an $\mathrm{R}$-module and $A_i,i\in I$ is a family of submodules of $\mathrm{A}$.Then $\cap_{i\in I}A_i$ is a submodule of $\mathrm{A}$
    \end{enumerate}
    \vspace{0.5cm}
    \begin{myDef}
        Let $\mathrm{R}$ be a ring, $\mathrm{A}$ a $\mathrm{R}$-module. X is a nonempty set of $\mathrm{A}$. $\mathbf{A\;submodule\;generated\\by\;X}$ is the intersection of all submodules that 
        contains $\mathrm{X}$.Let B is the submodule generated by X. If X is finite, then B is called $\mathbf{finitely\;generated}$; If $X=\{a\}$,then B is called $\mathbf{cyclic\;submodule}$.Let $B_i,i\in I$
        be a family of submodules of A, the submodule generated by $\cup_{i\in I}B_i$ is called the $\mathbf{sum}$ of submodules $B_i,i\in I$.
    \end{myDef}
    \textbf{REMARK} Submodule generated by $X$ is the smallest submodule that contains $X$. In other words, Let $B$ be the submodule of $A$ generated by $X$ and $C$ is any submodule of $A$ that contains $X$, we must have: $B\subset C$.\\
    To prove this, we only need to notice that $B=\cap_{X\subset C}C$. For any submodule that contains $X$, it must on the right side.
    \vspace{0.5cm}
    \begin{myTheo}
        Let $\mathrm{R}$ be a ring, $\mathrm{A}$ an $\mathrm{R}$-module, $\mathrm{X}$ a subset of $\mathrm{A}$, $\{B_i\;|\;i\in I\}$ a family of submodules of $\mathrm{A}$ and $a\in \mathrm{A}$. Let $\mathrm{Ra}=\{ra\;|\;r\in R\}$.
        \begin{enumerate}[(i)]
            \item $\mathrm{Ra}$ is a submodule of $\mathrm{A}$
            \item The cyclic submodule C generated by $\{a\}$ is $\{ra+na\;|\; r\in R, n\in \Z\}$
            \item The submodule generated by $\mathrm{X}$ is \\
            $$
            \{\sum_{i=1}^nr_ia_i+\sum_{j=1}^ms_jb_j\;|\; r_i\in \mathrm{R}, n,m\in \N^*,a_i,b_j\in X,s_j\in \Z\}
            $$
        \end{enumerate}
    \end{myTheo}
    \begin{proof}
        \begin{enumerate}[(i)]
            \item $\forall ra,sa\in \mathrm{Ra},\forall t\in\mathrm{R}$,we have:
            $$
            ra-sa=(r-s)a\in \mathrm{Ra},\;t(sa)=(ts)a\in \mathrm{Ra}
            $$
            According to the definiton of submodule, $\mathrm{Ra}$ is a submodule of A.
            \item First we need to show that $C=\{ra+na\;|\;r\in R, n\in \Z\}$ itself is a submoduel of A. The reason is as follows: $\forall r_1,r_2,s\in \mathrm{R},n_1,n_2\in \Z$:
            $$
            \begin{aligned}
            &(r_1a+n_1a)-(r_2a+n_2a)=(r_1-r_2)a+(n_1-n_2)a\in C\\
            &s(r_1a+n_1a)=(sr_1)a+s(n_1a)=(sr_1)a+(n_1s)a=(sr_1+n_1s)a\in C
            \end{aligned}
            $$ 
            Hence $C$ is a submodule of $\mathrm{A}$ that contains $\{a\}$. Besides, for any submodule $B_i$ that contains $\{a\}$, it's obviously $ra\in B_i,r\in R$ and $na\in B_i,n\in \Z$. Hence $C\subset B_i$.\\
            Let $B$ be the submodule generated by $X$. Then $B=\cap_{i\in I}B_i$, $B\subset C$ because $C$ is a submodule of $A$ contains $X$, hence one of $B_i$. $C\subset B$ is trivial since $C\subset B_i$ hence $C\subset \cap_{i\in I} B_i=B$. Therefore, $B=C$. 
            
            \item The method to proving (iii) is the same as the method used in (ii).
        \end{enumerate}
    \end{proof}
    \noindent
    \textbf{REMARK} In Theorem 1(ii), if $R$ is a ring with identity and $C$ an unitary module over $R$. The submodule generated by $\{a\}$ is $Ra$ as $na=(n1_R)a,n1_R\in R$.
\end{document}