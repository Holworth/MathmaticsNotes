\documentclass[a4paper]{article}
\usepackage{amsmath,amsthm,amssymb,amsfonts}
\usepackage{graphics}
\usepackage{tikz-cd} 
\usepackage{enumerate}
\newtheorem{myDef}{Definition} 
\newtheorem{myTheo}{Theorem}
\newtheorem{myCol}{Corollary}
%% Sets page size and margins
\usepackage[a4paper,top=3cm,bottom=2cm,left=3cm,right=3cm,marginparwidth=1.75cm]{geometry}
\newcommand{\R}{\mathbb{R}}
\newcommand{\N}{\mathbb{N}}
\newcommand{\Z}{\mathbb{Z}}
\providecommand{\C}{\mathbb{C}}
\title{Modules and Homomorphism}
\date{July 27,2020}
\begin{document}
    \maketitle
    \begin{myDef}
        Let $\mathrm{R}$ be a ring, an (left) $\mathbf{R-module}$ (denoted by $\mathrm{A}$) is an abelian group
        with a function $R\times A\rightarrow A$ satifies $\forall r,s \in \mathrm{R}, \forall a,b\in \mathrm{A}$,
        the following conditons holds:
        $$
        \begin{aligned}
        &r(a+b)=ra+rb\\
        &(r+s)a=ra+sa\\
        &(rs)a =r(sa)
        \end{aligned}
        $$
    \end{myDef}
    \noindent
    \textbf{note}:
    \begin{enumerate}[(i)]
        \item Let $\mathrm{R}$ be a ring with identity, and $\mathrm{A}$ satifies:$1_Ra=a, \forall a\in \mathrm{A}$,then $\mathrm{A}$ is called \textbf{unitary $\mathbf{R}$-module}
        \item If $\mathrm{R}$ is a division ring and $\mathrm{A}$ is an \textbf{unitary $\mathbf{R}$-module}, then $\mathrm{A}$ is called \textbf{vector space}\\
    \end{enumerate}
    
    \begin{myCol}
        $\forall r \in \mathrm{R},a\in\mathrm{A}$,we have:
        \begin{enumerate}[(i)]
            \item $r0_A = 0_A,0_ra=0_A$
            \item $-ra=(-r)a=r(-a)$
            \item n(ra)=(nr)a=r(na)
        \end{enumerate}
        \vspace{0.1cm}
        \begin{proof}
            the proof is trivial
        \end{proof}
        \vspace{0.3cm}
    \end{myCol}
    
    \begin{myDef}
        Let $\mathrm{R}$ be a ring and $\mathrm{A,B}$ be $\mathrm{R}$-module. A $\mathbf{R-module \;homomorphism}$ f is an abelian group homomorphism
        $A\rightarrow B$ satifies:$\forall a,b\in \mathrm{A},r\in \mathrm{R}$:\\
        $$
        f(a+b)=f(a)+f(b),f(ra)=rf(a)
        $$ 
        if $f$ is an abelian group $\mathbf{monomorphism}(resp.\mathbf{epimorphism,isomorphism})$ then $f$ is called an $\mathrm{R}$-module $\mathbf{monomorphism}(resp.\mathbf{epimorphism,isomorphism})$.
    The kernel of $f$ is the kernel of $f$ as an abelian group homomorphism: $\ker f=\{a\in A|f(a)=0_B\}$
    \end{myDef}
    \noindent
    \textbf{note}:
    \begin{enumerate}[(i)]
        \item $f$ is monomorphism if and only if $\ker f$={$0_A$}
        \item $f$ is isomorphism if and only if there is an R-module $g:B\rightarrow A$ such that:$fg=1_B,gf=1_A$
        \item $f(0_A)=0_B$
    \end{enumerate}
    \vspace{0.5cm}
    \begin{myDef}
        Let $\mathrm{R}$ be a ring and $\mathrm{A}$ be an $\mathrm{R}$-module. A submodule of $\mathrm{A}$, say $\mathrm{B}$, is a subset of $\mathrm{A}$,satifies:
        $\forall a,b\in \mathrm{A}, r\in \mathrm{R}$:
        $$
        a-b\in \mathrm{B},ra\in \mathrm{B}
        $$
        In other words, $\mathrm{B}$ is a subgroup of $\mathrm{A}$ and is closed under the map. It's obviously that $\mathrm{B}$ is an $\mathrm{R}$-module itself. A submodule of a 
        vector space is called a subspace.
    \end{myDef}
    \textbf{EXAMPLES}
    \noindent
    \begin{enumerate}[(i)]
        \item Let $f:A\rightarrow B$ be an $\mathrm{R}$-module homomorphism, then $\ker f$ is a submodule of $\mathrm{A}$ and $\mathrm{Im} f$ is a submodule of $\mathrm{B}$
        \item Let $I$ be a left ideal of $R$, $A$ an $\mathrm{R}$-module, $S$ a nonempty subset of $A$.Define $IS$ as follows:\\
        $$
        IS=\{\sum_{i=1}^nr_is_i|r_i\in I, s_i\in S, n\in \N^{*}\}
        $$
        then $IS$ is a submodule of $A$
        \item Let $A$ be an $\mathrm{R}$-module and $A_i,i\in I$ is a family of submodules of $\mathrm{A}$.Then $\cap_{i\in I}A_i$ is a submodule of $\mathrm{A}$
    \end{enumerate}
    \vspace{0.5cm}
    \begin{myDef}
        Let $\mathrm{R}$ be a ring, $\mathrm{A}$ a $\mathrm{R}$-module. X is a nonempty set of $\mathrm{A}$. $\mathbf{A\;submodule\;generated\\by\;X}$ is the intersection of all submodules that 
        contains $\mathrm{X}$.Let B is the submodule generated by X. If X is finite, then B is called $\mathbf{finitely\;generated}$; If $X=\{a\}$,then B is called $\mathbf{cyclic\;submodule}$.Let $B_i,i\in I$
        be a family of submodules of A, the submodule generated by $\cup_{i\in I}B_i$ is called the $\mathbf{sum}$ of submodules $B_i,i\in I$.
    \end{myDef}
    \textbf{REMARK} Submodule generated by $X$ is the smallest submodule that contains $X$. In other words, Let $B$ be the submodule of $A$ generated by $X$ and $C$ is any submodule of $A$ that contains $X$, we must have: $B\subset C$.\\
    To prove this, we only need to notice that $B=\cap_{X\subset C}C$. For any submodule that contains $X$, it must on the right side.
    \vspace{0.5cm}
    \begin{myTheo}
        Let $\mathrm{R}$ be a ring, $\mathrm{A}$ an $\mathrm{R}$-module, $\mathrm{X}$ a subset of $\mathrm{A}$, $\{B_i\;|\;i\in I\}$ a family of submodules of $\mathrm{A}$ and $a\in \mathrm{A}$. Let $\mathrm{Ra}=\{ra\;|\;r\in R\}$.
        \begin{enumerate}[(i)]
            \item $\mathrm{Ra}$ is a submodule of $\mathrm{A}$
            \item The cyclic submodule C generated by $\{a\}$ is $\{ra+na\;|\; r\in R, n\in \Z\}$
            \item The submodule generated by $\mathrm{X}$ is \\
            $$
            \{\sum_{i=1}^nr_ia_i+\sum_{j=1}^ms_jb_j\;|\; r_i\in \mathrm{R}, n,m\in \N^*,a_i,b_j\in X,s_j\in \Z\}
            $$
        \end{enumerate}
    \end{myTheo}
    \begin{proof}
        \begin{enumerate}[(i)]
            \item $\forall ra,sa\in \mathrm{Ra},\forall t\in\mathrm{R}$,we have:
            $$
            ra-sa=(r-s)a\in \mathrm{Ra},\;t(sa)=(ts)a\in \mathrm{Ra}
            $$
            According to the definiton of submodule, $\mathrm{Ra}$ is a submodule of A.
            \item First we need to show that $C=\{ra+na\;|\;r\in R, n\in \Z\}$ itself is a submoduel of A. The reason is as follows: $\forall r_1,r_2,s\in \mathrm{R},n_1,n_2\in \Z$:
            $$
            \begin{aligned}
            &(r_1a+n_1a)-(r_2a+n_2a)=(r_1-r_2)a+(n_1-n_2)a\in C\\
            &s(r_1a+n_1a)=(sr_1)a+s(n_1a)=(sr_1)a+(n_1s)a=(sr_1+n_1s)a\in C
            \end{aligned}
            $$ 
            Hence $C$ is a submodule of $\mathrm{A}$ that contains $\{a\}$. Besides, for any submodule $B_i$ that contains $\{a\}$, it's obviously $ra\in B_i,r\in R$ and $na\in B_i,n\in \Z$. Hence $C\subset B_i$.\\
            Let $B$ be the submodule generated by $X$. Then $B=\cap_{i\in I}B_i$, $B\subset C$ because $C$ is a submodule of $A$ contains $X$, hence one of $B_i$. $C\subset B$ is trivial since $C\subset B_i$ hence $C\subset \cap_{i\in I} B_i=B$. Therefore, $B=C$. 
            
            \item The method to proving (iii) is the same as the method used in (ii).
        \end{enumerate}
    \end{proof}
    \noindent
    \textbf{REMARK} In Theorem 1(ii), if $R$ is a ring with identity and $C$ an unitary module over $R$. The submodule generated by $\{a\}$ is $Ra$ as $na=(n1_R)a,n1_R\in R$.
    \vspace{0.5cm}
    \begin{myTheo}
        Let $\mathrm{B}$ be a submodule of a module $\mathrm{A}$ over a ring $\mathrm{R}$. Then the quotient group $\mathrm{A/B}$ is an $\mathrm{R}$-module with the action of $\mathrm{R}$ on $\mathrm{A/B}$ given by:
        $$
        \mathrm{r(a+B)=ra+B}
        $$ 
    The map $\pi:\mathrm{A}\rightarrow \mathrm{A/B}$ given by $\mathrm{a}\mapsto\mathrm{a+B}$ is an $\mathrm{R}$-module epimorphism with $\ker\pi=\mathrm{B}$
    \begin{proof}
       First, we will show the ring acts on $A/B$ is well-defined: Let $a+B=a'+B$, hence $a-a'\in B$.For any $r\in R$, we have $r(a-a')\in B$ as $B$ is a submodule of $A$.Hence we have $ra+B=ra'+B$,which means 
       $r(a+B)=r(a'+B)$. Therefore the action is well-defined.\\

       Second, we will show $A/B$ is an R-module with action given above. $A/B$ is iteself an abelian group. For any $r,s\in R$, we have: 
       $$(r+s)(b+B)=(r+s)b+B=(rb+sb)+B=(rb+B)+(sb+B)=r(b+B)+s(b+B)$$
       For any $a+B,b+B\in A/B,r\in R$,we have:
       $$
        r((a+B)+(b+B))=r(a+b+B)=r(a+b)+B=(ra+rb)+B=(ra+B)+(rb+B)=r(a+B)+r(b+B)
       $$  
       The associative law is easy to prove.Thus $A/B$ is an $R$-module with action given above.
    \end{proof}
    \end{myTheo}
    \vspace{0.5cm}   
    \begin{myTheo}
       (isomorphism theorems)
       \begin{enumerate}[(i)]
           \item Let $A,B$ be $R$-module and $f:A\rightarrow B$ an $R$-module homomorphism.Then we have:\\
           $$
           A/\ker f\cong \mathrm{Im}f
           $$ 
           If $f$ is an epimorphism then $A/\ker \cong B$
           \item Let $B$ and $C$ be submodules of a module $A$ over a ring $R$.Then:
           $$
           B/(B\cap C)\cong(B+C)/C
           $$
           \item Let $B$ and $C$ be submodules of a module $A$ over a ring $R$.If $C\subset B$,then $B/C$ is a submodule of $A/C$,and:
           $$
           (A/C)/(B/C)\cong A/B
           $$
       \end{enumerate} 
       \begin{proof}
           Proofs of the theorem is the same as those in the condtions of group and ring.
       \end{proof}
    \end{myTheo}
    \vspace{0.5cm}
    \begin{myTheo}
        Let $R$ be a ring and $\{A_i\;|\;i\in I\}$ a nonempty family of $R$-modules,$\prod_{i\in I}A_i$ the direct product of the abelian group
        $A_i$ and $\sum_{i\in I}$ the direct sum of the abelian group $A_i$.
        \begin{enumerate}[(i)]
            \item $\prod_{i\in I}A_i$ is an $R$-module with the action of $R$ given by $r\{a_i\}=\{ra_i\}$
            \item $\sum_{i\in I}A_i$ is a submodule of $\prod_{i\in I}A_i$
            \item For each $k\in I$, the canonical projection $\pi_k:\prod A_i\rightarrow A_k$ is an $R$-module epimorphism.
            \item For each $k\in I$, the canonical injection $\iota_k:A_k\rightarrow \sum A_i$ is an $R$-module monomorphism.
        \end{enumerate}
    \end{myTheo}
    \begin{proof}
        (i) $\prod_{i\in I}A_i$ is itself an abelian group.For any $r,s\in R$, $\{a_i\},\{b_i\}\in \prod_{i\in I}A_i$ we have:
        $$
        \begin{aligned}
            r(\{a_i\}+\{b_i\})&=r(\{a_i+b_i\})=\{r(a_i+b_i)\}\\&=\{ra_i+rb_i\}=\{ra_i\}+\{rb_i\}\;\;(\mbox{by definition of plus in direct product})\\&=r\{a_i\}+r\{b_i\}\\
            (r+s)\{a_i\} &=\{(r+s)a_i\}=\{ra_i+sa_i\}\\&=\{ra_i\}+\{sa_i\}\\&=r\{a_i\}+s\{a_i\}\\
            (rs)\{a_i\}&=\{(rs)a_i\}=\{r(sa_i)\}=r\{sa_i\}=r(s\{a_i\})
        \end{aligned}
        $$
        Thus $\prod_{i \in I} A_i$ is an $R$-module.\\
        \vspace{0.2cm}\\
        (ii) $\sum_{i\in I}A_i$ consists of those elements $\{a_i\}$ with only finite number of $a_k$ are not $0_{A_k}$. Thus $\sum_{i\in I}A_i$ is obviously a subset of $\prod_{i\in I}A_i$.
        For any $\{a_i\},\{b_i\}\in \sum_{i\in I}A_i$:
        $$
        \{a_i\}-\{b_i\}=\{a_i-b_i\}
        $$
        It's trivial that $\{a_i-b_i\}$ has at most $n_1+n_2$ elements are not $0$, where $n_1$ is the number of elements in $\{a_i\}$ that are not $0$ and similar for $n_2$.Hence $\{a_i-b_i\}\in \sum_{i\in I}A_i$ \\
        For any $r\in R$, we have:\\
        $$
        r\{a_i\}=\{ra_i\}
        $$
        $\{ra_i\}$ has the same number of non-zero elements as $\{a_i\}$ does.Hence $\{ra_i\}\in \sum_{i\in I}A_i$. Therefore, $\sum_{i\in I}A_i$ is a submodule of $\prod_{i\in I}A_i$.\\
        \vspace{0.2cm}\\
        (iii) Canonical projection $\pi_k:\prod_{i\in I}A_i\rightarrow A_k,\{a_i\}\mapsto a_k$ satifies:\\
        $$
        \begin{aligned}
        &\pi_k(\{a_i\}+\{b_i\})=\pi_k(\{a_i+b_i\})=a_k+b_k=\pi_k(\{a_i\})+\pi_k(\{b_i\})\\
        &\pi_k(r\{a_i\})=\pi_k(\{ra_i\})=(ra_i)_k=ra_k=r\pi_k(\{a_i\})
        \end{aligned}
        $$
        Thus $\pi_k$ is an $R$-module homomorphism. It's obviously that $\pi_k$ is epimorphism since for each $a_k\in A_k$, we have:$\pi_k(\mathbf{a_k'})=a_k$ where $\mathbf{a_k'}$ is the element with only the $k^{th}$ element is $a_k$ and others are 0.\\
        \vspace{0.2cm}\\
        (iv) Canonical injection $\iota_k:A_k\rightarrow \sum_{i\in I}A_i,a_k\mapsto \mathbf{a_k}$ where $\mathbf{a_k}$ is the element with $k^{th}$ element is $a_k$ and others are 0. $\iota_k$ is easily to be proved as an $R$-module 
        homomorphism. And it's trivial that $\ker \iota_k={0_{A_k}}$. Therefore $\iota_k$ is monomorphism.
    \end{proof}
    \vspace{0.5cm}
    \begin{myTheo}
        If $\mathrm{R}$ is a ring, $\{A_i\;|\;i\in I\}$ a family of $\mathrm{R}$-modules,$\mathrm{C}$ an $\mathrm{R}$-module,and $\{\phi_i:\mathrm{C}\rightarrow \mathrm{A}_i\;|\;i\in I\}$ a family of $\mathrm{R}$-module homomorphisms, then there is a unique
        $\mathrm{R}$-module homomorphism $\phi:\mathrm{C}\rightarrow \prod_{i\in I}A_i$ such that $\pi_i\phi=\phi_i,\forall i\in I$.Hence $\prod_{i\in I}A_i$ is the product of $\{A_i\;|\;i\in I\}$ in the category of $\mathrm{R}$-modules.
    \end{myTheo}
    \begin{proof}
        The $\mathrm{R}$-module homomorphism is easy to see:
        $$
        \phi:\mathrm{C}\rightarrow \prod_{i\in I}A_i,c\mapsto \{\phi_i(c)\}_{i\in I}
        $$
        $\phi$ is easy to be proved as an $\mathrm{R}$-module homomorphism. Hence we have:$\pi_k\phi(c)=\pi_k(\{\phi_i(c)\}_{i\in I})=phi_k(c),c\in \mathrm{C},k\in I$. Thus we have $\pi_i\phi=\phi_i,i\in I$.\\
        To prove the uniqueness of $\phi$, let $f$ be another $\mathrm{R}$-module homomorphism $f:C\rightarrow \prod_{i\in I}A_i$ with $\pi_if=\phi_i,i\in I$.We need to prove that $\phi=f$. If there is some $c\in\mathrm{C}$ such that $f(c)\neq\phi(c)$,then $f(c)$ and $\phi(c)$ 
        have at lease one position with different elements,let's say the $k^{th}$ element. Then we have:$\pi_k(\phi(c))\neq\pi_k(f(c))$, which means $\phi_k(c)\neq \phi_k(c)$. This is obviously not gonna happen. Therefore we must have $\phi=f$.
    \end{proof}
    \vspace{0.5cm}
    \begin{myTheo}
        If $\mathrm{R}$ is a ring, $\{A_i\;|\;i\in I\}$ a family of $\mathrm{R}$-modules,$\mathrm{D}$ an $\mathrm{R}$-module,and $\{\psi_i:A_i\rightarrow \mathrm{D}\;|\;i\in I\}$ a family of $\mathrm{R}$-module homomorphisms, then there is a unique
        $\mathrm{R}$-module homomorphism $\psi:\sum_{i\in I}A_i\rightarrow \mathrm{D}$ such that $\psi\iota_i=\psi_i,\forall i\in I$.Hence $\prod_{i\in I}A_i$ is the coproduct of $\{A_i\;|\;i\in I\}$ in the category of $\mathrm{R}$-modules.
    \end{myTheo}
    \begin{proof}
        The $\mathrm{R}$-module homomorphism $\psi$ is easy to see:
        $$
        \psi: \sum_{i\in I}A_i\rightarrow D,\{a_i\}_{i\in I}\mapsto \sum_{i\in I}\psi_i(a_i)
        $$
        Here $\sum_{i\in I}\psi(a_i)$ means we add finite many nonzero elements together. $\psi$ is easy to be seen as an $\mathrm{R}$-module homomorphism. And it's easy to prove that $\psi\iota_i=\psi_i$.\\
        To prove the uniqueness of $\psi$, let $f$ be another $\mathrm{R}$-module homomorphism with $f\iota_i=\psi_i$. Then for any $\{a_i\}\in \sum_{i\in I}A_i$, we have:
        $$
        f(\{a_i\})=f(\sum_{i\in I}\mathbf{a_i})=f(\sum_{i\in I}\iota_i(a_i))=\sum_{i\in I}(f\iota_i)(a_i)=\sum_{i\in I}\psi_i(a_i)
        $$
        Thus $f=\psi$.We have proved the uniqueness of $\psi$
    \end{proof}
    \vspace{0.5cm}
    \begin{myTheo}
        Let $\mathrm{R}$ be a ring and $\mathrm{A_1,A_2,...,A_n}$ $\mathrm{R}$-modules. Then $\mathrm{A}\cong\mathrm{A_1}\bigoplus\mathrm{A_2}\bigoplus...\bigoplus\mathrm{A_n}$ if and only if for each $i=1,2,...,n$ there are $\mathrm{R}$-module homomorphism $\pi_i:\mathrm{A}\rightarrow \mathrm{A_i}$
        and $\iota_i:\mathrm{A}_i\rightarrow \mathrm{A}$ such that:\\
        \vspace{0.05cm}\\
        $\mathrm{(i)}$ $\pi_i\iota_i = 1_{A_i}$ for $i=1,2,...,n$\\
        $\mathrm{(ii)}$ $\pi_j\iota_i=0$ for $j\neq i$\\
        $\mathrm{(iii)}$ $\iota_1\pi_1 + \iota_2\pi_2 +...+\iota_n\pi_n=1_A$
    \end{myTheo}
    \begin{proof}
        ($\Rightarrow$) If $\mathrm{A}\cong \mathrm{A_1}\bigoplus\mathrm{A_2}\bigoplus...\bigoplus\mathrm{A_n}$, let $\pi_i,\iota_i$ be the canonical projection and injection. It's easy to prove
        that $\pi_i,\iota_i$ satify condtions(i)-(iii)\\

        ($\Leftarrow$) If $\pi_i,\iota_i$ satisfy (i)-(iii). Let $\pi_i',\iota_i'$ be the canonical projection and injection between
         $\mathrm{A_1}\bigoplus\mathrm{A_2}\bigoplus...\bigoplus \mathrm{A_n}$ and $\mathrm{A_i}$.Let $\phi:A_1\bigoplus A_2\bigoplus...\bigoplus A_n \rightarrow A$ be given by $\phi=\iota_1\pi_1'+\iota_2\pi_2'+...+\iota_n\pi_n'$ and $psi:A\rightarrow A_1\bigoplus A_2\bigoplus...\bigoplus A_n$ by 
         $\psi=\iota_1'\pi_1+\iota_2'\pi_2+...+\iota_n'\pi_n$. Then it's easy to verify that $\phi\psi=1_A$ and $\psi\phi=1_{A_1\bigoplus A_2\bigoplus...\bigoplus A_n}$. Therefore $A\cong A_1\bigoplus A_2\bigoplus...\bigoplus A_n$.
    \end{proof}
\end{document}