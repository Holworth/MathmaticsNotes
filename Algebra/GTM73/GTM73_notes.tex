\documentclass[a4paper]{article}
\usepackage{amsmath,amsthm,amssymb,amsfonts}
\usepackage{graphics}
\usepackage{tikz-cd} 
\newtheorem{myDef}{Definition} 
\newtheorem{myTheo}{Theorem}
\newtheorem{myCol}{Corollary}
%% Sets page size and margins
\usepackage[a4paper,top=3cm,bottom=2cm,left=3cm,right=3cm,marginparwidth=1.75cm]{geometry}
\newcommand{\R}{\mathbb{R}}
\newcommand{\N}{\mathbb{N}}
\newcommand{\Z}{\mathbb{Z}}
\providecommand{\C}{\mathbb{C}}

\title{Projective and Injective Modules}
\begin{document}
    \maketitle
    \section*{1.Projective Modules}
    \begin{myDef}
        A module P over a ring R is said to be $\textbf{projective}$, if given any diagram of R-module 
        homomorphisms\\
        \begin{center}
            \begin{tikzcd}[column sep=0.7cm, row sep=1.6cm]
                & P \arrow[d, "f"]\\ 
            A  \arrow[r, "g"] & B \arrow[r]& 0
        \end{tikzcd}\\
        \end{center}
        with the bottom row exact, there exists an R-module homomorphism $h:P\rightarrow A$ such that the 
        diagram\\
        \begin{center}
            \begin{tikzcd}[column sep=0.7cm, row sep=1.6cm]
                & P \arrow[d, "f"] \arrow[ld, "h"]\\ 
            A  \arrow[r, "g"] & B \arrow[r]& 0
        \end{tikzcd}\\
        \end{center}
        is commutative.
    \end{myDef}
    \noindent
    \textbf{note} The bottom row exact means homomorphism $g$ is an epimorphism. An R-module P is said to be projective
     means for any epimorphism $g:A\rightarrow B$ and R-module homomorphism $f:P\rightarrow B$ there exists $h:P\rightarrow A$
     such that:$f=gh$
    \begin{myTheo}
        Every free module $\mathrm{F}$ over a ring $\mathrm{R}$ with identity is projective.
        \begin{proof}
            Let F be a free module over ring R and F is free on set X. Let $\iota:X\rightarrow F$.
            For any pair of R-module A,B with $g:A\rightarrow B$ an epimorphism and $f:F\rightarrow B$ a homomorphism\\
        \begin{center}
            \begin{tikzcd}[column sep=0.7cm, row sep=1.6cm]
                & F \arrow[d, "f"] \\ 
            A  \arrow[r, "g"] & B \arrow[r]& 0
            \end{tikzcd}
        \end{center}
        $f(\iota(x))\in B, x\in X$. Since $g$ is an epimorphism, there exists some $a_x\in A$ such that $g(a_x)=f(\iota(x))$.Let
        $h': X\rightarrow A, x\mapsto a_x$ then $h'$ induces a homomorphism $h:F\rightarrow A$.Hence we have: $h(\iota(x))=a_x$ and
        $g(h(\iota(x))) = g(a_x)=f(\iota(x))$.This holds for any $x\in X$. Hence $gh\iota=f\iota$.We have following diagram\\
        \begin{center}
            \begin{tikzcd}
                X \arrow[r, "\iota"] \arrow[rd, "\phi"]& F\arrow[d, "f"] \\
                  & A
            \end{tikzcd}
        \end{center}
        According to the property of free module: For any homomorphism $\phi:X\rightarrow A$,there exists unique homomorphism $f:F\rightarrow A$
        such that $\phi=f\iota$. Here let $\phi=f\iota=gh\iota$. We have: $f=gh$
        \end{proof}
    \end{myTheo}

    \begin{myCol}
       Every module A over a ring R is the homomorphism image of a projective R-module 
       \begin{proof}
           Every module A(over R) is homomorphism image of a free module over R. By $\mathrm{Theorem 1}$, every free module is a projective module.
       \end{proof}
    \end{myCol}
\end{document}



